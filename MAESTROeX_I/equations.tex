\section{Governing Equations}
Low Mach number models for reacting flow were originally derived using asymptotic analysis
\citep{rehm1978equations,majda1985derivation} and used in terrestrial combustion applications
\citep{knio1999semi,day2000numerical}.  These models have been extended to nuclear flames
in astrophysical environments using adaptive algorithms in space and time \citep{Bell:2004}.
In Papers I-III, we derived a model and algorithm suitable for stratified astrophysical flow.
In our model, we take the standard equations of reacting, compressible flow, and recast the equation
of state (EOS) as a divergence constraint on the velocity field.
The resulting model is a series of evolution equations for mass, momentum, and energy, subject
to these constraints:
\begin{eqnarray}
\frac{\partial\Ub}{\partial t} &=& -\Ub\cdot\nabla\Ub  - \frac{1}{\rho}\nabla\pi - \frac{\rho-\rho_0}{\rho} g\eb_r,\label{eq:momentum}\\
\frac{\partial(\rho X_k)}{\partial t} &=& -\nabla\cdot(\rho X_k\Ub) + \rho\omegadot_k,\label{eq:species}\\
\frac{\partial(\rho h)}{\partial t} &=& -\nabla\cdot(\rho h\Ub) + \frac{Dp_0}{Dt} + \rho\Hnuc,\label{eq:enthalpy}
\end{eqnarray}
\begin{equation}
\nabla\cdot(\beta_0\Ub) = \beta_0\left(S - \frac{1}{\gammaonebar p_0}\frac{\partial p_0}{\partial t}\right).\label{eq:U divergence}
\end{equation}
Here $\rho$, $\Ub$, and $h$ are the mass density,
velocity and specific enthalpy, respectively, and
$X_k$ are the mass fractions of species $k$ with associated
production rate $\omegadot_k$.  The species are constrained
such that $\sum_k X_k = 1$ giving $\rho = \sum_k (\rho X_k)$ and
\begin{equation}
\frac{\partial\rho}{\partial t} = -\nabla\cdot(\rho\Ub).
\end{equation}
Here $\Hnuc$ is the nuclear energy generation rate per unit mass.
The pressure is decomposed into a hydrostatic base state
 pressure, $p_0 = p_0(r,t)$, and a dynamic pressure, $\pi = \pi(\xb,t)$, such that 
$|\pi|/p_0 = \mathcal{O}(M^2)$ (we use $\xb$ to represent the Cartesian coordinate 
directions of the full state and $r$ to represent the radial coordinate direction for 
the base state).  We also define a base state density, $\rho_0 = \rho_0(r,t)$, 
which is in hydrostatic equilibrium with $p_0$, i.e., 
$\nabla p_0 = -\rho_0 g\eb_r$, where $g=g(r,t)$ is
the magnitude of the gravitational acceleration and $\eb_r$ is the unit vector in the
outward radial direction. 

Mathematically, equations (\ref{eq:momentum})-(\ref{eq:enthalpy}) must still be closed by an equation of state.  
This is done by taking the Langrangian derivative of the equation of state for pressure as a function of the density, composition, and enthalpy, requiring that the pressure is a prescribed function of altitude and time based on the hydrostatic equilibrium condition.
Details are given in Papers I and II.
After substituting the equations for mass and energy into the Lagrangian derivative, the result
is a divergence constraint on the velocity field (\ref{eq:U divergence}),
where $\beta_0$ is a density-like variable that carries background stratification, defined as
\begin{equation}
\beta_0(r,t) = \rho_0(0,t)\exp\left(\int_0^r\frac{1}{\gammaonebar p_0}\frac{\partial p_0}{\partial r'}dr'\right),
\end{equation}
where $\gammaonebar$ is the lateral average of $\Gamma_1 = d(\log p)/d(\log\rho)$ at constant entropy.
The expansion term, $S$, incorporates local compressibility effects due to heat release from reactions, compositional changes, and external sources,
\begin{equation}
S = -\sigma\sum_k\xi_k\omegadot_k + \frac{1}{\rho p_\rho}\sum_k p_{X_k}\omegadot_k + \sigma\Hnuc,\label{eq:S}
\end{equation}
where $p_{X_k} \equiv \left. \partial p / \partial X_k
\right|_{\rho,T,X_{j,j\ne k}}$, $\xi_k \equiv \left. \partial h /
\partial X_k \right |_{p,T,X_{j,j\ne k}},
p_\rho \equiv \left.
\partial p/\partial \rho \right |_{T, X_k}$, and $\sigma \equiv
p_T/(\rho c_p p_\rho)$, with $p_T \equiv \left. \partial p / \partial
T \right|_{\rho, X_k}$ and $c_p \equiv \left.  \partial h / \partial T
\right|_{p,X_k}$ is the specific heat at constant pressure.
