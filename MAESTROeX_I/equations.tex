\section{Governing Equations}\label{sec:equations}
Low Mach number models for reacting flow were originally derived using asymptotic analysis
\citep{rehm1978equations,majda1985derivation} and used in terrestrial combustion applications
\citep{knio1999semi,day2000numerical}.  These models have been extended to nuclear flames
in astrophysical environments using adaptive algorithms in space and time \citep{Bell:2004}.
In Papers I-III, we extended this work and the atmospheric model by \citet{durran:1989} by deriving a model and algorithm suitable for stratified astrophysical flow.
In our model, we take the standard equations of reacting, compressible flow, and recast the equation
of state (EOS) as a divergence constraint on the velocity field.
The resulting model is a series of evolution equations for mass, momentum, and energy, subject
to to an additional constraint on velocity.  The evolution equations are
\begin{eqnarray}
\frac{\partial(\rho X_k)}{\partial t} &=& -\nabla\cdot(\rho X_k\Ub) + \rho\omegadot_k,\label{eq:species}\\
\frac{\partial\Ub}{\partial t} &=& -\Ub\cdot\nabla\Ub  - \frac{\beta_0}{\rho}\nabla\left(\frac{\pi}{\beta_0}\right) - \frac{\rho-\rho_0}{\rho} g\eb_r,\label{eq:momentum}\\
\frac{\partial(\rho h)}{\partial t} &=& -\nabla\cdot(\rho h\Ub) + \frac{Dp_0}{Dt} + \rho\Hnuc.\label{eq:enthalpy}
\end{eqnarray}
Here $\rho$, $\Ub$, and $h$ are the mass density,  
velocity and specific enthalpy, respectively, and
$X_k$ are the mass fractions of species $k$ with associated
production rate $\omegadot_k$ and energy release per time per unit mass $\Hnuc$.
The species are constrained such that $\sum_k X_k = 1$ giving $\rho = \sum_k (\rho X_k)$ and
\begin{equation}
\frac{\partial\rho}{\partial t} = -\nabla\cdot(\rho\Ub).
\end{equation}
Note that the momentum equation above enforces conservation of total energy in the 
low Mach number system in the absence of external heating or viscous terms \citep{kleinpauluis,Vasil2013}, 
and has been previously demonstrated in modeling sub-Chandrasekhar white dwarfs using MAESTRO \citep{subChandra_II}. 
The total pressure is decomposed into a one-dimensional hydrostatic base state
 pressure, $p_0 = p_0(r,t)$, and a dynamic pressure, $\pi = \pi(\xb,t)$, such that 
$p = p_0 + \pi$ and $|\pi|/p_0 = \mathcal{O}({\rm Ma}^2)$ (we use $\xb$ to represent the Cartesian coordinate 
directions of the full state and $r$ to represent the radial coordinate direction for 
the base state).  We also define a one-dimensional base state density, $\rho_0 = \rho_0(r,t)$, that represents the lateral average (see Section \ref{Sec:Spatial}) of $\rho$ and is in hydrostatic equilibrium with $p_0$, i.e., 
\begin{equation}
\nabla p_0 = -\rho_0 g\eb_r, \label{eq:HSE}
\end{equation}
where $g=g(r,t)$ is the magnitude of the gravitational acceleration and $\eb_r$ is the unit vector in the outward radial direction. 
Thermal diffusion is not discussed in this paper, but we have previously described the modifications to the original algorithm required 
for implicit thermal diffusion in \cite{XRB_I}; inclusion of these effects in the new algorithm presented here is straightforward.

Mathematically, equations \ref{eq:momentum}--\ref{eq:enthalpy} must still be closed by the EOS.  
This is done by taking the Langrangian derivative of the EOS for pressure as a function of the thermodynamic variables, 
substituting in the equations of motion for mass and energy,
and requiring that the pressure is a prescribed function of altitude and time based on the hydrostatic equilibrium condition.
See Papers I and II for details of this derivation.
The resulting equation is a divergence constraint on the velocity field,
\begin{equation}
\nabla\cdot(\beta_0\Ub) = \beta_0\left(S - \frac{1}{\gammaonebar p_0}\frac{\partial p_0}{\partial t}\right).\label{eq:U divergence}
\end{equation}
Here $\beta_0$ is a density-like variable that carries background stratification, defined as
\begin{equation}
\beta_0(r,t) = \rho_0(0,t)\exp\left(\int_0^r\frac{1}{\gammaonebar p_0}\frac{\partial p_0}{\partial r'}dr'\right),
\end{equation}
where $\gammaonebar$ is the lateral average of $\Gamma_1 = d(\log p)/d(\log\rho) |_s$ (evaluated with entropy, $s$, constant).
The expansion term, $S$, incorporates local compressibility effects due to compositional changes and heat release from reactions,
\begin{equation}
S = -\sigma\sum_k\xi_k\omegadot_k + \frac{1}{\rho p_\rho}\sum_k p_{X_k}\omegadot_k + \sigma\Hnuc,\label{eq:S}
\end{equation}
where 
$p_{X_k} \equiv \left. \partial p / \partial X_k \right|_{\rho,T,X_{j,j\ne k}}$, 
$\xi_k \equiv \left. \partial h /\partial X_k \right |_{p,T,X_{j,j\ne k}}$,
$p_\rho \equiv \left.\partial p/\partial \rho \right |_{T, X_k}$, and 
$\sigma \equiv p_T/(\rho c_p p_\rho)$, with $p_T \equiv \left. \partial p / \partial
T \right|_{\rho, X_k}$ and $c_p \equiv \left.  \partial h / \partial T
\right|_{p,X_k}$ is the specific heat at constant pressure.

To summarize, we model evolution equations for momentum, mass, and energy, equations \ref{eq:momentum}--\ref{eq:enthalpy} subject to a divergence constraint on the velocity, equation \ref{eq:U divergence}, and the hydrostatic equilibrium condition, equation \ref{eq:HSE}.
