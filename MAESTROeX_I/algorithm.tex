\section{Numerical Algorithm}\label{eq:algorithm}

\subsection{Spatial Discretization}\label{Sec:Spatial}
\MarginPar{needs work}
MAESTRO contains variables defined on Cartesian grids 
one-dimensional base state stored on radial arrays.

 associated with square (in 2D) or cubic (in 3D)


(description of base state mapping in planar, and original spherical, and new spherical)

Mapping from the base state to the full state

Mapping a Cartesian variable to a one-dimensional radial array uses the lateral averaging operator.

One of the key numerical modules is the ``lateral average'', which computes the average over a 
layer of a Cartesian grid variable onto a one-dimensional radial array.
In planar geometries, this is a straightforward arithmetic average of cells at
a particular height since the radial cell centers are in alignment
with the Cartesian grid cell centers.
However for spherical problems, the procedure is much more complicated.
In Section 4 of Paper V, we describe how there is a finite, easily computable set of radii that any Cartesian cell-center can map to.  
Specifically, for every Cartesian cell, there exists an integer $m$ such that the radius to the center of the star is given by $\hat{r}_m=\Delta x\sqrt{0.75+2m}$.
We average by binning all the cells associated with each radii and taking the average.
Then we interpolate this data onto a one-dimensional radial array.  Previously, MAESTRO only allowed for radial arrays with a constant $\Delta r$ (typically equal to $\Delta x/5$).
Here we present a new option to retain an irregularly-spaced radial array to eliminate mapping errors back onto the Cartesian grid state.
Consider a spherical star in hydrostatic equlibrium at rest.  In the absense of reactions, the star should remain at rest.
The buoyancy forcing term in the momentum equation contains $\rho-\rho_0$.  With the original scheme, interpolation errors in computing $\rho_0$ by averaging would cause artificial acceleration in the velocity field due to the interpolation error from the Cartesian grid to and from the radial base state.  By retaining the radial base state as an irregularly spaced array, the effects due to interpolation error are completely eliminated.


\subsection{Temporal Integration Scheme}\label{Sec:Temporal Integration Scheme}
Previously we adopted an approach where we split the velocity into a base state component, $w_0(r,t)$, 
and a local velocity $\Ubt(\xb,t)$, so that
\begin{equation}
\Ub = \Ubt(\xb,t) + w_0(r,t)\eb_r.
\end{equation}
We used $w_0$ to provide an estimate of the base state density evolution over a time step.
This resulted in some unnecessary complications to the temporal integration scheme.
Our new temporal integration scheme does not use this splitting, and is much simpler than the scheme from Paper V since we evolve the velocity directly rather than a more complex evolution equation for the perturbational velocity.
Additionally, the new scheme uses a simpler predictor-corrector approach to the base state density and pressure that no longer requires evolution equations and the associated discretizations for the base state, greatly simplifying the algorithm while retaining the same overall second-order accuracy in time.

We now describe the new temporal integration scheme, which we note can be used for the original base state mapping (with constant base state grid spacing) as well as the new irregularly spaced base state mapping.
At the beginning of each time step we have the cell-centered state,
$(\Ub,\rho X_k,\rho h,\rho_0,p_0)^n$, and nodal state, $\pi^{n-\myhalf}$.
At any time, the associated density, composition, and enthalpy can be trivially computed using, e.g.,
\begin{equation}
\rho^n = \sum_k(\rho X_k)^n, \quad
X_k^n = (\rho X_k)^n / \rho^n, \quad
h^n = (\rho h)^n / \rho^n.
\end{equation}
Temperature is computed using the equation of state, e.g.,
\footnote{As described in Paper V, for planar problems we compute temperature using $h$ instead of $p_0$, since we have successfully developed planar volume discrepancy schemes to effectively couple the enthalpy to the rest of the solution; see \cite{XRB_I}.  We are still exploring this option for spherical stars.}
\begin{equation}
T = T(\rho,p_0,X_k),
\end{equation}
and ($\gammaonebar,\beta_0)$ are computed from $(\rho,p_0,X_k)$ (see Appendix A of Paper I and Appendix C of Paper III for details).

The overall flow of the algorithm is a second-order Strang splitting approach for the coupling of the reactions and advection of the thermodynamic variables.  
We use a predictor-corrector approach within the Strang splitting scheme to achieve second-order accuracy in time.
We integrate the velocity using a standard second-order projection method to enforce the divergence constraint.
To summarize:
\begin{itemize}
\item In {\bf Step 1} we react the thermodynamic variables over $\Delta t/2$.
\item In {\bf Steps 2-4} we advect the thermodynamic variables over $\Delta t$.  Specifically, we compute an estimate for the expansion term, $S$, project the face-centered, time-centered velocities so they satisfy the divergence constraint, and then advect the thermodynamic variables.
\item In {\bf Step 5} we react the thermodynamic variables over $\Delta t/2$.\footnote{After this step we could skip to the velocity advance in {\bf Steps 10-11}, however the overall scheme would be only first-order in time, so {\bf Steps 6-9} can be thought of as a trapezoidal corrector step.}
\item In {\bf Steps 6-8} we redo the advection in {\bf Steps 2-4} but are able to use the trapezoidal rule to time-center certain quantities such as $S$, $\rho_0$, etc.
\item In {\bf Step 9} we redo the reactions from {\bf Step 5} using the improved results for the corrector advection step.
\item In {\bf Steps 10-11} we and update the velocity, compute the expansion term, $S$, and project the cell-centered velocities.
\end{itemize}

There are a few key numerical modules we use in each time step.
\begin{itemize}
\item {\bf Average}$[\phi]\rightarrow[\overline\phi]$ computes the lateral average of a quantity over a layer at constant radius $r$, as described above in Section \ref{Sec:Spatial}.
\item {\bf Enforce HSE}$[\rho_0]\rightarrow[p_0]$ computes the base state pressure, $p_0$, from a base state density, $\rho_0$ by integrating the hydrostatic equilibrium condition in one dimension; see equation (A10) in Paper V.  The base state pressure remains equal to a constant value from the location of a prescribed cutoff density outward for the entire simulation.
\MarginPar{add note about changes for irregular}
\item {\bf React State}$[(\rho X_k)^{\rm in},(\rho h)^{\rm in},p_0]\rightarrow[(\rho X_k)^{\rm out},(\rho h)^{\rm out},(\rho\dot\omega),(\rho\Hnuc)]$ integrates the species and enthalpy due to reactions over $\Delta t/2$ by solving
\begin{equation}
\frac{dX_k}{dt} = \dot\omega_k(\rho,X_k,T); \qquad
\frac{dT}{dt} = \frac{1}{c_p}\left(-\sum_k\xi_k\dot\omega_k + H_{\rm nuc})\right)
\end{equation}
The inputs are the species, enthalpy, and base state pressure, and the outpus are the species, enthalpy, reaction rates, and nuclear energy generation rate.
See Papers III and V for details.
\end{itemize}
Each time step is constrained by the standard advective CFL condition,
\begin{equation}
\Delta t = \sigma^{\rm CFL} \min_i(\Delta x / U_i),
\end{equation}
where for our simulations we typically use $\sigma^{\rm CFL}\sim 0.7$ and the minimum is taken over all spatial directions over all cells.
There are additional constraints on the timestep that are typically much less restrictive than the advective CFL including
the acceleration due to the buoyancy force (sometimes in effect when the velocity is approximately zero at the start of some simulations) 
and the local magnitude of the divergence constraint (to prevent too much mass evacuation from a cell in a time step); see Section 3.4 in Paper III for details.

In stratified low Mach number models, due to the extreme variation in density, the velocity can become very large in low density regions at the edge of the star.
Throughout Papers II-V, we have employed two techniques to help control these dynamics without significantly affecting the dynamics in the convective region.
First, we use a cutoff density technique, where we hold the density constant outside a specified radius (typically near where the density is $\sim$4 orders of magnitude smaller than the largest densities in the simulation.
Second, we employ a sponge technique where we artificially damp the velocities near and beyond the cutoff region.
For more details, refer to Paper V and the previous references cited within.

The temporal integration scheme contains the following steps:
\begin{description}

%--------------------------------------------------------------------------
% STEP 1
%--------------------------------------------------------------------------
\item[Step 1] {\em React the full state through the first time interval of $\dt / 2.$}

Call {\bf React State}$[(\rho X_k)^n, (\rho h)^n, p_0^n] \rightarrow [(\rho X_k)^{(1)}, (\rho h)^{(1)}, (\rho \omegadot_k)^{(1)}, (\rho \Hnuc)^{(1)}]$.

%--------------------------------------------------------------------------
% STEP 2
%--------------------------------------------------------------------------

\item[Step 2] {\em Compute the provisional time-centered expansion,
    $S^{\nph,\star}$.}

We compute an estimate for the time-centered expansion term in the velocity
divergence constraint.  Following \citet{Bell:2004}, we extrapolate
to the half-time using $S$ at the previous and current
time steps,
\begin{equation}
S^{\nph,\star} = S^n + \frac{\dt^n}{2} \left(\frac{S^n - S^{n-1}}{\dt^{n-1}}\right).
\end{equation}
Note that in the first time step we average $S^0$ and $S^1$ from the
initialization step.

%--------------------------------------------------------------------------
% STEP 3
%--------------------------------------------------------------------------
\item[Step 3] {\em Construct the provisional time-centered advective velocity on
edges, $\uadvone$.}

The construction of face-centered time-centered states used to discretize the
advection terms for velocity, species, and enthalpy, are performed using
a standard multidimensional corner transport upwind approach
\citep{colella1990multidimensional,saltzman1994unsplit} with piecewise-parabolic
one-dimensional tracing \citep{colella1984piecewise}.  The full details of this
Godunov advection approach for all steps in this algorithm are described 
in Appendix A of \cite{XRB_III}.

In this step, using equation (\ref{eq:momentum}), 
we compute time-centered edge velocities, $\uadvonedag$, using
$\Ub^n$.  The $\dagger$ superscript refers to the
fact that the predicted velocity field does not satisfy the divergence
constraint.  We then construct $\uadvone$ from $\uadvonedag$
using a MAC projection (see Appendix \ref{Sec:Projection} for details).
We note that $\uadvone$ satisfies the divergence constraint
\begin{equation}
\nabla \cdot \left(\beta_0^n \uadvone\right) = \beta_0^n \left[S^{\nph,\star} - \frac{1}{\gammaonebar^np_0^n}\left(\frac{\partial p_0}{\partial t}\right)^{n-\myhalf} \right].\label{eq:div1}
\end{equation}
In practice, for all the projections in the algorithm, we solve a split dynamics
system for a base state velocity, and a perturbational velocity, so that we can properly
enforce boundary conditions at the edge of the star.
Refer to Appendix \ref{Sec:Projection} for details.

%--------------------------------------------------------------------------
% STEP 4
%--------------------------------------------------------------------------
\item[Step 4] {\em Advect the full state through a time interval of $\dt.$}

\begin{enumerate}
\renewcommand{\theenumi}{{\bf \Alph{enumi}}}

\item Update $(\rho X_k)$ using a discretized version of
%
\begin{equation}
\frac{\partial(\rho X_k)}{\partial t} = -\nabla\cdot(\rho X_k\Ub) + \rho\omegadot_k,
\end{equation}
%
omitting the reaction terms, which were already
accounted for in {\bf React State}.  The update consists of two steps:

\begin{enumerate}
\renewcommand{\labelenumii}{{\bf \roman{enumii}}.}

\item Compute the time-centered species edge states, $(\rho X_k)^{\nph,\pred}$,
  for the conservative update of $(\rho X_k)^{(1)}$ using a Godunov approach \citep{XRB_III}.
  As described in Paper V, we predict $\rho'^n=\rho^n-\rho_0^n$ and $X_k^n$ to edges
  and here we spatially interpolate $\rho_0^n$ to edges to assemble the fluxes.

\item Evolve $(\rho X_k)^{(1)} \rightarrow (\rho X_k)^{(2),\star}$ using
\begin{equation}
(\rho X_k)^{(2),\star} = (\rho X_k)^{(1)}
  - \dt \left\{ \nabla \cdot \left[ \uadvone (\rho X_k)^{\nph,\pred} \right] \right\},
\end{equation}

\end{enumerate}

\item Update $\rho_0$ by calling {\bf Average}$[\rho^{(2),\star}]\rightarrow[\rho_0^{n+1,\star}]$.

\item Update $p_0$ by calling {\bf Enforce HSE}$[\rho_0^{n+1,\star}] \rightarrow [p_0^{n+1,\star}]$.

\item Update the enthalpy using a discretized version of equation
%
\begin{equation}
\frac{\partial(\rho h)}{\partial t} = -\nabla\cdot(\rho h\Ub) + \frac{Dp_0}{Dt} + \rho\Hnuc,
\end{equation}
%
again omitting the reaction terms since we already accounted for
them in {\bf React State}.  This equation takes the form:
\begin{equation}
\frac{\partial (\rho h)}{\partial t}  = - \nabla \cdot (\Ub \rho h) + \frac{\partial p_0}{\partial t} + (\Ub \cdot \eb_r) \frac{\partial p_0}{\partial r}.
\end{equation}
For spherical geometry, we solve the analytically equivalent form,
\begin{equation}
\frac{\partial (\rho h)}{\partial t}  = - \nabla \cdot (\Ub \rho h) + \frac{\partial p_0}{\partial t} + \nabla \cdot (\Ub p_0) - p_0 \nabla \cdot \Ub.
\end{equation}
The update consists of two steps:

\begin{enumerate}
\renewcommand{\labelenumii}{{\bf \roman{enumii}}.}

\item Compute the time-centered enthalpy edge state, $(\rho h)^{\nph,\pred},$
  for the conservative update of $(\rho h)^{(1)}$ using using a Godunov approach \citep{XRB_III}.
  As described in Paper V, we predict $(\rho h)'^n=(\rho h)^n-(\rho h)_0^n$ to edges,
  where $(\rho h)_0^n$ is obtained by calling {\bf Average}$[(\rho h)^n]\rightarrow[(\rho h)_0]$,
  and here we spatially interpolate $(\rho h)_0^n$ to edges to assemble the fluxes.

\item Evolve $(\rho h)^{(1)} \rightarrow (\rho h)^{(2),\star}$ using
\begin{equation}
(\rho h)^{(2),\star}
= (\rho h)^{(1)} - \dt \left\{ \nabla \cdot \left[ \uadvone (\rho h)^{\nph,\pred} \right] \right\} + \Delta t\frac{Dp_0}{Dt}
\end{equation}
where here
\begin{equation}
\frac{Dp_0}{Dt} =
\begin{cases}
\frac{p_0^{n+1,*} - p_0^n}{\Delta t} + \left(\uadvone \cdot \eb_r\right) \left(\frac{\partial p_0}{\partial r} \right)^{n}& {\rm (planar)} \\
\frac{p_0^{n+1,*} - p_0^n}{\Delta t} + \left[ \nabla \cdot \left (\uadvone p_0^{\nph} \right ) - p_0^{\nph} \nabla \cdot \uadvone \right]& {\rm (spherical)}
\end{cases}
,
\end{equation}
and $p_0^\nph = (p_0^n+p_0^{n+1,*})/2$.
\end{enumerate}
\end{enumerate}

%--------------------------------------------------------------------------
% STEP 5
%--------------------------------------------------------------------------
\item[Step 5] {\em React the full state through a second time interval of $\dt / 2.$}

Call {\bf React State}$[ (\rho X_k)^{(2),\star}, (\rho h)^{(2),\star}, p_0^{n+1,\star}] 
\rightarrow 
[ (\rho X_k)^{n+1,\star}, (\rho h)^{n+1,\star}, (\rho \omegadot_k)^{n+1,\star}, (\rho \Hnuc)^{n+1,\star} ].$

%--------------------------------------------------------------------------
% STEP 6
%--------------------------------------------------------------------------
\item[Step 6] {\em Compute the time-centered expansion, $S^{\nph,\star}$.}

First, compute $S^{n+1,\star}$ with
\begin{equation}
S^{n+1,\star} =  \left(-\sigma  \sum_k  \xi_k  \omegadot_k  + \frac{1}{\rho p_\rho} \sum_k p_{X_k}  {\omegadot}_k + \sigma \Hnuc\right)^{n+1,\star}.
\end{equation}
  Then, define
\begin{equation}
 S^{\nph} = \frac{S^n + S^{n+1,\star}}{2},
\end{equation}

%--------------------------------------------------------------------------
% STEP 7
%--------------------------------------------------------------------------
\item[Step 7] {\em Construct the time-centered advective velocity on edges, $\uadvtwo$.}

The procedure to construct $\uadvtwodag$ is identical to the Godunov procedure
for computing $\uadvonedag$ in {\bf Step 3}, but uses
the updated value $S^{\nph}$ rather than $S^{\nph,\star}$.
The $\dagger$ superscript refers to the
fact that the predicted velocity field does not satisfy the divergence
constraint.  We then construct $\uadvtwo$ from $\uadvtwodag$
using a MAC projection (see Appendix \ref{Sec:Projection} for details).
We note that $\uadvtwo$ satisfies the divergence constraint
\begin{equation}
\nabla \cdot \left(\beta_0^{\nph} \uadvtwo\right) =
\beta_0^{\nph} \left[S^{\nph} - \frac{1}{\gammaonebar^{\nph}p_0^{\nph}}\left(\frac{\partial p_0}{\partial t}\right)^{\nph}\right],\label{eq:div2}
\end{equation}
where
\begin{equation}
\beta_0^{\nph} = \frac{ \beta_0^n +  \beta_0^{n+1,\star} }{2},
\qquad
\gammaonebar^{\nph} = \frac{ \gammaonebar^n +  \gammaonebar^{n+1,\star} }{2}.
\end{equation}

%--------------------------------------------------------------------------
% STEP 8
%--------------------------------------------------------------------------
\item[Step 8] {\em Advect the base state and full state through a time interval of $\dt.$}

\begin{enumerate}
\renewcommand{\theenumi}{{\bf \Alph{enumi}}}

\item Update $(\rho X_k)$.  This step is identical to {\bf Step 4A} except we use
  the updated values $\uadvtwo$ and $\rho_0^{n+1,\star}$ rather than
  $\uadvone$ and $\rho_0^{n+1,\star}$.  In particular:

\begin{enumerate}
\renewcommand{\labelenumii}{{\bf \roman{enumii}}.}

\item Compute the time-centered species edge states, $(\rho X_k)^{\nph,\pred}$,
  for the conservative update of $(\rho X_k)^{(1)}$ using a Godunov approach \citep{XRB_III}.
  Again, we predict $\rho'^n=\rho^n-\rho_0^n$ and $X_k^n$ to edges
  and here we spatially interpolate $(\rho_0^n+\rho_0^{n+1,*})/2$ to edges to assemble the fluxes.


\item Evolve $(\rho X_k)^{(1)} \rightarrow (\rho X_k)^{(2)}$ using
\begin{equation}
(\rho X_k)^{(2)} = (\rho X_k)^{(1)}
- \dt \left\{ \nabla \cdot \left[\uadvtwo (\rho X_k)^{\nph,\pred} \right] \right\},
\end{equation}

\end{enumerate}

\item Update $\rho_0$ by calling {\bf Average}$[\rho^{(2)}]\rightarrow[\rho_0^{n+1}]$.

\item Update $p_0$ by calling {\bf Enforce HSE}$[\rho_0^{n+1}] \rightarrow [p_0^{n+1}]$.

\item Update the enthalpy.  This step is identical to {\bf Step 4D} except we use
  the updated values $\uadvtwo, \rho_0^{n+1}, (\rho h)_0^{n+1}$, and $p_0^{\nph}$
  rather than
  $\uadvone, \rho_0^{n+1,\star}, (\rho h)_0^{n+1,\star}$, and $p_0^n$.
  In particular:

\begin{enumerate}
\renewcommand{\labelenumii}{{\bf \roman{enumii}}.}

\item Compute the time-centered enthalpy edge state, $(\rho h)^{\nph,\pred},$
  for the conservative update of $(\rho h)^{(1)}$ using a Godunov approach \citep{XRB_III}.
  Again, we predict $(\rho h)'^n=(\rho h)^n-(\rho h)_0^n$ to edges
  and here we spatially interpolate $[(\rho h)_0^n)+(\rho h)_0^{n+1,*}]/2$ to edges to assemble the fluxes.

\item Evolve $(\rho h)^{(1)} \rightarrow (\rho h)^{(2)}$.
\begin{equation}
(\rho h)^{(2)}
= (\rho h)^{(1)} - \dt \left\{ \nabla \cdot \left[ \uadvtwo (\rho h)^{\nph,\pred} \right] \right\} + \Delta t\frac{Dp_0}{Dt}
\end{equation}
where here
\begin{equation}
\frac{Dp_0}{Dt} =
\begin{cases}
\frac{p_0^{n+1} - p_0^n}{\Delta t} + \left(\uadvtwo \cdot \eb_r\right) \left(\frac{\partial p_0}{\partial r} \right)^{n}& {\rm (planar)} \\
\frac{p_0^{n+1} - p_0^n}{\Delta t} + \left[ \nabla \cdot \left (\uadvtwo p_0^{\nph} \right ) - p_0^{\nph} \nabla \cdot \uadvtwo \right]& {\rm (spherical)}
\end{cases}
,
\end{equation}
and $p_0^\nph = (p_0^n+p_0^{n+1})/2$.
\end{enumerate}
\end{enumerate}

%--------------------------------------------------------------------------
% STEP 9
%--------------------------------------------------------------------------
\item[Step 9] {\em React the full state through a second time interval of $\dt / 2.$}

Call {\bf React State}$[(\rho X_k)^{(2)},(\rho h)^{(2)}, p_0^{n+1}] \rightarrow [(\rho X_k)^{n+1}, (\rho h)^{n+1}, (\rho \omegadot_k)^{n+1}, (\rho \Hnuc)^{n+1}].$

%--------------------------------------------------------------------------
% STEP 10
%--------------------------------------------------------------------------
\item[Step 10] {\em Define the new time expansion, $S^{n+1}$.}

\begin{enumerate}
\renewcommand{\theenumi}{{\bf \Alph{enumi}}}
\item Define
\begin{equation}
  S^{n+1} =  \left(-\sigma  \sum_k  \xi_k \omegadot_k  + \sigma \Hnuc +
  \frac{1}{\rho p_\rho} \sum_k p_{X_k}  \omegadot_k\right)^{n+1}.
\end{equation}

\end{enumerate}


%--------------------------------------------------------------------------
% STEP 11
%--------------------------------------------------------------------------
\item[Step 11] {\em Update the velocity}.

First, we compute the time-centered edge velocities, $\Ub^{\nph,\pred}$
using a Godunov approach \citep{XRB_III}. Then, we define
\begin{equation}
\rho^\nph = \frac{\rho^n + \rho^{n+1}}{2}, \qquad \rho_0^\nph = \frac{\rho_0^n + \rho_0^{n+1}}{2}.
\end{equation}
We update the velocity field $\Ub^n$ to $\Ub^{n+1,\dagger}$ by discretizing
equation (\ref{eq:momentum}) as
\begin{equation}
\Ub^{n+1,\dagger}
= \Ub^n - \dt \left[\uadvtwo \cdot \nabla \Ub^{\nph,\pred} \right]
 - \dt \left[ \frac{1}{\rho^\nph} \nabla \pi^\nmh + \frac{\left(\rho^\nph-\rho_0^\nph\right)}{\rho^\nph} g^{\nph} \eb_r \right],
\end{equation}
Again, the $\dagger$ superscript refers
to the fact that the updated velocity does not satisfy the divergence
constraint.

Finally, we use an approximate projection to define $\Ub^{n+1}$
from $\Ub^{n+1,\dagger},$  such that $\Ub^{n+1}$ approximately
satisfies the divergence constraint,
\begin{equation}
\nabla \cdot \left(\beta_0^{n+1} \Ub^{n+1} \right) = \beta_0^{n+1} \left[ S^{n+1} - \frac{1}{\gammaonebar^{n+1}p_0^{n+1}}\left(\frac{\partial p_0}{\partial t}\right)^{\nph}\right].\label{eq:div3}
\end{equation}
As part of the projection we also define the new-time perturbational pressure, $\pi^\nph$.
This pressure $\pi$ is a nodal quantity, and thus this projection step is often referred to as a nodal projection.
This projection necessarily differs from the MAC projection used in
{\bf Step 3} and {\bf Step 7} because the velocities in those steps are defined
on edges and $\Ub^{n+1}$ is defined at cell centers, requiring different divergence
and gradient operators.

Refer to Appendix \ref{Sec:Projection} for more details about the projection steps.

\end{description}
This completes one step of the algorithm.

To initialize the simulation we use the same procedure described in Paper III.
At the beginning of each simulation, we define $(\Ub,\rho X_k,\rho h)$.
We set initial values for $\Ub, \rho X_k$, and $\rho h$ and perform a sequence of projections 
(to ensure the velocity field satisfies the divergence constraint) 
followed by a small number of steps of the temporal advancement scheme to iteratively 
find initial values for $\pi^{n-\myhalf}$ and $S^0$ and $S^1$ for use in the first time step.

Our approach to adaptive mesh refinement is algorithmically the same as the treatment described in Section 5 of Paper V; we refer the reader there for details.
We note that for spherical problems, AMR is only available for the case of a regularly-spaced base state.
