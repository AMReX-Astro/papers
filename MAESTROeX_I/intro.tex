\section{Introduction} \label{sec:intro}
Many astrophysical flows are highly subsonic.  In this regime,
sound waves carry sufficiently little energy that they do not
significantly affect the convective dynamics of the system.  In many
of these flows, modeling long-time convective dynamics are of
interest, and numerical approaches based on compressible hydrodynamics
are intractable, even on modern supercomputers.  One approach to this
problem is to use low Mach number models.  In a low Mach number
approach, sound waves are eliminated from the governing equations
while retaining compressibilitiy effects due to, e.g., nuclear energy
release, stratification, compositional changes, and thermal diffusion.  When the Mach
number (the ratio of the characteristic fluid velocity over the
characteristic sound speed; Ma $= U/c$) is small, the resulting system
can be numerically integrated with much larger time steps than a
compressible model.  Specifically, the time step increase is at least
a factor of $\sim 1/{\rm Ma}$ larger.  Each time step is more
computationally expensive due to the presence of additional linear
solves, but for many problems of interest the overall gains in
efficiency can easily be an order of magnitude or more.

Low Mach number models have been developed for a variety of contexts
including combustion \citep{day2000numerical}, terrestrial atmospheric
modeling \citep{durran:1989,oneill:2014,duarte2015low}, and elastic
solids \citep{abbate2017all}.  For astrophysical applications, a
number of approaches to modeling low Mach number flows have been
developed in recent years.  These include methods atmospheric model similar to
what is presented here~\cite{Lin:2006}.  There are also  semi-implicit all-Mach
number solvers, where the Euler equations are split into an acoustic
part and an advective part
\citep{Kwatra2009,Degond2009,Cordier2012,Haack2012,Happenhofer2013,Chalons2016,Padioleau2019}.
The fast acoustic waves are then solved using implicit time
integration, while the slow material waves are solved explicitly.
Another approach is to use preconditioned all-Mach number solvers
\citep{Miczek2014,Barsukow2016}, where the numerical flux is
multiplied by a preconditioning matrix.  This reduces the stiffness of
the system at low Mach numbers, while retaining the correct scaling
behavior. In the reduced speed of sound technique (RSST) and related
methods, the speed of sound is artificially reduced by including a
suitable scaling factor in the continuity equation, reducing the
restriction on the size of the timestep
\citep{Rempel2005,Hotta2012,Wang2015,Takeyama2017,Iijima2018}.
Finally, there are fully implicit time integration codes for the
compressible Euler equations
\citep{Viallet2011,kifonidis:2012,Viallet2015,Goffrey2016}.  The MUSIC code
uses fully implicit time integration for the compressible Euler
equations, which therefore allows for arbitrarily large timesteps.


Previously, we developed the low Mach number astrophysical solver, MAESTRO.
MAESTRO is a block-structured, Cartesian grid finite-volume, adaptive mesh refinement (AMR) code that has been successfully used for many years for a number of applications, detailed below.
Unlike several of the references above, MAESTRO is not an all-Mach solver, but is suitable for flows where the Mach number is small ($\sim 0.1$ or smaller).  
Furthermore, the low Mach number model in MAESTRO is specifically designed for, but not limited to, astrophysical settings with significant atmospheric stratification.
This includes full spherical stars, as well as planar simulations of dynamics within localized regions of a star.
MAESTRO uses an explicit Godunov approach for advection, a stiff ODE package (VODE) \citep{vode} for reactions, and multigrid-based linear solvers for the pressure-projection steps, and is thus time-step limited by an advective CFL constraint based on the fluid velocity, not the sound speed.
The time-advancement strategy uses Strang splitting to couple the thermodynamic variables, and a second-order projection method to integrate the velocity equations subject to a divergence constraint.
Central to the algorithm is a time-varying, one-dimensional stratified background (or base) state density and pressure held in hydrostatic equilibrium that couple to the full Cartesian grid solution.
The original MAESTRO code was developed in the pure-Fortran 90 FBoxLib software framework (available in the same github organization as AMReX).

The key numerical developments of the original MAESTRO algorithm are presented in a series of papers which we refer to as Papers I-V:
\begin{itemize}
\item In Paper I \citep{MAESTRO_I}, we derive the low Mach number equation set for stratified environments from the fully compressible equations.
\item In Paper II \citep{MAESTRO_II}, we incorporate the effects of atmospheric expansion through the use of a time-dependent background state.
\item In Paper III \citep{MAESTRO_III}, we incorporate reactions and the associated coupling to the hydrodynamics.
\item In Paper IV \citep{MAESTRO_IV}, we describe our treatment of spherical stars in a three-dimensional Cartesian geometry.
\item In Paper V \citep{MAESTRO_V}, we describe the use of block-structured adaptive mesh refinement to focus spatial resolution in regions of interest.
\end{itemize}

Since then, there have been many scientific investigations using MAESTRO, which have included additional algorithmic enhancements.  Topics include:
\begin{itemize}
\item The convective phase preceding Chandrasekhar mass models for type Ia supernovae \citep{MAESTRO_convection,MAESTRO_AMR,MAESTRO_CASTRO}.
\item Convection in massive stars \citep{Gilet:2013,gilkis:2016}.
\item Sub-Chandrasekhar white dwarfs \citep{subChandra_I,subChandra_II}.
%  In the latter paper we introduced an optional modification to the momentum equation of velocity constraint that conserves total energy, can give results closer to compressible codes of convective dynamics, including low density regions at the surface of a star.
\item Type I X-ray bursts \citep{XRB_I,XRB_II,XRB_III}.
%  In \cite{XRB_I} we discuss the modifications required for implicit thermal conduction, as well as an additional ``volume discrepancy'' modification to the velocity constraint to force the species and enthalpy to evolve in a manner consistent with the background pressure.
\end{itemize}

In this paper, we present new algorithmic methodology that improves upon Paper V in a number of ways.
First, the overall temporal algorithm has been greatly simplified without compromising accuracy.
The key design decisions were to eliminate the splitting of the velocity into average and perturbational components, 
and also to replace the hydrodynamic evolution of the base state with a predictor-corrector approach.
Not only does this greatly simplify the dynamics of the base
state, but this treatment is more amenable to higher-order multiphysics coupling strategies
based on method-of-lines integration.  
In particular, schemes based on spectral deferred corrections have been used to generate high-order temporal integrators for problems of reactive flow \citep{dutt2000spectral} and low Mach number combustion \citep{pazner2016high,nonaka2018conservative}.
Second, we explore the effects of alternative spatial mapping routines for coupling the base state and the Cartesian grid state in spherical problems.
Finally, the code has been completely ported to a new software framework suitable for modern manycore and GPU-based supercomputers.
We examine the performance of our new MAESTROeX implementation in the new C++/F90 AMReX public software library \citep{AMReX,AMReX_JOSS}.
MAESTROeX uses MPI+OpenMP parallelism and scales well to over 10,000 MPI processes, with each MPI process supporting tens of threads.
The resulting code is publicly available on GitHub ({\tt https://github.com/AMReX-Astro/MAESTROeX}),
uses the Starkiller-Astro microphysics libraries \citep{starkiller},
and the AMReX software library for block-structured adaptive mesh refinement ({\tt https://github.com/AMReX-Codes/amrex}).

The rest of this paper is organized as follows.
In Section \ref{sec:equations} we review our model for stratified low Mach number astrophysical flow.
In Section \ref{eq:algorithm} we present our numerical algorithm in detail, highlighting the new temporal integration scheme as well as spatial base state mapping options.
In Section \ref{sec:results} we validate our new approach and examine the performance of our algorithm on full spherical star problems used in previous scientific investigations.
We conclude in Section \ref{sec:conclusions}.
