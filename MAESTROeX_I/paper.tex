%% This manuscript uses the AASTeX v6.2 LaTeX 2e macros
%%
%% AASTeX is now based on Alexey Vikhlinin's emulateapj.cls
%% (Copyright 2000-2015).  See the classfile for details.

%% AASTeX requires revtex4-1.cls (http://publish.aps.org/revtex4/) and
%% other external packages (latexsym, graphicx, amssymb, longtable, and epsf).
%% All of these external packages should already be present in the modern TeX
%% distributions.  If not they can also be obtained at www.ctan.org.

%% The first piece of markup in an AASTeX v6.x document is the \documentclass
%% command. LaTeX will ignore any data that comes before this command. The
%% documentclass can take an optional argument to modify the output style.
%% The command below calls the preprint style  which will produce a tightly
%% typeset, one-column, single-spaced document.  It is the default and thus
%% does not need to be explicitly stated.
%%
%%
%% using aastex version 6.2
\documentclass{aastex62}

%% The default is a single spaced, 10 point font, single spaced article.
%% There are 5 other style options available via an optional argument. They
%% can be envoked like this:
%%
%% \documentclass[argument]{aastex62}
%%
%% where the layout options are:
%%
%%  twocolumn   : two text columns, 10 point font, single spaced article.
%%                This is the most compact and represent the final published
%%                derived PDF copy of the accepted manuscript from the publisher
%%  manuscript  : one text column, 12 point font, double spaced article.
%%  preprint    : one text column, 12 point font, single spaced article.
%%  preprint2   : two text columns, 12 point font, single spaced article.
%%  modern      : a stylish, single text column, 12 point font, article with
%% 		  wider left and right margins. This uses the Daniel
%% 		  Foreman-Mackey and David Hogg design.
%%  RNAAS       : Preferred style for Research Notes which are by design
%%                lacking an abstract and brief. DO NOT use \begin{abstract}
%%                and \end{abstract} with this style.
%%
%% Note that you can submit to the AAS Journals in any of these 6 styles.
%%
%% There are other optional arguments one can envoke to allow other stylistic
%% actions. The available options are:
%%
%%  astrosymb    : Loads Astrosymb font and define \astrocommands.
%%  tighten      : Makes baselineskip slightly smaller, only works with
%%                 the twocolumn substyle.
%%  times        : uses times font instead of the default
%%  linenumbers  : turn on lineno package.
%%  trackchanges : required to see the revision mark up and print its output
%%  longauthor   : Do not use the more compressed footnote style (default) for
%%                 the author/collaboration/affiliations. Instead print all
%%                 affiliation information after each name. Creates a much
%%                 long author list but may be desirable for short author papers
%%
%% these can be used in any combination, e.g.
%%
%% \documentclass[twocolumn,linenumbers,trackchanges]{aastex62}
%%
%% AASTeX v6.* now includes \hyperref support. While we have built in specific
%% defaults into the classfile you can manually override them with the
%% \hypersetup command. For example,
%%
%%\hypersetup{linkcolor=red,citecolor=green,filecolor=cyan,urlcolor=magenta}
%%
%% will change the color of the internal links to red, the links to the
%% bibliography to green, the file links to cyan, and the external links to
%% magenta. Additional information on \hyperref options can be found here:
%% https://www.tug.org/applications/hyperref/manual.html#x1-40003
%%
%% If you want to create your own macros, you can do so
%% using \newcommand. Your macros should appear before
%% the \begin{document} command.
%%
\newcommand{\vdag}{(v)^\dagger}
\newcommand\aastex{AAS\TeX}
\newcommand\latex{La\TeX}

% for non-stacked fractions
\newcommand{\sfrac}[2]{\mathchoice
  {\kern0em\raise.5ex\hbox{\the\scriptfont0 #1}\kern-.15em/
   \kern-.15em\lower.25ex\hbox{\the\scriptfont0 #2}}
  {\kern0em\raise.5ex\hbox{\the\scriptfont0 #1}\kern-.15em/
   \kern-.15em\lower.25ex\hbox{\the\scriptfont0 #2}}
  {\kern0em\raise.5ex\hbox{\the\scriptscriptfont0 #1}\kern-.2em/
   \kern-.15em\lower.25ex\hbox{\the\scriptscriptfont0 #2}}
  {#1\!/#2}}

\newcommand{\myhalf}{\sfrac{1}{2}}
\newcommand{\thalf}{\sfrac{3}{2}}

\newcommand{\eb}{{\bf{e}}}
\newcommand{\Ub}{{\bf{U}}}
\newcommand{\Ubt}{\widetilde{\Ub}}
\newcommand{\Vb}{{\bf{V}}}
\newcommand{\xb}{{\bf{x}}}

\newcommand{\dr}{\Delta r}
\newcommand{\dt}{\Delta t}

\newcommand{\etarho}{\eta_\rho}
\newcommand{\gammaonebar}{\overline{\Gamma}_1}
\newcommand{\Hnuc}{H_{\rm nuc}}
\newcommand{\omegadot}{\dot\omega}
\newcommand{\pred}{{\rm pred}}
\newcommand{\Sbar}{\overline{S}}

\newcommand{\inp}{\mathrm{in}}
\newcommand{\outp}{\mathrm{out}}
\newcommand{\nph}{{n+\myhalf}}
\newcommand{\nmh}{{n-\myhalf}}
\newcommand{\ow}{\overline{w_0}}
\newcommand{\dw}{\delta w_0}
\newcommand{\uadvone}{\Ub^{\mathrm{ADV},\star}}
\newcommand{\uadvonedag}{\Ub^{\mathrm{ADV},\dagger,\star}}
\newcommand{\uadvtwo}{\Ub^{\mathrm{ADV}}}
\newcommand{\uadvtwodag}{\Ub^{\mathrm{ADV},\dagger}}
\newcommand{\gcc}{\mathrm{g~cm^{-3} }}

% for the red MarginPars
\usepackage{color}
% make the MarginPars look pretty
\setlength{\marginparwidth}{0.5in}
\newcommand{\MarginPar}[1]{\marginpar{\vskip-\baselineskip\raggedright\tiny\sffamily
\hrule\smallskip{\color{red}#1}\par\smallskip\hrule}}

%% Reintroduced the \received and \accepted commands from AASTeX v5.2
\received{XXX X, XXXX}
\revised{XXX X, XXXX}
\accepted{XXX X, XXXX}
%% Command to document which AAS Journal the manuscript was submitted to.
%% Adds "Submitted to " the arguement.
\submitjournal{ApJ}

%% Mark up commands to limit the number of authors on the front page.
%% Note that in AASTeX v6.2 a \collaboration call (see below) counts as
%% an author in this case.
%
%\AuthorCollaborationLimit=3
%
%% Will only show Schwarz, Muench and "the AAS Journals Data Scientist
%% collaboration" on the front page of this example manuscript.
%%
%% Note that all of the author will be shown in the published article.
%% This feature is meant to be used prior to acceptance to make the
%% front end of a long author article more manageable. Please do not use
%% this functionality for manuscripts with less than 20 authors. Conversely,
%% please do use this when the number of authors exceeds 40.
%%
%% Use \allauthors at the manuscript end to show the full author list.
%% This command should only be used with \AuthorCollaborationLimit is used.

%% The following command can be used to set the latex table counters.  It
%% is needed in this document because it uses a mix of latex tabular and
%% AASTeX deluxetables.  In general it should not be needed.
%\setcounter{table}{1}

%%%%%%%%%%%%%%%%%%%%%%%%%%%%%%%%%%%%%%%%%%%%%%%%%%%%%%%%%%%%%%%%%%%%%%%%%%%%%%%%
%%
%% The following section outlines numerous optional output that
%% can be displayed in the front matter or as running meta-data.
%%
%% If you wish, you may supply running head information, although
%% this information may be modified by the editorial offices.
\shorttitle{MAESTROeX Low Mach Number Astrophysics}
\shortauthors{Fan et al.}
%%
%% You can add a light gray and diagonal water-mark to the first page
%% with this command:
% \watermark{text}
%% where "text", e.g. DRAFT, is the text to appear.  If the text is
%% long you can control the water-mark size with:
%  \setwatermarkfontsize{dimension}
%% where dimension is any recognized LaTeX dimension, e.g. pt, in, etc.
%%
%%%%%%%%%%%%%%%%%%%%%%%%%%%%%%%%%%%%%%%%%%%%%%%%%%%%%%%%%%%%%%%%%%%%%%%%%%%%%%%%

%% This is the end of the preamble.  Indicate the beginning of the
%% manuscript itself with \begin{document}.

\begin{document}

\title{MAESTROeX: A Massively Parallel Low Mach Number Astrophysical Solver}

%% LaTeX will automatically break titles if they run longer than
%% one line. However, you may use \\ to force a line break if
%% you desire. In v6.2 you can include a footnote in the title.

%% A significant change from earlier AASTEX versions is in the structure for
%% calling author and affilations. The change was necessary to implement
%% autoindexing of affilations which prior was a manual process that could
%% easily be tedious in large author manuscripts.
%%
%% The \author command is the same as before except it now takes an optional
%% arguement which is the 16 digit ORCID. The syntax is:
%% \author[xxxx-xxxx-xxxx-xxxx]{Author Name}
%%
%% This will hyperlink the author name to the author's ORCID page. Note that
%% during compilation, LaTeX will do some limited checking of the format of
%% the ID to make sure it is valid.
%%
%% Use \affiliation for affiliation information. The old \affil is now aliased
%% to \affiliation. AASTeX v6.2 will automatically index these in the header.
%% When a duplicate is found its index will be the same as its previous entry.
%%
%% Note that \altaffilmark and \altaffiltext have been removed and thus
%% can not be used to document secondary affiliations. If they are used latex
%% will issue a specific error message and quit. Please use multiple
%% \affiliation calls for to document more than one affiliation.
%%
%% The new \altaffiliation can be used to indicate some secondary information
%% such as fellowships. This command produces a non-numeric footnote that is
%% set away from the numeric \affiliation footnotes.  NOTE that if an
%% \altaffiliation command is used it must come BEFORE the \affiliation call,
%% right after the \author command, in order to place the footnotes in
%% the proper location.
%%
%% Use \email to set provide email addresses. Each \email will appear on its
%% own line so you can put multiple email address in one \email call. A new
%% \correspondingauthor command is available in V6.2 to identify the
%% corresponding author of the manuscript. It is the author's responsibility
%% to make sure this name is also in the author list.
%%
%% While authors can be grouped inside the same \author and \affiliation
%% commands it is better to have a single author for each. This allows for
%% one to exploit all the new benefits and should make book-keeping easier.
%%
%% If done correctly the peer review system will be able to
%% automatically put the author and affiliation information from the manuscript
%% and save the corresponding author the trouble of entering it by hand.

\correspondingauthor{Doreen Fan; DFan@lbl.gov}

\author[0000-0002-3246-4315]{Duoming Fan}
\affil{Lawrence Berkelely National Laboratory \\
Center for Computational Sciences and Engineering \\
One Cyclotron Road, MS 50A-3111 \\
Berkeley, CA 94720, USA}

\author[0000-0003-1791-0265]{Andrew Nonaka}
\affil{Lawrence Berkelely National Laboratory \\
Center for Computational Sciences and Engineering \\
One Cyclotron Road, MS 50A-3111 \\
Berkeley, CA 94720, USA}

\author[0000-0003-2103-312X]{Ann S. Almgren}
\affil{Lawrence Berkelely National Laboratory \\
Center for Computational Sciences and Engineering \\
One Cyclotron Road, MS 50A-3111 \\
Berkeley, CA 94720, USA}

\author[0000-0002-1530-781X]{Alice Harpole}
\affil{Stony Brook University \\
Department of Physics and Astronomy \\
Stony Brook, NY 11794-3800, USA}

\author[0000-0001-8401-030X]{Michael Zingale}
\affil{Stony Brook University \\
Department of Physics and Astronomy \\
Stony Brook, NY 11794-3800, USA}


%% Note that the \and command from previous versions of AASTeX is now
%% depreciated in this version as it is no longer necessary. AASTeX
%% automatically takes care of all commas and "and"s between authors names.

%% AASTeX 6.2 has the new \collaboration and \nocollaboration commands to
%% provide the collaboration status of a group of authors. These commands
%% can be used either before or after the list of corresponding authors. The
%% argument for \collaboration is the collaboration identifier. Authors are
%% encouraged to surround collaboration identifiers with ()s. The
%% \nocollaboration command takes no argument and exists to indicate that
%% the nearby authors are not part of surrounding collaborations.

%% Mark off the abstract in the ``abstract'' environment.
\begin{abstract}
We present MAESTROeX, a massively parallel solver for low Mach number astrophysical flows.
Our model equations allow for long-time integration for highly subsonic flows compared to compressible approaches.
The code leverages the new AMReX framework for block-structured adaptive mesh refinement calculations, and also provides additional algorithmic features from the original MAESTRO code.
We present a new temporal integration option that is much simpler than our previous approach while retaining the same order order of accuracy.
We also provide a more accurate spatial discretization for mapping the base state onto the full state Cartesian grid.
Using our previous studies on the convective phase of single-degenerate progenitor models of Type Ia supernovae, we characterize the performance of the code and validate the new algorithmic features.
\end{abstract}

%% Keywords should appear after the \end{abstract} command.
%% See the online documentation for the full list of available subject
%% keywords and the rules for their use.
\keywords{convection, hydrodynamics, methods: numerical, nuclear reactions, nucleosynthesis, abundances, supernovae: general}

%% From the front matter, we move on to the body of the paper.
%% Sections are demarcated by \section and \subsection, respectively.
%% Observe the use of the LaTeX \label
%% command after the \subsection to give a symbolic KEY to the
%% subsection for cross-referencing in a \ref command.
%% You can use LaTeX's \ref and \label commands to keep track of
%% cross-references to sections, equations, tables, and figures.
%% That way, if you change the order of any elements, LaTeX will
%% automatically renumber them.
%%
%% We recommend that authors also use the natbib \citep
%% and \citet commands to identify citations.  The citations are
%% tied to the reference list via symbolic KEYs. The KEY corresponds
%% to the KEY in the \bibitem in the reference list below.

\section{Introduction} \label{sec:intro}

TODO list:
\begin{itemize}
\item Weak scaling tests of original algorithm, comparing MAESTRO to MAESTROeX.  Use cori knl, 4 MPI per node, 16 threads per MPI
\item Run at effective $512^3$ resolution (maybe to ignition?... need to see how max T trends go) (1) classic MAESTRO, (2) MAESTROeX with original algorithm, (3) MAESTROeX with new temporal integrator, (4) MAESTROeX with new temporal integrator and irregular dr, (5) MAESTROeX with original algorithm and 3-levels of AMR, (6) MAESTROeX with new temporal integrator and 3 levels of AMR.
\item (perhaps) a demonstration of what goes wrong when you don't split the velocity dynamics in the projection
\end{itemize}

Many astrophysical flows are highly subsonic; sound waves carry sufficiently little energy that they do not significantly affect the convective dynamics of the system.
In many of these flows, modeling long-time convective dynamics are of interest, and numerical approaches based on compressible hydrodynamics are intractable, even on modern supercomputers.
One approach to this problem is to use low Mach number models.
In a low Mach number approach, sound waves are eliminated from the governing equations while retaining compressibilitiy effects due to, e.g., nuclear energy release, compositional changes, and thermal diffusion.
The resulting model can be numerically integrated with much larger time steps than a compressible model.

Previously, we have developed the low Mach number astrophysical solver, MAESTRO.
Low Mach number models have been developed for a variety of contexts including combustion \citep{day2000numerical}, atmospherics \citep{duarte2015low}, and elastic solids \citep{abbate2017all}.
The low Mach number model in MAESTRO is unique in that it is specifically designed for astrophysical settings with significant atmospheric stratification.
Central to the algorithm is a stratified background (or base) state density and pressure held in hydrostatic equilibrium, and also vary as a function of altitude and time.
MAESTRO is suitable for for spherical stars, as well as planar simulations of local dynamics.

The key numerical developments of the original MAESTRO algorithm are presented in a series of papers which we refer to as Papers I-V:
\begin{itemize}
\item In Paper I \citep{MAESTRO_I}, we derive the low Mach number equation set from the fully compressible equations.
\item In Paper II \citep{MAESTRO_II}, we incorporate the effects atmospheric expansion through the use of a time-dependent background state.
\item In Paper III \citep{MAESTRO_III}, we incorporate reactions and the associated coupling to the hydrodynamics.
\item In Paper IV \citep{MAESTRO_IV}, we describe our treatment of spherical stars in a three-dimensional Cartesian geometry.
\item In Paper V \citep{MAESTRO_V}, we describe the use of block-structured adaptive mesh refinement to focus spatial resolution in regions of interest.
\end{itemize} 

Since then, there have been many scientific investigations using MAESTRO, which include additional algorithmic enhancements.  Topics include:
\begin{itemize}
\item The convective phase preceding Chandrasekhar mass models for type Ia supernovae \citep{MAESTRO_convection,MAESTRO_AMR,MAESTRO_CASTRO}.
\item Convection in massive stars \citep{Gilet:2013}.
\item Sub-Chandrasekhar white dwarfs \citep{subChandra_I,subChandra_II}.  In the latter paper we introduced an optional modification to the momentum equation of velocity constraint that conserves total energy, can give results closer to compressible codes of convective dynamics, including low density regions at the surface of a star.
\item Type I X-ray bursts \citep{XRB_I,XRB_II,XRB_III}.  In \cite{XRB_I} we discuss the modifications required for implicit thermal conduction, as well as an additional ``volume discrepancy'' modification to the velocity constraint to force the species and enthalpy to evolve in a manner consistent with the background pressure.
\end{itemize}

In this paper, we characterize the performance of the MAESTROeX algorithm compared to the previous implementation.
We also develop simpler, alternative temporal integrators, with a longer term goal of developing high-order coupling schemes.
We also present a new treatment of the base state mapping to Cartesian grids for spherical problsm.

\section{Governing Equations}
The derivation of the low Mach number hydrodynamics equations is given in Papers I-III.
We take the standard equations of reacting, compressible flow, and recast the equation
of state as a divergence constraint on the velocity field.  This is done by taking
the Langrangian derivative of the equation of state for pressure as a function of the
density, composition, and enthalpy, enforcing the constraint that the pressure is
a prescribed function of altitutde and time based on the hydrostatic equilibrium condition,
and substituting the equations for species and enthalpy in the resulting Lagrangian derivative.
The resulting equations are
\begin{eqnarray}
\frac{\partial\Ub}{\partial t} &=& -\Ub\cdot\nabla\Ub  - \frac{1}{\rho}\nabla\pi - \frac{\rho-\rho_0}{\rho} g\eb_r,\label{eq:momentum}\\
\frac{\partial(\rho X_k)}{\partial t} &=& -\nabla\cdot(\rho X_k\Ub) + \rho\omegadot_k,\label{eq:species}\\
\frac{\partial(\rho h)}{\partial t} &=& -\nabla\cdot(\rho h\Ub) + \frac{Dp_0}{Dt} + \rho\Hnuc,\label{eq:enthalpy}
\end{eqnarray}
\begin{equation}
\nabla\cdot(\beta_0\Ub) = \beta_0\left(S - \frac{1}{\gammaonebar p_0}\frac{\partial p_0}{\partial t}\right).\label{eq:U divergence}
\end{equation}
Here $\rho$, $\Ub$, and $h$ are the mass density,
velocity and specific enthalpy, respectively, and
$X_k$ are the mass fractions of species $k$ with associated
production rate $\omegadot_k$.  The species are constrained
such that $\sum_k X_k = 1$ giving $\rho = \sum_k (\rho X_k)$ and
\begin{equation}
\frac{\partial\rho}{\partial t} = -\nabla\cdot(\rho\Ub).
\end{equation}
Here $\Hnuc$ is the nuclear energy generation rate per unit mass.
The pressure is decomposed into a hydrostatic base state
 pressure, $p_0 = p_0(r,t)$, and a dynamic pressure, $\pi = \pi(\xb,t)$, such that 
$|\pi|/p_0 = \mathcal{O}(M^2)$ (we use $\xb$ to represent the Cartesian coordinate 
directions of the full state and $r$ to represent the radial coordinate direction for 
the base state).  We also define a base state density, $\rho_0 = \rho_0(r,t)$, 
which is in hydrostatic equilibrium with $p_0$, i.e., 
$\nabla p_0 = -\rho_0 g\eb_r$, where $g=g(r,t)$ is
the magnitude of the gravitational acceleration and $\eb_r$ is the unit vector in the
outward radial direction. 

Mathematically, equations (\ref{eq:momentum})-(\ref{eq:enthalpy}) must still be closed by the equation of state which we
express as a divergence constraint on the velocity field (\ref{eq:U divergence}),
\MarginPar{This is different from the Jacobs 2016 paper where we switched to alt\_energy\_fix.}
where $\beta_0$ is a density-like variable that carries background stratification, defined as
\begin{equation}
\beta_0(r,t) = \rho_0(0,t)\exp\left(\int_0^r\frac{1}{\gammaonebar p_0}\frac{\partial p_0}{\partial r'}dr'\right),
\end{equation}
where $\gammaonebar$ is the lateral average of $\Gamma_1 = d(\log p)/d(\log\rho)$ at constant entropy.
The expansion term, $S$, incorporates local compressibility effects due to heat release from reactions, compositional changes, and external sources,
\begin{equation}
S = -\sigma\sum_k\xi_k\omegadot_k + \frac{1}{\rho p_\rho}\sum_k p_{X_k}\omegadot_k + \sigma\Hnuc,\label{eq:S}
\end{equation}
where $p_{X_k} \equiv \left. \partial p / \partial X_k
\right|_{\rho,T,X_{j,j\ne k}}$, $\xi_k \equiv \left. \partial h /
\partial X_k \right |_{p,T,X_{j,j\ne k}},
p_\rho \equiv \left.
\partial p/\partial \rho \right |_{T, X_k}$, and $\sigma \equiv
p_T/(\rho c_p p_\rho)$, with $p_T \equiv \left. \partial p / \partial
T \right|_{\rho, X_k}$ and $c_p \equiv \left.  \partial h / \partial T
\right|_{p,X_k}$ is the specific heat at constant pressure.

Previously we adopted an approach where we split the velocity into a base state component, $w_0(r,t)$, 
and a local velocity $\Ubt(\xb,t)$, so that
\begin{equation}
\Ub = \Ubt(\xb,t) + w_0(r,t)\eb_r,
\end{equation}
We used $w_0$ to provide an estimate of the base state density evolution over a time step.
In our new simplified approach, we instead use a predictor-corrector approach for evolving the base state.

\section{Using Exact Spherical Base States on Irregular Grid} \label{sec:exactbasestate}

\subsection{Potential Problems}
\begin{enumerate}
\item How do we define \verb|r_edge_loc| for irregularly-spaced base states whose cell-centered values are defined at $r_m=\Delta x\sqrt{\frac{3}{4} + 2m}$?
\item We would stil need to approximate the values at edges when advecting the base states (i.e. $\rho_0^{\nph,\pred}$). This may affect the nice properties we would get with exact cell-centered base state values.
\item Excessive memory storage may be required (\verb|nr_irreg >> nr_fine|).
\end{enumerate}

\subsection{A Possible Solution}
We could choose to no longer advect the base states for both predictor and corrector steps. This would solve the problems with computing state values at cell edges. Instead, we propose to only update the base states $\rho_0$ and $p_0$ by computing the average of the full state values after advecting the full states. This would result in the following changes in the algorithm from Paper V:
%
\begin{enumerate}
\item Steps 4 and 8 will be greatly simplified, leading to a more efficient algorithm.
\item \sout{There is no longer a need for $w_0$, so we can use the full velocity $\Ub$ instead of $\Ubt$. As a consequence, $\eta$ does not need to be computed. This has the added advantage of saving some memory storage by removing many of the terms needed when we decided to split terms into its base state and perturbation state in our previous implementation.} We split the velocity for the projection to handle the cutoff.
\item By placing the base states exactly at cell-centers, we eliminate interpolation errors and simplify the existing code.
\end{enumerate}
%

\subsection{Main Algorithm Description}\label{Sec:Main Algorithm Description}
Our new temporal integration scheme is much simpler than the scheme from Paper V.
The primary reason is that we no longer split the velocity field into a perturbational and base state components.  
Previously, we used the base state velocity to predict the expansion of the base state.  
In this new scheme we use a simpler, predictor-corrector approach to the base state density and pressure that no longer requires the complex algorithm from Paper V, yet still retains the same overall second-order accuracy.

We now describe the new temporal integration scheme, which has been greatly simplified
from Paper V without formal loss of order of accuracy.
Note that this temporal integration scheme is valid for the original base state mapping (with constant base state grid spacing), or the new irregularly spaced base state mapping.

At the beginning of each time step we have the cell-centered state,
$(\Ub,\rho X_k,\rho h,\rho_0,p_0)^n$, and nodal state, $\pi^{n-\myhalf}$.
At any time, the associated density, composition, and enthalpy can be trivially computed using, e.g.,
\begin{equation}
\rho^n = \sum_k(\rho X_k)^n, \quad
X_k^n = (\rho X_k)^n / \rho^n, \quad
h^n = (\rho h)^n / \rho^n.
\end{equation}
Temperature is computed using the equation of state, e.g.,
\footnote{As described in Paper V, for planar problems we compute temperature using $h$ instead of $p_0$, since we have successfully developed planar volume discrepancy schemes to effectively couple the enthalpy to the rest of the solution.  We are still exploring this option for spherical stars.}
\begin{equation}
T = T(\rho,p_0,X_k),
\end{equation}
and ($\gammaonebar,\beta_0)$ are computed from $(\rho,p_0,X_k)$ (see Appendix A of Paper I and Appendix C of Paper III for details).

The overall flow of the algorithm is a second-order Strang splitting approach for the coupling of the reactions and advection of the thermodynamic variables.  
We use a predictor-corrector approach within the Strang splitting scheme to achieve second-order accuracy in time.
We integrate the velocity using a standard second-order projection method to enforce the divergence constraint.
To summarize:
\begin{itemize}
\item In {\bf Step 1} we react the thermodynamic variables over $\Delta t/2$.
\item In {\bf Steps 2-4} we advect the thermodynamic variables over $\Delta t$.  Specifically, we compute an estimate for the expansion term, $S$, project the velocities so they satisfy the divergence constraint, and then advect the thermodynamic variables.
\item In {\bf Step 5} we react the thermodynamic variables over $\Delta t/2$.\footnote{Note that we could skip to the velocity advance in {\bf Steps 10-11}, however the overall scheme would be only first-order in time, so {\bf Steps 6-9} can be thought of as a trapezoidal corrector step.}
\item In {\bf Steps 6-8} we redo the advection in {\bf Steps 2-4} but are able to use the trapezoidal rule to time-center certain quantities such as $S$, $\rho_0$, etc.
\item In {\bf Step 9} we redo the reactions from {\bf Step 5} using the improved results for the corrector advection step.
\item In {\bf Steps 10-11} we update the velocity.
\end{itemize}

Each time step satisfies the standard advective CFL condition,
\begin{equation}
\Delta t = \sigma^{\rm CFL} \min_i(\Delta x / U_i),
\end{equation}
where for our simulations we typically use $\sigma^{\rm CFL}\sim 0.5$.
There are additional constraints on the time involving the buoyancy force (commonly used when the velocity is approximately zero at the start of some simulations) and the divergence constraint (to prevent too much mass evacuation from a cell in a time step); see Section 3.4 in Paper III for details.

At the beginning of each simulation, we define $(\Ub,\rho X_k,\rho h)$...

\MarginPar{mention cutoff densities}

\MarginPar{AJN: eliminated references to volume discrepancy for now; we'd only use it for planar at the moment, don't want to open up that can of worms here}

\begin{description}

%--------------------------------------------------------------------------
% STEP 1
%--------------------------------------------------------------------------
\item[Step 1] {\em React the full state through the first time interval of $\dt / 2.$}

Call {\bf React State}$[(\rho X_k)^n, (\rho h)^n, p_0^n] \rightarrow [(\rho X_k)^{(1)}, (\rho h)^{(1)}, (\rho \omegadot_k)^{(1)}, (\rho \Hnuc)^{(1)}]$.
\MarginPar{AJN: I think we can simplify the description here... this is somewhat confusing now.}

%--------------------------------------------------------------------------
% STEP 2
%--------------------------------------------------------------------------

\item[Step 2] {\em Compute the provisional time-centered expansion,
    $S^{\nph,\star}$.}

We compute an estimate for the time-centered expansion term in the velocity
divergence constraint.  Following \citet{Bell:2004}, we extrapolate
to the half-time using $S$ at the previous and current
time steps,
\begin{equation}
S^{\nph,\star} = S^n + \frac{\dt^n}{2} \left(\frac{S^n - S^{n-1}}{\dt^{n-1}}\right).
\end{equation}
Note that in the first time step we average $S^0$ and $S^1$ from the
initialization step.

%--------------------------------------------------------------------------
% STEP 3
%--------------------------------------------------------------------------
\item[Step 3] {\em Construct the provisional time-centered advective velocity on
edges, $\uadvone$.}

The construction of face-centered time-centered states used to discretize the
advection terms for velocity, species, and enthalpy, are performed using
a standard multidimensional corner transport upwind approach
\citep{colella1990multidimensional,saltzman1994unsplit} with piecewise-parabolic
one-dimensional tracing \citep{colella1984piecewise}.  The full details of this
Godunov advection approach for all steps in this algorithm are described 
in Appendix A of \cite{XRB_III}.

In this step, using equation (\ref{eq:momentum}), 
we compute time-centered edge velocities, $\uadvonedag$, using
$\Ub^n$.  The $\dagger$ superscript refers to the
fact that the predicted velocity field does not satisfy the divergence
constraint.  We then construct $\uadvone$ from $\uadvonedag$
using a MAC projection.
We note that $\uadvone$ satisfies the divergence constraint
\begin{equation}
\nabla \cdot \left(\beta_0^n \uadvone\right) = \beta_0^n \left(S^{\nph,\star} - \frac{1}{\gammaonebar^np_0^n}\frac{\partial p_0}{\partial t} \right).
\end{equation}
In practice we solve for.

%--------------------------------------------------------------------------
% STEP 4
%--------------------------------------------------------------------------
\item[Step 4] {\em Advect the full state through a time interval of $\dt.$}

\begin{enumerate}
\renewcommand{\theenumi}{{\bf \Alph{enumi}}}

\item Update $(\rho X_k)$ using a discretized version of
%
\begin{equation}
\frac{\partial(\rho X_k)}{\partial t} = -\nabla\cdot(\rho X_k\Ub) + \rho\omegadot_k,
\end{equation}
%
omitting the reaction terms, which were already
accounted for in {\bf React State}.  The update consists of two steps:

\begin{enumerate}
\renewcommand{\labelenumii}{{\bf \roman{enumii}}.}

\item Compute the time-centered species edge states, $(\rho X_k)^{\nph,\pred}$,
  for the conservative update of $(\rho X_k)^{(1)}$ using a Godunov approach \citep{XRB_III}.
\MarginPar{We may not need $\rho_0$ if we predict $\rho X$ or $\rho$ and $X$.  AJN: If we end up keeping prediction of $\rho'$ and $X$, and $(\rho h)'$ we need to summarize this, particularly since $\rho_0$ and $(\rho h)_0$ treatment differs from previous papers.}

\item Evolve $(\rho X_k)^{(1)} \rightarrow (\rho X_k)^{(2),\star}$ using
\begin{equation}
(\rho X_k)^{(2),\star} = (\rho X_k)^{(1)}
  - \dt \left\{ \nabla \cdot \left[ \uadvone (\rho X_k)^{\nph,\pred} \right] \right\},
\end{equation}

\end{enumerate}

\item Update $\rho_0$ by setting $\rho_0^{n+1,\star} =$ {\bf Avg}$(\rho^{(2),\star})$.

\item Update $p_0$ by calling
{\bf Enforce HSE}$[p_0^n,\rho_0^{n+1,\star}] \rightarrow [p_0^{n+1,\star}]$.

\item Update the enthalpy using a discretized version of equation
%
\begin{equation}
\frac{\partial(\rho h)}{\partial t} = -\nabla\cdot(\rho h\Ub) + \frac{Dp_0}{Dt} + \rho\Hnuc,
\end{equation}
%
again omitting the reaction and heating terms
since we already accounted for
them in {\bf React State}.  This equation takes the form:
\begin{equation}
\frac{\partial (\rho h)}{\partial t}  = - \nabla \cdot (\Ub \rho h) + \frac{\partial p_0}{\partial t} + (\Ub \cdot \eb_r) \frac{\partial p_0}{\partial r}.
\end{equation}
For spherical geometry, we solve the
analytically equivalent form,
\begin{equation}
\frac{\partial (\rho h)}{\partial t}  = - \nabla \cdot (\Ub \rho h) + \frac{\partial p_0}{\partial t} + \nabla \cdot (\Ub p_0) - p_0 \nabla \cdot \Ub.
\end{equation}
The update consists of two steps:

\begin{enumerate}
\renewcommand{\labelenumii}{{\bf \roman{enumii}}.}

\item Compute the time-centered enthalpy edge state, $(\rho h)^{\nph,\pred},$
  for the conservative update of $(\rho h)^{(1)}$
  using using a Godunov approach \citep{XRB_III}.

\item Evolve $(\rho h)^{(1)} \rightarrow (\rho h)^{(2),\star}$.
\begin{description}
\item For planar geometry,
\begin{eqnarray}
(\rho h)^{(2),\star}
&=& (\rho h)^{(1)} \nonumber \\
&&- \dt^n \left\{ \nabla \cdot \left[ \uadvone (\rho h)^{\nph,\pred} \right] \right\} \nonumber \\
&& + \dt^n \left(\frac{p_0^n - p_0^{n-1}}{\dt^{n-1}}\right) \nonumber \\
&& + \dt^n \left(\uadvone \cdot \eb_r\right) \left(\frac{\partial p_0}{\partial r} \right)^{n}
\end{eqnarray}

\item For spherical geometry,
\begin{eqnarray}
(\rho h)^{(2),\star}
&=& (\rho h)^{(1)} \nonumber \\
&&- \dt^n \left\{ \nabla \cdot \left[ \uadvone (\rho h)^{\nph,\pred} \right] \right\} \nonumber \\
&& + \dt^n \left(\frac{p_0^n - p_0^{n-1}}{\dt^{n-1}}\right) \nonumber \\
&& + \dt^n \left \{ \nabla \cdot \left (\uadvone p_0^{n} \right ) - p_0^{n} \nabla \cdot \uadvone \right \}
\end{eqnarray}
\end{description}

\end{enumerate}

Then, for each Cartesian cell where $\rho^{(2),\star} < \rho_\mathrm{cutoff}$,
we recompute enthalpy using
\begin{equation}
(\rho h)^{(2),\star} = \rho^{(2),\star}h\left(\rho^{(2),\star},p_0^{n+1,\star},X_k^{(2),\star}\right).
\end{equation}

\end{enumerate}

%--------------------------------------------------------------------------
% STEP 5
%--------------------------------------------------------------------------
\item[Step 5] {\em React the full state through a second time interval of $\dt / 2.$}

Call {\bf React State}$[ (\rho X_k)^{(2),\star}, (\rho h)^{(2),\star}, p_0^{n+1,\star}] 
\rightarrow 
[ (\rho X_k)^{n+1,\star}, (\rho h)^{n+1,\star}, (\rho \omegadot_k)^{n+1,\star}, (\rho \Hnuc)^{n+1,\star} ].$

%--------------------------------------------------------------------------
% STEP 6
%--------------------------------------------------------------------------
\item[Step 6] {\em Compute the time-centered expansion, $S^{\nph,\star}$.}

First, compute $S^{n+1,\star}$ with
\begin{equation}
S^{n+1,\star} =  \left(-\sigma  \sum_k  \xi_k  \omegadot_k  + \frac{1}{\rho p_\rho} \sum_k p_{X_k}  {\omegadot}_k + \sigma \Hnuc\right)^{n+1,\star}.
\end{equation}
  Then, define
\begin{equation}
 S^{\nph} = \frac{S^n + S^{n+1,\star}}{2},
\end{equation}

%--------------------------------------------------------------------------
% STEP 7
%--------------------------------------------------------------------------
\item[Step 7] {\em Construct the time-centered advective velocity on edges, $\uadvtwo$.}

The procedure to construct $\uadvtwodag$ is identical to the Godunov procedure
for computing $\uadvonedag$ in {\bf Step 3}, but uses
the updated value $S^{\nph}$ rather than $S^{\nph,\star}$.
We note that $\uadvtwo$ satisfies the divergence constraint
\begin{equation}
\nabla \cdot \left(\beta_0^{\nph} \uadvtwo\right) =
\beta_0^{\nph} \left(S^{\nph} - \frac{1}{\gammaonebar^{n+\myhalf}p_0^{n+\myhalf}}\frac{\partial p_0}{\partial t}\right),
\end{equation}
where
\begin{equation}
\beta_0^{\nph} = \frac{ \beta_0^n +  \beta_0^{n+1,\star} }{2},
\qquad
\gammaonebar^{\nph} = \frac{ \gammaonebar^n +  \gammaonebar^{n+1,\star} }{2}.
\qquad
p_0^{\nph} = \frac{ p_0^n + p_0^{n+1,\star} }{2},
\end{equation}

%--------------------------------------------------------------------------
% STEP 8
%--------------------------------------------------------------------------
\item[Step 8] {\em Advect the base state and full state through a time interval of $\dt.$}

\begin{enumerate}
\renewcommand{\theenumi}{{\bf \Alph{enumi}}}

\item Update $(\rho X_k)$.  This step is identical to {\bf Step 4A} except we use
  the updated values $\uadvtwo$ and $\rho_0^{n+1,\star}$ rather than
  $\uadvone$ and $\rho_0^{n+1,\star}$.  In particular:

\begin{enumerate}
\renewcommand{\labelenumii}{{\bf \roman{enumii}}.}

\item Compute the time-centered species edge states, $(\rho X_k)^{\nph,\pred}$,
  for the conservative update of $(\rho X_k)^{(1)}$ usingusing a Godunov approach \citep{XRB_III}.

\item Evolve $(\rho X_k)^{(1)} \rightarrow (\rho X_k)^{(2)}$ using
\begin{equation}
(\rho X_k)^{(2)} = (\rho X_k)^{(1)}
- \dt \left\{ \nabla \cdot \left[\uadvtwo (\rho X_k)^{\nph,\pred} \right] \right\},
\end{equation}

\end{enumerate}

\item Update $\rho_0$ by setting $\rho_0^{n+1} =$ {\bf Avg}$(\rho^{(2)})$.

\item Update $p_0$ by calling
{\bf Enforce HSE}$[p_0^n,\rho_0^{n+1}] \rightarrow [p_0^{n+1}]$.

\item Update the enthalpy.  This step is identical to {\bf Step 4D} except we use
  the updated values $\uadvtwo, \rho_0^{n+1}, (\rho h)_0^{n+1}$, and $p_0^{n+\myhalf}$
  rather than
  $\uadvone, \rho_0^{n+1,\star}, (\rho h)_0^{n+1,\star}$, and $p_0^n$.
  In particular:
\MarginPar{Again, we may not need $\rho_0$ and $(\rho h)_0$.}

\begin{enumerate}
\renewcommand{\labelenumii}{{\bf \roman{enumii}}.}

\item Compute the time-centered enthalpy edge state, $(\rho h)^{\nph,\pred},$
  for the conservative update of $(\rho h)^{(1)}$ using a Godunov approach \citep{XRB_III}..

\item Evolve $(\rho h)^{(1)} \rightarrow (\rho h)^{(2)}$.
\begin{description}
\item For planar geometry,
\begin{eqnarray}
(\rho h)^{(2)}
&=& (\rho h)^{(1)} - \dt \left\{ \nabla \cdot \left[ \uadvtwo (\rho h)^{\nph,\pred} \right] \right\} \nonumber \\
&& + \dt \left( \frac{p_0^{n+1} - p_0^n}{\dt} \right)\nonumber \\
&& + \dt \left(\uadvtwo \cdot \eb_r\right) \left(\frac{\partial p_0}{\partial r} \right)^\nph,
\end{eqnarray}

\item For spherical geometry,
\begin{eqnarray}
(\rho h)^{(2)}
&=& (\rho h)^{(1)} - \dt \left\{ \nabla \cdot \left[ \uadvtwo (\rho h)^{\nph,\pred} \right] \right\} \nonumber \\
&& + \dt \left( \frac{p_0^{n+1} - p_0^n}{\dt} \right)\nonumber \\
&& + \dt \left[ \nabla \cdot \left (\uadvtwo p_0^{\nph} \right ) - p_0^{\nph} \nabla \cdot \uadvtwo \right],
\end{eqnarray}
\end{description}
where $p_0^\nph$ is defined as $p_0^\nph = (p_0^n+p_0^{n+1})/2$.

\end{enumerate}

Then, for each Cartesian cell where $\rho^{(2)} < \rho_\mathrm{cutoff}$, we recompute enthalpy using
\begin{equation}
(\rho h)^{(2)} = \rho^{(2)}h\left(\rho^{(2)},p_0^{n+1},X_k^{(2)}\right).
\end{equation}

\end{enumerate}

%--------------------------------------------------------------------------
% STEP 9
%--------------------------------------------------------------------------
\item[Step 9] {\em React the full state through a second time interval of $\dt / 2.$}

Call {\bf React State}$[(\rho X_k)^{(2)},(\rho h)^{(2)}, p_0^{n+1}] \rightarrow [(\rho X_k)^{n+1}, (\rho h)^{n+1}, (\rho \omegadot_k)^{n+1}, (\rho \Hnuc)^{n+1}].$

%--------------------------------------------------------------------------
% STEP 10
%--------------------------------------------------------------------------
\item[Step 10] {\em Define the new time expansion, $S^{n+1}$.}

\begin{enumerate}
\renewcommand{\theenumi}{{\bf \Alph{enumi}}}
\item Define
\begin{equation}
  S^{n+1} =  \left(-\sigma  \sum_k  \xi_k \omegadot_k  + \sigma \Hnuc +
  \frac{1}{\rho p_\rho} \sum_k p_{X_k}  \omegadot_k\right)^{n+1}.
\end{equation}

\end{enumerate}


%--------------------------------------------------------------------------
% STEP 11
%--------------------------------------------------------------------------
\item[Step 11] {\em Update the velocity}.

First, we compute the time-centered edge velocities, $\Ub^{\nph,\pred}$
using a Godunov approach \citep{XRB_III}. Then, we define
\begin{equation}
\rho^\nph = \frac{\rho^n + \rho^{n+1}}{2}, \qquad \rho_0^\nph = \frac{\rho_0^n + \rho_0^{n+1}}{2}.
\end{equation}
We update the velocity field $\Ub^n$ to $\Ub^{n+1,\dagger}$ by discretizing
equation (\ref{eq:momentum}) as
\begin{equation}
\Ub^{n+1,\dagger}
= \Ub^n - \dt \left[\uadvtwo \cdot \nabla \Ub^{\nph,\pred} \right]
 - \dt \left[ \frac{1}{\rho^\nph} \nabla \pi^\nmh + \frac{\left(\rho^\nph-\rho_0^\nph\right)}{\rho^\nph} g^{\nph} \eb_r \right],
\end{equation}
Again, the $\dagger$ superscript refers
to the fact that the updated velocity does not satisfy the divergence
constraint.

Finally, we use an approximate nodal projection to define $\Ub^{n+1}$
from $\Ub^{n+1,\dagger},$  such that $\Ub^{n+1}$ approximately
satisfies
\begin{equation}
\nabla \cdot \left(\beta_0^{n+1} \Ub^{n+1} \right) = \beta_0^{n+1} \left( S^{n+1} - \frac{1}{\gammaonebar^{n+1}p_0^{n+1}}\frac{\partial p_0}{\partial t} \right).
\end{equation}
As part of the projection we also define the new-time perturbational pressure,
$\pi^\nph.$  This projection necessarily differs from the MAC projection used in
{\bf Step 3} and {\bf Step 7} because the velocities in those steps are defined
on edges and $\Ub^{n+1}$ is defined at cell centers, requiring different divergence
and gradient operators.  Details of the approximate projection are given in Paper III.

\end{description}

This completes one step of the algorithm.

To initialize the simulation



%--------------------------------------------------------------------------
% STEP 0
%--------------------------------------------------------------------------
{\em Initialization}
\sout{
The initialization step only occurs at the beginning of the simulation.
The initialization step computes initial values for 
$\pi^{-\myhalf}, \eta_{\rho}^{-\myhalf}, S^0$, and $S^1$ for use in time stepping routine.  
We pass through {\bf Step 1} -- {\bf Step 11}, but only keep these values.
The initial values for $\Ub^0, \rho^0, (\rho h)^0, X_k^0, T^0,
\rho_0^0, p_0^0$, and $\gammaonebar^0$ are specified from the problem-dependent
initial conditions.  The initial time step, $\dt^0$, is computed as in
Paper III.
}

\section{Test Problems}

\subsection{Convergence Test}
3D reacting bubble rise; to demonstrate 2nd-order convergence of the new temporal algorithm.

\subsection{Scaling}
Scaling with OpenMP

\subsection{Performance Compared to Original MAESTRO}
Compare to original MAESTRO

\subsection{White Dwarf Convection}
white dwarf convection runs with 3 algorithms (original, new temporal, new temporal + irregular base state)

\section{Conclusions and Future Work}

science: rotation, solar physics, MHD

algorithm: SDC \cite{dutt2000spectral}, high-order, multi-implicit \cite{bourlioux2003high}.
Consider SDC ideas used for terrestrial combustion \cite{pazner2016high,nonaka2018conservative}

implementation: GPUs.  Forward cite next CASTRO paper with GPU strategy.

\appendix
\section{Projection Details}


After {\bf Step 8D}, we define a radial cell-centered $\etarho^{\nph}$.

\begin{description}
\item For planar geometry, $\etarho = \overline{\rho'(\Ub\cdot\eb_r)}$,
\begin{equation}
 \etarho^{\nph} =  {\rm {\bf Avg}} \sum_k \left[ \left(\uadvtwo \cdot \eb_r \right) (\rho X_k)^{\nph,\pred} \right]
\end{equation}
\item For spherical geometry, first construct 
$\etarho^{{\rm cart},\nph} = [\rho'(\Ub\cdot\eb_r)]^{\nph}$ on Cartesian cell centers using:
\begin{equation}
\etarho^{{\rm cart},\nph} = \left[\left(\frac{\rho^{(1)}+\rho^{(2)}}{2}\right)-\left(\frac{\rho_0^n+\rho_0^{n+1}}{2}\right)\right] \cdot \left( \uadvtwo \cdot \eb_r \right).
\end{equation}
Then,
\begin{equation}
\etarho^{\nph,\star} = {\rm {\bf Avg}}\left(\etarho^{{\rm cart},\nph}\right).
\end{equation}
\end{description}

\section{Old Notes}

\section{Computing $w_0$ For Spherical Problems}
Recall that we want to solve
\begin{equation}
\frac{1}{r^2}\frac{\partial}{\partial r}(r^2 w_0) = \Sbar - \frac{1}{\gammaonebar p_0}\left(\frac{\partial p_0}{\partial t} + w_0\frac{\partial p_0}{\partial r}\right)
\end{equation}
First, note that $\partial p_0/\partial r = -\rho_0 g$.
The key observation here is that we can compute $\partial p_0/\partial t$ directly instead of making complicated substitutions involving $\eta_\rho$ to eliminate it.
Let's discretize this as follows:
\begin{equation}
\frac{1}{r_j^2}\left[\left(r^2 w_0\right)_{j+\myhalf} - \left(r^2 w_0\right)_{j-\myhalf}\right] = 
\Sbar_j - \frac{1}{\left(\gammaonebar p_0\right)_j}\left[\left(\frac{\partial p_0}{\partial t}\right)_j - \left(\frac{w_{0,j-\myhalf}+w_{0,j+\myhalf}}{2}\right)(\rho_0 g)_j\right]
\end{equation}
Rearranging this to solve for $w_{0,j+\myhalf}$ gives us an explicit update:
\begin{equation}
\left[\frac{r_{j+\myhalf}^2}{r_j^2} - \frac{(\rho_0 g)_j}{2\left(\gammaonebar p_0\right)_j}\right] w_{0,j+\myhalf} =
\left[\frac{r_{j-\myhalf}^2}{r_j^2} + \frac{(\rho_0 g)_j}{2\left(\gammaonebar p_0\right)_j}\right] w_{0,j-\myhalf}
+ \Sbar_j - \frac{1}{\left(\gammaonebar p_0\right)_j}\left(\frac{\partial p_0}{\partial t}\right)_j
\end{equation}
As before, once $\rho_0$ falls below $\rho_{\rm cutoff}$, we hold $r^2 w_0$ constant.\\ \\
The interface to the routine in the algorithm description is
{\bf Make $\mathbf{w_0}$}$[\rho_0,p_0,\partial p_0/\partial t,\gammaonebar,\Sbar] \rightarrow [w_0]$

\acknowledgements

The work at LBNL was supported by the U.S. Department of Energy's Scientific Discovery Through Advanced Computing (SciDAC) program under contract No. DE-AC02-05CH11231.
This research used resources of the National Energy Research Scientific Computing Center (NERSC), a U.S. Department of Energy Office of Science User Facility operated under Contract No. DE-AC02-05CH11231.

\bibliographystyle{aasjournal.bst}
\bibliography{references}

\end{document}
