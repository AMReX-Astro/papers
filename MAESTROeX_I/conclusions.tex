\section{Conclusions and Future Work}\label{sec:conclusions}
We have developed a new temporal integrator and spatial mapping options into our low Mach number solver, MAESTROeX.
The new AMReX-enabled code scales well on large fractions of supercomputers with multicore architectures.
Future software enhancements will include GPU implementation.
In particular, the AMReX-based companion code, the compressible CASTRO code \citep{CASTRO}, has recently ported hydrodynamics and reactions to GPUs \citep{CASTRO_GPU}.
We plan to leverage the newly implemented mechanisms for offloading compute kernels to GPUs inside of the AMReX software library itself.
Our future scientific investigations include convection in massive rotating stars, \citep{heger2000presupernova}, the convective Urca process in white dwarfs \citep{willcox2016type}, solar physics \citep{wood2018self}, and magnetohydrodynamics \citep{wood2015three,wood2011sun}.
Our future algorithmic enhancements include more accurate and higher-order multiphysics coupling strategies based on spectral deferred corrections \citep{dutt2000spectral,bourlioux2003high}.
This framework has been successfully used in terrestrial combustion \citep{pazner2016high,nonaka2018conservative}

\acknowledgements

The work at LBNL was supported by the U.S. Department of Energy's
Scientific Discovery Through Advanced Computing (SciDAC) program under
contract No. DE-AC02-05CH11231.  The work at Stony Brook was supported
by DOE/Office of Nuclear Physics grant DE-FG02-87ER40317 and through
the SciDAC program DOE grant DE-SC0017955.  This research used
resources of the National Energy Research Scientific Computing Center
(NERSC), a U.S. Department of Energy Office of Science User Facility
operated under Contract No. DE-AC02-05CH11231.


\software{AMReX \citep{AMReX, AMReX_JOSS}, StarKiller Microphysics \citep{starkiller}}
\facilities{NERSC}
