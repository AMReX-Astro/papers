%\documentclass[12pt, preprint]{aastex}
\documentclass[preprint,times,tighten]{aastex63}

\usepackage{epsf,color,amsmath}

\usepackage{cancel}

\newcommand{\sfrac}[2]{\mathchoice%
  {\kern0em\raise.5ex\hbox{\the\scriptfont0 #1}\kern-.15em/
    \kern-.15em\lower.25ex\hbox{\the\scriptfont0 #2}}
  {\kern0em\raise.5ex\hbox{\the\scriptfont0 #1}\kern-.15em/
    \kern-.15em\lower.25ex\hbox{\the\scriptfont0 #2}}
  {\kern0em\raise.5ex\hbox{\the\scriptscriptfont0 #1}\kern-.2em/
    \kern-.15em\lower.25ex\hbox{\the\scriptscriptfont0 #2}} {#1\!/#2}}

\newcommand{\myhalf}{\sfrac{1}{2}}
\newcommand{\nph}{{n+\myhalf}}
\newcommand{\nmh}{{n-\myhalf}}

\newcommand{\inp}{\mathrm{in}}
\newcommand{\outp}{\mathrm{out}}

% boldsymbol means bold italic
\newcommand{\eb}{{\bf{e}}}
\newcommand{\Ub}{{\bf{U}}}
\newcommand{\xb}{{\bf{x}}}
\newcommand{\kb}{{\bf{k}}}
\newcommand{\Vb}{{\bf{V}}_n}
\newcommand{\Vbhat}{{\bf{\widehat{V}}}_n}
\newcommand{\Omegab}{{\bf{\Omega}}}
\newcommand{\gb}{{\bf{g}}}
\newcommand{\rb}{{\bf{r}}}

\newcommand{\pb}{p_\mathrm{base}}
\newcommand{\epsdot}{\dot{\epsilon}}
\newcommand{\qburn}{q_\mathrm{burn}}
\newcommand{\rt}{\tilde{r}_0}


\newcommand{\nablab}{\mathbf{\nabla}}
\newcommand{\dt}{\Delta\ t}

\newcommand{\omegadot}{\dot{\omega}}
\newcommand{\enucdot}{\dot{e}_{nuc}}

\newcommand{\Hext}{H_{\rm ext}}
\newcommand{\Hnuc}{H_{\rm nuc}}
\newcommand{\kth}{k_{\rm th}}

\newcommand{\Gammaonebar}{\overline{\Gamma}_1}
\newcommand{\Sbar}{\overline{S}}

\newcommand{\etarho}{\eta_\rho}
\newcommand{\etarhoh}{\eta_{\rho~h}}

\newcommand{\Ubt}{\widetilde{\Ub}}
\newcommand{\wt}{\widetilde{w}}

\newcommand{\He}{$^4$He}
\newcommand{\C}{$^{12}$C}
\newcommand{\Fe}{$^{56}$Fe}

\newcommand{\isot}[2]{$^{#2}\mathrm{#1}$}
\newcommand{\isotm}[2]{{}^{#2}\mathrm{#1}}

\newcommand{\maestro}{{\sf MAESTRO}}
\newcommand{\castro}{{\sf Castro}}
\newcommand{\amrex}{{\sf AMReX}}
\newcommand{\pynucastro}{{\sf pynucastro}}

\newcommand{\avg}[1]{\overline{#1}}
\newcommand{\avgtwod}[1]{\langle~#1 \rangle}
\newcommand{\rms}[2]{\left(\delta#1\right)_{r_{#2}}}
\newcommand{\mymax}[1]{\left(#1\right)_{\rm max}}

\newcommand{\Tb}{\ensuremath{T_\mathrm{base}}}
\newcommand{\gcc}{\mathrm{g~cm^{-3} }}
\newcommand{\cms}{\mathrm{cm~s^{-1} }}

\newcommand{\half}{\frac{1}{2}}

\setlength{\marginparwidth}{0.5in}

\newcommand{\MarginPar}[1]{
    \marginpar{\vskip-\baselineskip%
               \raggedright%
               \tiny\sffamily%
               {\color{red}\hrule%
               \smallskip%
               #1\par%
               \smallskip%
               \hrule}}%
}

\newcommand{\AssignTo}[1]{
    \marginpar{\vskip-\baselineskip%
               \raggedright%
               \tiny\sffamily%
               {\color{blue}\hrule%
               \smallskip%
               #1\par%
               \smallskip%
               \hrule}}%
}

\begin{document}
%======================================================================
% Title
%======================================================================
\title{Dynamics of Laterally Propagating Flames in X-ray Bursts. II. Realistic Burning \& Rotation}

\shorttitle{Lateral Flame Dynamics II}
\shortauthors{Harpole et al.}

\author[0000-0002-1530-781X]{Alice Harpole}
\affiliation{Dept.\ of Physics and Astronomy, Stony Brook University,
             Stony Brook, NY 11794-3800}

\author[0000-0001-6191-4285]{Kiran Eiden}
\affiliation{Dept.\ of Physics and Astronomy, Stony Brook University,
             Stony Brook, NY 11794-3800}

\author[0000-0001-8401-030X]{Michael Zingale}
\affiliation{Dept.\ of Physics and Astronomy, Stony Brook University,
             Stony Brook, NY 11794-3800}
\affiliation{Center for Computational Astrophysics, Flatiron Institute, New York, NY 10010}

\author[0000-0003-2300-5165]{Donald Willcox}
\affiliation{Lawrence Berkeley National Laboratory, Berkeley, CA}

\author[0000-0002-6447-3603]{Yuri Cavecchi}
\affiliation{Mathematical Sciences and STAG Research Centre, University of Southampton, SO17 1BJ}

\author[0000-0003-0439-4556]{Max P.\ Katz}
\affiliation{NVIDIA Corp}


\correspondingauthor{Alice Harpole}
\email{alice.harpole@stonybrook.edu}


%======================================================================
% Abstract and Keywords
%======================================================================
\begin{abstract}
We continue to investigate laterally propagating flames in XRBs...


\end{abstract}

\keywords{X-ray bursts (1814), Nucleosynthesis (1131), Hydrodynamical simulations (767), Hydrodynamics (1963), Neutron stars (1108), Open source software (1866), Computational methods (1965)}

%======================================================================
% Introduction
%======================================================================
\section{Introduction}\label{Sec:Introduction}

Unlike our previous paper, all runs are now conducted without artificially boosting the speed of the flame in order to reduce the computational cost. This is possible because of the work done to offload the code to GPUs, where it runs significantly faster. We can therefore run the tests without any boosting. 


We investigate the effect of rotation rate on the flame. With the exception of (citation to the weird 11 Hz flame), most observations of XRBs come from sources with rotation rates of 200-600Hz \citep{bilous:2019,galloway:2020} \MarginPar{check whether this is true for XRBs or just burst oscillations}. There are a number of factors that could explain this lack of observations below 200 Hz. It could be that there is some physical process which inhibits the flame ignition and/or spread at lower rotation rates. It could be that burst oscillations at lower rotation rates are smaller in amplitude and therefore more difficult to detect. It could be that it's not got anything to do with the flame at all, but that neutron stars in accreting low mass X-ray binaries rarely have rotation rates below 200 Hz \MarginPar{look into this?}. 

Insert some stuff here about previous work on rotation 

In the following sections, we conduct a series of simulations at various rotation rates to investigate its effect on the flame. We find that at lower rotation rates, the flame itself becomes harder to ignite. We discuss the implications that this may have for burst physics and observations. 


%======================================================================
% Numerical Approach
%======================================================================
\section{Numerical Approach}\label{Sec:numerics}

We use the Castro hydrodynamics code \citep{castro} and the simulation framework introduced in \citet{flame_wave1}.
The main difference is that we no longer boost the flames by adjusting
the reaction rates and conductivities.  By porting \castro\ to GPUs,
we are able to run the unboosted flame simulations with fewer
resources than the boosted simulations used in our earlier paper.

We add the geometric source terms from \citet{bernard-champmartin}

%======================================================================
% Results
%======================================================================
\section{Simulations and Results}\label{Sec:results}


Things to show:

\begin{figure}[t]
\centering
\plotone{time_series_500}
\caption{Time series of the 500 Hz run showing $\bar{A}$.}
\end{figure}

\begin{figure}[t]
\centering
\plotone{time_series_1000}
\caption{Time series of the 1000 Hz run showing $\bar{A}$.}
\end{figure}

lateral profiles of X vs distance behind the flame

vertical profiles of the atmosphere behind the flame---how does it compare to isentropic?

what is the composition on the photosphere across the flame

different initial models?

different rotation rates 

look at flame speed, shape of flame, amount of mixing (e.g. it looks like slower rotation rates have less mixed flames as the hurricane is less strong.)

\begin{figure}[t]
	\centering
	\plottwo{speeds_1_100th_max.pdf}{speeds_1_1000th_max.pdf}
	\caption{Plots of flame front position vs.\ time for the 500 Hz and 1000 Hz runs. The left-hand plot shows the results when tracking the $\frac{1}{100} \max[\enucdot]$ contour, while the right-hand plot shows the results for the $\frac{1}{1000} \max[\enucdot]$ contour.}
\end{figure}

Looking at the phase plots in Figures~\ref{fig:urho}-\ref{fig:abar}, we see:
\begin{itemize}
    \item From the $u$-$\rho$ and $u$-$v$ plots, we see that the burning occurs in a region of high density where the fluid is moving in a circular motion. From the previous plots (see: streamlines?), this most likely corresponds to the leading edge of the flame. This feature is not (yet?) developed in the 250~Hz flame. (will this ever develop, or will it just fizzle out completely?)
    \item From the $\bar{A}$-$T$ plots in Figure \ref{fig:abar}, we can see that as the rotation rate increases, so does the peak temperature. We believe that the tracks in the plot correspond to reaction pathways in the reaction network. In the 1000~Hz plot, these tracks are somewhat disrupted, possibly due to the more vigorous mixing of the vortex at the flame front. (If this continues or if the rotation was cranked even higher, would the reactions get so disrupted that the burning is snuffed out, similar to what happens in combustion when the turbulent vortices become so small they penetrate the reaction zone and completely disrupt it?)
\end{itemize}

\begin{figure}[t]
    \centering
    \plotone{urho_plot_0_1}
    \caption{\label{fig:urho}Phase plot of the horizontal $x$-velocity $u$, density $\rho$ and the nuclear energy generation rate $\dot{e}_{\rm nuc}$ for the 250, 500 and 1000~Hz runs.}
\end{figure}

\begin{figure}[t]
    \centering
    \plotone{uv_plot_0_1}
    \caption{\label{fig:uv}Phase plot of the horizontal $x$-velocity $u$, vertical $y$-velocity $v$ and the nuclear energy generation rate $\dot{e}_{\rm nuc}$ for the 250, 500 and 1000~Hz runs.}
\end{figure}

\begin{figure}[t]
    \centering
    \plotone{abar_plot_0_1}
    \caption{\label{fig:abar}Phase plot of the mean molecular weight $\bar{A}$, temperature $T$ and the nuclear energy generation rate $\dot{e}_{\rm nuc}$ for the 250, 500 and 1000~Hz runs.}
\end{figure}
    


\acknowledgements \castro\ is open-source and freely available at
\url{http://github.com/AMReX-Astro/Castro}.  The problem setups used
here are available in the git repo as {\tt flame} and {\tt
  flame\_wave}.  The work at Stony Brook was supported by DOE/Office
of Nuclear Physics grant DE-FG02-87ER40317 and the SciDAC program DOE
grant DE-SC0017955.  MZ acknowledges support from the Simons
Foundation.  This research used resources of the National Energy
Research Scientific Computing Center, a DOE Office of Science User
Facility supported by the Office of Science of the U.~S.\ Department
of Energy under Contract No.\ DE-AC02-05CH11231.  This research used
resources of the Oak Ridge Leadership Computing Facility at the Oak
Ridge National Laboratory, which is supported by the Office of Science
of the U.S. Department of Energy under Contract No. DE-AC05-00OR22725,
awarded through the DOE INCITE program.  We thank NVIDIA Corporation
for the donation of a Titan X and Titan V GPU through their academic
grant program.  This research has made use of NASA's Astrophysics Data
System Bibliographic Services.

\facilities{NERSC, OLCF}

\software{AMReX \citep{amrex_joss},
          Castro \citep{castro},
          GCC (\url{https://gcc.gnu.org/}),
          linux (\url{https://www.kernel.org/}),
          matplotlib (\citealt{Hunter:2007}, \url{http://matplotlib.org/}),
          NumPy \citep{numpy,numpy2},
          python (\url{https://www.python.org/}),
          valgrind \citep{valgrind},
          VODE \citep{vode},
          yt \citep{yt}}



%======================================================================
% References
%======================================================================

\bibliographystyle{aasjournal}
\bibliography{ws}


\end{document}
