%\documentclass[12pt, preprint]{aastex}
\documentclass[preprint,times,tighten]{aastex63}

\usepackage{epsf,color,amsmath}

\usepackage{cancel}

\newcommand{\sfrac}[2]{\mathchoice%
  {\kern0em\raise.5ex\hbox{\the\scriptfont0 #1}\kern-.15em/
    \kern-.15em\lower.25ex\hbox{\the\scriptfont0 #2}}
  {\kern0em\raise.5ex\hbox{\the\scriptfont0 #1}\kern-.15em/
    \kern-.15em\lower.25ex\hbox{\the\scriptfont0 #2}}
  {\kern0em\raise.5ex\hbox{\the\scriptscriptfont0 #1}\kern-.2em/
    \kern-.15em\lower.25ex\hbox{\the\scriptscriptfont0 #2}} {#1\!/#2}}

\newcommand{\myhalf}{\sfrac{1}{2}}
\newcommand{\nph}{{n+\myhalf}}
\newcommand{\nmh}{{n-\myhalf}}

\newcommand{\inp}{\mathrm{in}}
\newcommand{\outp}{\mathrm{out}}

% boldsymbol means bold italic
\newcommand{\eb}{{\bf{e}}}
\newcommand{\Ub}{{\bf{U}}}
\newcommand{\xb}{{\bf{x}}}
\newcommand{\kb}{{\bf{k}}}
\newcommand{\Vb}{{\bf{V}}_n}
\newcommand{\Vbhat}{{\bf{\widehat{V}}}_n}
\newcommand{\Omegab}{{\bf{\Omega}}}
\newcommand{\gb}{{\bf{g}}}
\newcommand{\rb}{{\bf{r}}}

\newcommand{\pb}{p_\mathrm{base}}
\newcommand{\epsdot}{\dot{\epsilon}}
\newcommand{\qburn}{q_\mathrm{burn}}
\newcommand{\rt}{\tilde{r}_0}


\newcommand{\nablab}{\mathbf{\nabla}}
\newcommand{\dt}{\Delta\ t}

\newcommand{\omegadot}{\dot{\omega}}
\newcommand{\enucdot}{\dot{e}_{\rm nuc}}

\newcommand{\Hext}{H_{\rm ext}}
\newcommand{\Hnuc}{H_{\rm nuc}}
\newcommand{\kth}{k_{\rm th}}

\newcommand{\Gammaonebar}{\overline{\Gamma}_1}
\newcommand{\Sbar}{\overline{S}}

\newcommand{\etarho}{\eta_\rho}
\newcommand{\etarhoh}{\eta_{\rho~h}}

\newcommand{\Ubt}{\widetilde{\Ub}}
\newcommand{\wt}{\widetilde{w}}

\newcommand{\He}{$^4$He}
\newcommand{\C}{$^{12}$C}
\newcommand{\Fe}{$^{56}$Fe}

\newcommand{\isot}[2]{$^{#2}\mathrm{#1}$}
\newcommand{\isotm}[2]{{}^{#2}\mathrm{#1}}

\newcommand{\maestro}{{\sf MAESTRO}}
\newcommand{\castro}{{\sf Castro}}
\newcommand{\amrex}{{\sf AMReX}}
\newcommand{\pynucastro}{{\sf pynucastro}}

\newcommand{\avg}[1]{\overline{#1}}
\newcommand{\avgtwod}[1]{\langle~#1 \rangle}
\newcommand{\rms}[2]{\left(\delta#1\right)_{r_{#2}}}
\newcommand{\mymax}[1]{\left(#1\right)_{\rm max}}

\newcommand{\Tb}{\ensuremath{T_\mathrm{base}}}
\newcommand{\gcc}{\mathrm{g~cm^{-3} }}
\newcommand{\cms}{\mathrm{cm~s^{-1} }}

\newcommand{\half}{\frac{1}{2}}

% Approximately proportional to, from StackOverflow
\def\app#1#2{%
	\mathrel{%
		\setbox0=\hbox{$#1\sim$}%
		\setbox2=\hbox{%
			\rlap{\hbox{$#1\propto$}}%
			\lower1.1\ht0\box0%
		}%
		\raise0.25\ht2\box2%
	}%
}
\def\appropto{\mathpalette\app\relax}

\setlength{\marginparwidth}{0.5in}

\newcommand{\MarginPar}[1]{
    \marginpar{\vskip-\baselineskip%
               \raggedright%
               \tiny\sffamily%
               {\color{red}\hrule%
               \smallskip%
               #1\par%
               \smallskip%
               \hrule}}%
}

\newcommand{\AssignTo}[1]{
    \marginpar{\vskip-\baselineskip%
               \raggedright%
               \tiny\sffamily%
               {\color{blue}\hrule%
               \smallskip%
               #1\par%
               \smallskip%
               \hrule}}%
}

\begin{document}
%======================================================================
% Title
%======================================================================
\title{Dynamics of Laterally Propagating Flames in X-ray Bursts. II. Realistic Burning \& Rotation}

\shorttitle{Lateral Flame Dynamics II}
\shortauthors{Harpole et al.}

\author[0000-0002-1530-781X]{Alice Harpole}
\affiliation{Dept.\ of Physics and Astronomy, Stony Brook University,
             Stony Brook, NY 11794-3800}

\author[0000-0001-6191-4285]{Kiran Eiden}
\affiliation{Dept.\ of Physics and Astronomy, Stony Brook University,
             Stony Brook, NY 11794-3800}

\author[0000-0001-8921-3624]{Nicole Ford}
\affiliation{Lawrence Berkeley National Laboratory, Berkeley, CA}

\author[0000-0001-8401-030X]{Michael Zingale}
\affiliation{Dept.\ of Physics and Astronomy, Stony Brook University,
             Stony Brook, NY 11794-3800}
\affiliation{Center for Computational Astrophysics, Flatiron Institute, New York, NY 10010}

\author[0000-0003-2300-5165]{Donald Willcox}
\affiliation{Lawrence Berkeley National Laboratory, Berkeley, CA}

\author[0000-0002-6447-3603]{Yuri Cavecchi}
\affiliation{Mathematical Sciences and STAG Research Centre, University of Southampton, SO17 1BJ}

\author[0000-0003-0439-4556]{Max P.\ Katz}
\affiliation{NVIDIA Corp}


\correspondingauthor{Alice Harpole}
\email{alice.harpole@stonybrook.edu}


%======================================================================
% Abstract and Keywords
%======================================================================
\begin{abstract}
We continue to investigate laterally propagating flames in XRBs using
fully compressible hydrodynamics simulations.  In the current study we
relax the approximations used previously and explore the effect of
rotation rate and the thermal structure of the neutron star on the
flame.  We find that higher temperature atmospheres lead to faster
flames and observe acceleration of the flame front as it propagates
through the atmosphere.  All of the software used for these
simulations is freely available.


\end{abstract}

\keywords{X-ray bursts (1814), Nucleosynthesis (1131), Hydrodynamical simulations (767), Hydrodynamics (1963), Neutron stars (1108), Open source software (1866), Computational methods (1965)}

%======================================================================
% Introduction
%======================================================================
\section{Introduction}\label{Sec:Introduction}

Considerable evidence suggests that ignition in an X-ray burst starts
in a localized region and then spreads across the surface of the
neutron star~\citep{bhattacharyya:2007,chakraborty:2014}.  We continue
our study of flame spreading through fully compressible hydrodynamics
simulations of the flame.  Building on our previous
paper~\citep{flame_wave1}, we relax the approximations we made
previously (artificially boosting the speed of the flame in order to
reduce the computational cost) and explore how the flame properties
depend on rotation rate and the thermal structure of the neutron
star. This new set of realistic simulations is possible because of the
work done to offload our simulation code, \castro~\citep{castro_joss}, to GPUs, where it
runs significantly faster.


We investigate the effect of rotation rate on the flame. With the
exception of \citet{altamirano2010discovery}, most observations of
XRBs come from sources with rotation rates of 200-600Hz
\citep{bilous:2019,galloway:2020} \MarginPar{check whether this is
  true for XRBs or just burst oscillations}. There are a number of
factors that could explain this lack of observations below 200 Hz. It
could be that there is some physical process which inhibits the flame
ignition and/or spread at lower rotation rates. It could be that burst
oscillations at lower rotation rates are smaller in amplitude and
therefore more difficult to detect. It could be that it's not got
anything to do with the flame at all, but that neutron stars in
accreting low mass X-ray binaries rarely have rotation rates below 200
Hz \MarginPar{look into this?}.
 
Previous studies have found that rotation can have a significant
affect on the flame's propagation. As the rotation rate increases, the
Coriolis force whips the spreading flame up into a hurricane-like
structure \citep{spitkovsky2002,cavecchi:2013}. The stronger Coriolis
force leads to greater confinement of the hot accreted matter, leading
to easier ignition of the flame \citep{art-2015-cavecchi-etal}.

The temperature structure of the accreted fuel layer can also affect
the flame propagation.  \citet{Timmes00} showed that laminar helium flames 
have higher speeds when moving into hotter upstream fuel.
It is known that crustal heating is stronger at lower
accretion rates and weaker at higher accretion rates, due to the
effect of neutrino losses \citep{Cumming2006,johnston:2019}. In our
models, we keep the crust at a constant temperature, so by increasing
this temperature we can effectively increase the crustal heating and
mimic the effects of lower accretion rates. A shallow heating
mechanism of as yet unknown origin has been found necessary to
reproduce observed properties of XRBs
\citep{Deibel2015,Turlione2015,Keek2017}.

In the following sections, we conduct a series of simulations at various rotation rates to investigate its effect on the flame. We find that at lower rotation rates, the flame itself becomes harder to ignite. We discuss the implications that this may have for burst physics and observations. 


%======================================================================
% Numerical Approach
%======================================================================
\section{Numerical Approach}\label{Sec:numerics}
\MarginPar{mz}

We use the Castro hydrodynamics code \citep{castro,castro_joss} and
the simulation framework introduced in \citet{flame_wave1}.  The
current simulations are all performed in a two-dimensional
axisymmetric geometry.  For these axisymmetric simulations, we add a
missing geometric source terms from \citet{bernard-champmartin} that
captures the effects of the divergence of the flux operating on the
azimuthal unit vector.  This term is a small correction, but was
missing from our earlier simulations.  The simulations framework
initializes a fuel layer in hydrostatic equilibrium, laterally
blending a hot model on the left side of the domain (the coordinate
origin) and a cool model on the right.  The initial temperature
gradient between the hot and cool models drives a laterally
propagating flame through the cool fuel.  In our original set of
calculations, we artificially boosted the flame speed by adjusting the
conductivity and reaction rate to produce a flame moving 5--10$\times$
faster than the nominal laminar flame speed.  We also used high
rotation rates to reduce the lateral lengthscale at which the Coriolis
force balances the lateral flame spreading.  For the current
simulations, we no longer boost the flame speed---the true
conductivities (taken from \citealt{Timmes00}) and reaction rates are
used.  We continue to use a 13-isotope $\alpha$-chain to describe the
helium burning.  The port of \castro\ to GPUs~\citep{sc20_gpu} enables
us to run these simulations without the previous approximations, while
continuing to resolve the burning front.

The initial model is setup in the same fashion as described in
\citet{flame_wave1}.  In particular, we create a ``hot'' and ``cool''
hydrostatic model representing the ash and fuel states and blend the
two models laterally to create a hot region near the origin of
coordinates and a smooth transition to the cooler region at larger
radii.  The cool initial characterized by 3 temperatures:
$T_\mathrm{star}$ is the isothermal temperature of the underlying
neutron star, $T_\mathrm{hi}$ is the temperature at the base of the
fuel layer, and $T_\mathrm{lo}$ is the minimum temperature of the
atmosphere---the atmosphere structure is isentropic as it falls from
$T_\mathrm{hi}$ down to $T_\mathrm{lo}$.  For the hot model, we
replace $T_\mathrm{hi}$ with $T_\mathrm{hi} + \delta T$.  In the
calculations presented here, we explore the structure of the initial
models by varying these parameters.  All models have the same
peak temperature in the hot model, $T_\mathrm{star} + \delta T$.

\begin{figure}[t]
\centering
\plotone{hot}
\caption{\label{fig:hot_profiles} Initial temperature and density
  structure vs.\ height in the ``hot'' state.}
\end{figure}

\begin{figure}[t]
\centering
\plotone{cold}
\caption{\label{fig:cool_profiles} Initial temperature and density
  structure vs.\ height in the ``cool'' state.}
\end{figure}

For the current simulations, we explore a variety of initial rotation
rate and temperature conditions for the flame. The main parameters
describing the models are provided in Table
\ref{table:runs}. Figure~\ref{fig:hot_profiles} shows the
temperature and density structure for our hot models and
figure~\ref{fig:cool_profiles} shows the temperature and density
structure for the cool models.


\begin{deluxetable}{lrrrrr}
	\tablecaption{\label{table:runs} Rotation rate and temperature properties of the simulations.}
	\tablehead{\colhead{run} & \colhead{Rotation Rate (Hz)} & \colhead{$\delta T$ (K)} & \colhead{$T_\mathrm{hi}$ (K)} & \colhead{$T_\mathrm{star}$ (K)} & \colhead{$T_\mathrm{lo}$ (K)}} 
	\startdata
	{\tt F1000}     & $1000$ & $1.2\times 10^9$ & $2\times 10^8$ & $10^8$ & $8\times 10^6$ \\
	{\tt F500}      & $500$ & $1.2\times 10^9$ & $2\times 10^8$ & $10^8$ & $8\times 10^6$ \\
	{\tt F500\_2E8} & $500$ & $1.2\times 10^9$ & $2\times 10^8$ & $2\times 10^8$ & $8\times 10^6$ \\
	{\tt F500\_3E8} & $500$ & $1.1\times 10^9$ & $3\times 10^8$ & $3\times 10^8$ & $8\times 10^6$ \\
	{\tt F500\_4E8} & $500$ & $10^9$ & $4\times 10^8$ & $4\times 10^8$ & $8\times 10^6$ \\
	{\tt F250}      & $250$ & $1.2\times 10^9$ & $2\times 10^8$ & $10^8$ & $8\times 10^6$ \\
	\enddata
\end{deluxetable}



%======================================================================
% Results
%======================================================================
\section{Simulations and Results}\label{Sec:results}


We present 6 simulations in total, summarized in
Table~\ref{table:runs}.  These simulations encompass 3 different
rotation rates: 250~Hz, 500~Hz, and 1000~Hz, and for the 500~Hz run, 4
different temperature profiles.  In the following subsections, we look
at how the flame properties depends on the model parameters.

\subsection{Effect of Rotation Rate on Flame Structure}\label{ssec:rot_structure}

We run 3 models (F250, F500, and F1000) with the same initial model but differing rotation rates.

\begin{figure}[t]
\centering
\plotone{time_series_500}
\caption{\label{fig:time_series_500} Time series of the 500 Hz run showing $\bar{A}$.}
\end{figure}

\begin{figure}[t]
\centering
\plotone{time_series_1000}
\caption{\label{fig:time_series_1000} Time series of the 1000 Hz run showing $\bar{A}$.}
\end{figure}


Figures~\ref{fig:time_series_500} and \ref{fig:time_series_1000} show
time series of the mean molecular weight ($\bar{A}$) for the 500~Hz and 1000~Hz
runs.  Compared to those in \cite{flame_wave1}, ashes behind the flame
do not reach as high atomic weights.  This is not surprising, since
those early runs boosted the reaction rates.  Comparing these two new
runs, we see that the burning is much more evolved for the higher
rotation rate.  We believe that this is because the rotation confines
the initial perturbation and subsequent expansion from the burning
better, allowing the reactions to progress further.

MZ: lateral profiles of X vs distance behind the flame; 
vertical profiles of the atmosphere behind the flame---how does it compare to isentropic?

what is the composition on the photosphere across the flame

different rotation rates. 250 Hz failed to ignite, could indicate that another source of confinement (e.g.\ magnetic fields) would need to take
over at lower rotation rates to allow a burst to occur, at least for our current initial model. We've seen that increasing rotation rate can
increase nuclear reaction rate already

shape of flame, amount of mixing (e.g. it looks like slower rotation rates have less mixed flames as the hurricane is less strong.)

Don: look at ratio of enuc to e?

\subsection{Effect of Rotation Rate on Flame Propagation}\label{ssec:rot_propagation}

For the purpose of measuring the flame propagation speed and acceleration, we track the position of each of our flames
as a function of time. We define the position in terms of a particular energy generation rate ($\enucdot$) value, as we did in \citet{flame_wave1}.
To recapitulate: we first reduce the 2D $\enucdot$ data for each simulation run to a set of 1D radial profiles by averaging
over the vertical coordinate. After averaging, we take our reference $\enucdot$ value to be some fraction of the global $\enucdot$
maximum across all of these profiles. Since the flames in our simulations propagate rightward, we then search the region of
each profile to the right of the local $\enucdot$ maximum for the point where the $\enucdot$ first drops below this reference
value. This point gives us the location of our flame front.

In \citet{flame_wave1}, we used $0.1\%$ of the global $\enucdot$ maximum for our reference value. For the high temperature
unboosted flames, however, we found that the $\enucdot$ profiles failed to reach that small a value across the domain at most times,
which prevented us from getting reliable position measurements. We therefore use $1\%$ of the global $\enucdot$ maximum in this paper
rather than $0.1\%$. This is still sufficiently small that our measurements are not overly sensitive to turbulence and other local fluid
motions (the issue with simply tracking the local maximum), but allows us to avoid the pitfall encountered by the $0.1\%$ metric.

\begin{figure}[t]
	\centering
	\plotone{flame_speeds_rotation.pdf}
	\caption{\label{fig:flame_speeds_1} Flame front position vs.\ time for the standard ($10^8~\mathrm{K}$) 500 Hz and 1000 Hz runs. The dashed lines
		show quadratic least-squares fits to the data for $t \gtrsim 0.02~\mathrm{s}$.}
\end{figure}

\begin{deluxetable}{lrrr}
	\tablecaption{\label{table:flame_speeds} Flame speed and acceleration values measured for each simulation. $s_0$ is the speed at $t = 0$. The initial flame velocities and accelerations are derived from a quadratic least-squares fit to each of the datasets for times $t \gtrsim 0.02~\mathrm{s}$.}
	\tablehead{\colhead{run} & \colhead{$s_0$ (cm s$^{-1}$)} & \colhead{$a$ (cm s$^{-2}$)}} 
	\startdata
	{\tt F1000}     & $(3.41 \pm 0.01) \times 10^5$ & $(-1.03 \pm 0.13) \times 10^5$ \\
	{\tt F500}      & $(3.08 \pm 0.01) \times 10^5$ & $(5.41 \pm 0.19) \times 10^5$ \\
	{\tt F500\_2E8} & $(3.76 \pm 0.02) \times 10^5$ & $(1.74 \pm 0.17) \times 10^5$ \\
	{\tt F500\_3E8} & $(5.29 \pm 0.10) \times 10^5$ & $(3.57 \pm 0.02) \times 10^7$ \\
	\enddata
	%
\end{deluxetable}

Figure~\ref{fig:flame_speeds_1} gives position vs.\ time for the $\tt{F500}$ and $\tt{F1000}$ runs to show the dependence on rotation rate. As some of the hotter 500~Hz runs exhibit significant accelerations, we fit a quadratic curve to each set of data and calculate the acceleration and initial velocity from the fit parameters. These results are given in Table~\ref{table:flame_speeds}\MarginPar{How is the acceleration calculated for this table? I.e. Specifically, at what time / over what time window is it calculated?} \MarginPar{NF: I will update the error estimates}.

We found in \citet{flame_wave1} that the flame speed $s$ obeys:
\begin{equation}
	s \appropto L_R (\epsdot)^{\half} \propto \frac{(\epsdot)^{\half}}{\Omega}
\end{equation}
where $\epsdot$ is the specific energy generation rate and $L_R$ is the Rossby length. This was consistent with the results of
\citet{cavecchi:2013}, who noted that the flame speeds in their simulations scaled with the Rossby length. As seen in
Figure~\ref{fig:flame_speeds_1}, there is no obvious inverse scaling with rotation rate in this set of runs. We observed earlier,
however, that nuclear reactions progress more quickly at the higher rotation rate. This results in a higher $\enucdot$ --- up to 4
to 6 times higher near the burning front after the flame ignites --- which may counteract the reduction in flame speed from the
increased Coriolis confinement. We also observe that the 500 Hz run accelerates faster than the 1000 Hz run, which appears to experience a
small deceleration at early times. \MarginPar{This is speculation on my part -- does someone else have clear idea why?} This disparity
may be a direct result of the difference in Coriolis force. It could also be due to the 500 Hz flame taking more time to develop fully
and reach higher burning rates, resulting in an $\epsdot$ \MarginPar{Is this supposed to be enucdot???}increase over time.

% It might be worth plotting peak enucdot as a function of time for each run

Looking at the phase plots in Figures~\ref{fig:urho}-\ref{fig:abar}, we see:
\begin{itemize}
    \item From the $u$-$\rho$ and $u$-$v$ plots, we see that the burning occurs in a region of high density where the fluid is moving in a circular motion. From the previous plots (see: streamlines?), this most likely corresponds to the leading edge of the flame. This feature is not (yet?) developed in the 250~Hz flame. (will this ever develop, or will it just fizzle out completely?)
    \item From the $\bar{A}$-$T$ plots in Figure \ref{fig:abar}, we can see that as the rotation rate increases, so does the peak temperature. We believe that the tracks in the plot correspond to reaction pathways in the reaction network. In the 1000~Hz plot, these tracks are somewhat disrupted, possibly due to the more vigorous mixing of the vortex at the flame front. (If this continues or if the rotation was cranked even higher, would the reactions get so disrupted that the burning is snuffed out, similar to what happens in combustion when the turbulent vortices become so small they penetrate the reaction zone and completely disrupt it?)
\end{itemize}

\begin{figure}[t]
    \centering
    \plotone{urho_plot_0_1}
    \caption{\label{fig:urho}Phase plot of the horizontal $x$-velocity $u$, density $\rho$ and the nuclear energy generation rate $\dot{e}_{\rm nuc}$ for the 250, 500 and 1000~Hz runs.}
\end{figure}

\begin{figure}[t]
    \centering
    \plotone{uv_plot_0_1}
    \caption{\label{fig:uv}Phase plot of the horizontal $x$-velocity $u$, vertical $y$-velocity $v$ and the nuclear energy generation rate $\dot{e}_{\rm nuc}$ for the 250, 500 and 1000~Hz runs.}
\end{figure}

\begin{figure}[t]
    \centering
    \plotone{abar_plot_0_1}
    \caption{\label{fig:abar}Phase plot of the mean molecular weight $\bar{A}$, temperature $T$ and the nuclear energy generation rate $\dot{e}_{\rm nuc}$ for the 250, 500 and 1000~Hz runs.}
\end{figure}
    

\subsection{Effect of Temperature on Flame Structure}\label{ssec:temp_structure}

To explore the effect of different initial temperature configurations, we run four simulations fixed at a rotation rate of 500 Hz with temperatures as shown in Table \ref{table:500Hz_runs}. For all the 500 Hz simulations (with the exception of the coolest run, $\tt{F500}$), we set $T_\mathrm{star}$ = $T_\mathrm{hi}$, scaling $\delta T$ accordingly to maintain a consistent value of $T_\mathrm{hi} + \delta T$. If we let $T_\mathrm{star} < T_\mathrm{hi}$, the cooler neutron star surface might act as a heat sink and siphon away energy that would otherwise go into heating the burning layer.

\begin{figure}[t]
\centering
\plotone{compare_500Hz_abar}
\caption{\label{fig:compare_500Hz_abar} Comparison of $\bar{A}$ for 3 different 500~Hz models with different neutron star temperatures, $T_\star$, and resulting envelope structures.  Each flame is shown at 70~ms.} %This will be updated to show at 70ms
\end{figure}

Figure~\ref{fig:compare_500Hz_abar} shows three 500 Hz simulations with different initial temperature structures. The hottest of the three models, $\tt{F500\_3E8}$, has a faster propagating flame. It also appears to have a secondary burning region above the neutron star atmosphere layer (the atmosphere edge is located at $\sim 1.2\times 10^4$ cm). It is likely that a small amount of hot material broke out of the flame front and rose through the atmosphere.

In contrast to the models with $T_{\mathrm{star}} \leq 3 \times 10^8~\mathrm{K}$, $\tt{F500\_4E8}$ is so hot that the organized flame front structure is lost. The fuel layer essentially simmers in place, rather than forming a flame; this effect can occur in nature, for neutron stars with very high accretion rates \citep{fujimoto1981,bildsten1998thermonuclear,keek2009effect}. \MarginPar{NF: I remember Alice mentioning this a long time ago, but is there anything I can cite to support it?} This model burns so strongly that it is only run out to $40$ ms. Due to the lack of a clear burning front, we neglect to analyze the flame propagation speed and acceleration for this run. After an initial period of burning moves across the domain, residual burning continues and eventually ignites the entire fuel layer at late times, as shown in Figures \ref{fig:4e8_stacked_enuc} and \ref{fig:4e8_stacked_abar} for three snapshots taken in the last ten seconds of the simulation. In Figure~\ref{fig:4e8_stacked_enuc}, $\enucdot$ reaches values of $10^{18} - 10^{20}~\mathrm{erg}\,\mathrm{g}^{-1} \mathrm{s}^{-1}$ across the domain and at heights up to $\sim 0.5\times 10^4$ cm. There is still significant burning occuring even higher, with $\enucdot$ reaching $\sim 10^{15}~\mathrm{erg}\,\mathrm{g}^{-1} \mathrm{s}^{-1}$ at heights up to $\sim 0.8\times 10^4$ cm. For comparison, the next hottest run ($\tt{F500\_3E8}$) only reaches maximum $\enucdot$ values of $10^{18}~\mathrm{erg}\,\mathrm{g}^{-1} \mathrm{s}^{-1}$ and a maximum height of $\sim 0.5\times 10^4$ cm, even at the latest timesteps ($\sim 70~\mathrm{ms}$, when the flame is most developed). Significant $\enucdot$ values for $\tt{F500\_3E8}$ are confined to the flame front, rather than spanning the entire domain.
\MarginPar{NF: should I include a plot for this or is it ok to just say it?}. 
Figure~\ref{fig:4e8_stacked_abar} plots $\bar{A}$ for $\tt{F500\_4E8}$, and similarly shows burning that extends across the domain and high into the atmosphere and fuel layer, lacking the charactersitic flame structure shown in $\bar{A}$ plots for the lower temperature $500$ Hz runs (see Figure~\ref{fig:compare_500Hz_abar}). A distinct mass of material - resembling a more dramatic version of that shown in Figure~\ref{fig:compare_500Hz_abar} - appears to have broken out of the atmosphere near the axis of symmetry. %Can also add a note here about how high abar values are reached across the domain, maybe don or alice can help give a more detailed analysis of abar. Maybe I could also segue into Alice's phase plots


\begin{figure}[t]
	\centering
	\plotone{compare_500Hz_enuc_4e8.pdf}
	\caption{\label{fig:4e8_stacked_enuc} Time series of the 500~Hz run $\tt{F500\_4E8}$, with $T_{\mathrm{star}} = 4 \times 10^8~\mathrm{K}$, showing $\dot{e}_{nuc}$. This model burns so strongly that it is only run out to 40 ms; the snapshots shown here are at T = 30 ms, 35 ms, and 40 ms.}
\end{figure}

\begin{figure}[t]
	\centering
	\plotone{compare_500Hz_abar_4e8.pdf}
	\caption{\label{fig:4e8_stacked_abar} Time series of the 500~Hz run $\tt{F500\_4E8}$, with $T_{\mathrm{star}} = 4 \times 10^8~\mathrm{K}$, showing $\bar{A}$.}
\end{figure}

Looking at the phase plots in Figures~\ref{fig:urho_hot}-\ref{fig:abar_hot}, we see:
\begin{itemize}
    \item The $u$-$\rho$ plot in Figure~\ref{fig:urho_hot} shows that the horizontal velocity when $T_{\rm star} = 3\times10^8$~K is sufficiently greater than for the two cooler runs. This is consistent with the plot of the flame front position, where we saw that the hottest flame was significantly faster at later times than the cooler flames.
    \item The $u$-$v$ plot in Figure~\ref{fig:uv_hot} further demonstrates this, as again we see that the magnitude of the velocity of the flame for the hottest run is significantly larger. Also of interest in these plots is that the coolest run does not seem to have developed the characteristic vortex structure at the burning front (i.e. where $\dot{e}_{\rm nuc}$ is greatest) that can be clearly seen for the hotter runs.
    \item The $\overline{A}$-$T$ plot in Figure~\ref{fig:abar_hot} suggests that the actual reactions happening in the hottest run may be different than for the cooler runs. Not only is the flame itself significantly hotter overall, but there is a large causally connected region across a wide range of $\overline{A}$. This indicates that the hotter temperature has facilitated significant burning in reaction pathways that were not favored at the lower temperatures.
\end{itemize}

\begin{figure}[t]
    \centering
    \plotone{urho_plot_hotter}
    \caption{\label{fig:urho_hot}Phase plot of the horizontal $x$-velocity $u$, density $\rho$ and the nuclear energy generation rate $\dot{e}_{\rm nuc}$ for the 500~Hz runs with $T_{\rm star}$ of $10^8$~K, $2\times10^8$~K and $3\times10^8$~K.}
\end{figure}

\begin{figure}[t]
    \centering
    \plotone{uv_plot_hotter}
    \caption{\label{fig:uv_hot}Phase plot of the horizontal $x$-velocity $u$, vertical $y$-velocity $v$ and the nuclear energy generation rate $\dot{e}_{\rm nuc}$ for the 500~Hz runs with $T_{\rm star}$ of $10^8$~K, $2\times10^8$~K and $3\times10^8$~K.}
\end{figure}

\begin{figure}[t]
    \centering
    \plotone{abar_plot_hotter}
    \caption{\label{fig:abar_hot}Phase plot of the mean molecular weight $\bar{A}$, temperature $T$ and the nuclear energy generation rate $\dot{e}_{\rm nuc}$ for the 500~Hz runs with $T_{\rm star}$ of $10^8$~K, $2\times10^8$~K and $3\times10^8$~K.}
\end{figure}

\subsection{Effect of Temperature on Flame Propagation}\label{ssec:temp_prop}

The method for measuring flame propagation described in Section \ref{ssec:rot_propagation} is applied in Figure \ref{fig:flame_speeds_2} to the three 500 Hz runs with $T_{\mathrm{star}} \leq 3 \times 10^8~\mathrm{K}$. As the initial $T_{\mathrm{star}}$ is increased, the flame becomes greatly accelerated. The initial flame velocities and accelerations derived from a quadratic least-squares fit to each of the datasets are provided in Table \ref{table:flame_speeds}. The reason for the acceleration of the flame over time is not entirely clear. It could be that energy released from burning begins to dominate the flame's propagation as it evolves, increasing the flame speed over time. Another possibility is that the increased temperatures lead to enhanced turbulent mixing effects that pull in more fresh fuel for the flame to burn. Yet another possibility is that the higher initial $T_{\mathrm{star}}$ leads to a greater average temperature in the fuel layer over time, making it easier for the flame to burn the fuel and propagate.


\begin{figure}[t]
	\centering
	\plotone{500Hz_stacked_enuc1_2order.pdf}
	\caption{\label{fig:flame_speeds_2} Flame front position vs.\ time for the three 500 Hz runs with
		$T_{\mathrm{star}} \leq 3 \times 10^8~\mathrm{K}$. The dashed lines show quadratic least-squares fits to the data
		for $t \gtrsim 0.02~\mathrm{s}$. Note that, due to its rapid acceleration, $\tt{F500\_3E8}$ (orange) is only run out to $\sim 0.07~\mathrm{s}$ to avoid surpassing the domain boundary.}
\end{figure}


%======================================================================
% Conclusions
%======================================================================
\section{Discussion and Conclusions}\label{Sec:conclusions}

tl;dr: faster rotation and/or hotter surface temperature = more burning.

Next steps: 3-d, explore resolution more, H/He flames, MHD flames


\acknowledgments \castro\ is open-source and freely available at
\url{http://github.com/AMReX-Astro/Castro}.  The problem setup used
here is available in the git repo as {\tt
  flame\_wave}.  The work at Stony Brook was supported by DOE/Office
of Nuclear Physics grant DE-FG02-87ER40317.  This material is based upon work supported by the
U.S. Department of Energy, Office of Science, Office of Advanced
Scientific Computing Research and Office of Nuclear Physics, Scientific
Discovery through Advanced Computing (SciDAC) program under Award
Number DE-SC0017955.  This material is also based upon work supported by the U.S. Department
of Energy, Office of Science, Office of Advanced Scientific Computing Research, Department of
Energy Computational Science Graduate Fellowship under Award Number DE-SC0021110.
MZ acknowledges support from the Simons Foundation. 
This research used resources of the National Energy
Research Scientific Computing Center, a DOE Office of Science User
Facility supported by the Office of Science of the U.~S.\ Department
of Energy under Contract No.\ DE-AC02-05CH11231.  This research used
resources of the Oak Ridge Leadership Computing Facility at the Oak
Ridge National Laboratory, which is supported by the Office of Science
of the U.S. Department of Energy under Contract No. DE-AC05-00OR22725,
awarded through the DOE INCITE program.  We thank NVIDIA Corporation
for the donation of a Titan X and Titan V GPU through their academic
grant program.  This research has made use of NASA's Astrophysics Data
System Bibliographic Services.

\facilities{NERSC, OLCF}

\software{AMReX \citep{amrex_joss},
          Castro \citep{castro},
          GCC (\url{https://gcc.gnu.org/}),
          Jupyter \citep{Kluyver2016},
          linux (\url{https://www.kernel.org/}),
          matplotlib (\citealt{Hunter:2007}, \url{http://matplotlib.org/}),
          NumPy \citep{numpy,numpy2},
          python (\url{https://www.python.org/}),
          valgrind \citep{valgrind},
          VODE \citep{vode},
          yt \citep{yt}}



%======================================================================
% References
%======================================================================

\bibliographystyle{aasjournal}
\bibliography{ws}


\end{document}
