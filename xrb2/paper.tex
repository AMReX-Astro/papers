%\documentclass[12pt, preprint]{aastex}
\documentclass[preprint,times,tighten]{aastex63}

\usepackage{epsf,color,amsmath}

\usepackage{cancel}

\newcommand{\sfrac}[2]{\mathchoice%
  {\kern0em\raise.5ex\hbox{\the\scriptfont0 #1}\kern-.15em/
    \kern-.15em\lower.25ex\hbox{\the\scriptfont0 #2}}
  {\kern0em\raise.5ex\hbox{\the\scriptfont0 #1}\kern-.15em/
    \kern-.15em\lower.25ex\hbox{\the\scriptfont0 #2}}
  {\kern0em\raise.5ex\hbox{\the\scriptscriptfont0 #1}\kern-.2em/
    \kern-.15em\lower.25ex\hbox{\the\scriptscriptfont0 #2}} {#1\!/#2}}

\newcommand{\myhalf}{\sfrac{1}{2}}
\newcommand{\nph}{{n+\myhalf}}
\newcommand{\nmh}{{n-\myhalf}}

\newcommand{\inp}{\mathrm{in}}
\newcommand{\outp}{\mathrm{out}}

% boldsymbol means bold italic
\newcommand{\eb}{{\bf{e}}}
\newcommand{\Ub}{{\bf{U}}}
\newcommand{\xb}{{\bf{x}}}
\newcommand{\kb}{{\bf{k}}}
\newcommand{\Vb}{{\bf{V}}_n}
\newcommand{\Vbhat}{{\bf{\widehat{V}}}_n}
\newcommand{\Omegab}{{\bf{\Omega}}}
\newcommand{\gb}{{\bf{g}}}
\newcommand{\rb}{{\bf{r}}}

\newcommand{\pb}{p_\mathrm{base}}
\newcommand{\epsdot}{\dot{\epsilon}}
\newcommand{\qburn}{q_\mathrm{burn}}
\newcommand{\rt}{\tilde{r}_0}


\newcommand{\nablab}{\mathbf{\nabla}}
\newcommand{\dt}{\Delta\ t}

\newcommand{\omegadot}{\dot{\omega}}
\newcommand{\enucdot}{\dot{e}_{\rm nuc}}

\newcommand{\Hext}{H_{\rm ext}}
\newcommand{\Hnuc}{H_{\rm nuc}}
\newcommand{\kth}{k_{\rm th}}

\newcommand{\Gammaonebar}{\overline{\Gamma}_1}
\newcommand{\Sbar}{\overline{S}}

\newcommand{\etarho}{\eta_\rho}
\newcommand{\etarhoh}{\eta_{\rho~h}}

\newcommand{\Ubt}{\widetilde{\Ub}}
\newcommand{\wt}{\widetilde{w}}

\newcommand{\He}{$^4$He}
\newcommand{\C}{$^{12}$C}
\newcommand{\Fe}{$^{56}$Fe}

\newcommand{\isot}[2]{$^{#2}\mathrm{#1}$}
\newcommand{\isotm}[2]{{}^{#2}\mathrm{#1}}

\newcommand{\maestro}{{\sf MAESTRO}}
\newcommand{\castro}{{\sf Castro}}
\newcommand{\amrex}{{\sf AMReX}}
\newcommand{\pynucastro}{{\sf pynucastro}}

\newcommand{\avg}[1]{\overline{#1}}
\newcommand{\avgtwod}[1]{\langle~#1 \rangle}
\newcommand{\rms}[2]{\left(\delta#1\right)_{r_{#2}}}
\newcommand{\mymax}[1]{\left(#1\right)_{\rm max}}

\newcommand{\Tb}{\ensuremath{T_\mathrm{base}}}
\newcommand{\gcc}{\mathrm{g~cm^{-3} }}
\newcommand{\cms}{\mathrm{cm~s^{-1} }}

\newcommand{\half}{\frac{1}{2}}

% Approximately proportional to, from StackOverflow
\def\app#1#2{%
	\mathrel{%
		\setbox0=\hbox{$#1\sim$}%
		\setbox2=\hbox{%
			\rlap{\hbox{$#1\propto$}}%
			\lower1.1\ht0\box0%
		}%
		\raise0.25\ht2\box2%
	}%
}
\def\appropto{\mathpalette\app\relax}

\setlength{\marginparwidth}{0.5in}

\newcommand{\MarginPar}[1]{
    \marginpar{\vskip-\baselineskip%
               \raggedright%
               \tiny\sffamily%
               {\color{red}\hrule%
               \smallskip%
               #1\par%
               \smallskip%
               \hrule}}%
}

\newcommand{\AssignTo}[1]{
    \marginpar{\vskip-\baselineskip%
               \raggedright%
               \tiny\sffamily%
               {\color{blue}\hrule%
               \smallskip%
               #1\par%
               \smallskip%
               \hrule}}%
}

\begin{document}
%======================================================================
% Title
%======================================================================
\title{Dynamics of Laterally Propagating Flames in X-ray Bursts. II. Realistic Burning \& Rotation}

\shorttitle{Lateral Flame Dynamics II}
\shortauthors{Harpole et al.}

\author[0000-0002-1530-781X]{A.\ Harpole}
\affiliation{Dept.\ of Physics and Astronomy, Stony Brook University,
             Stony Brook, NY 11794-3800}

\author[0000-0001-8921-3624]{N.\ M.\ Ford}
\affiliation{Lawrence Berkeley National Laboratory, Berkeley, CA}

\author[0000-0001-6191-4285]{K.\ Eiden}
\affiliation{Dept.\ of Astronomy, University of California, Berkeley,
	CA 94720-3411}
\affiliation{Dept.\ of Physics and Astronomy, Stony Brook University,
	Stony Brook, NY 11794-3800}

\author[0000-0001-8401-030X]{M.\ Zingale}
\affiliation{Dept.\ of Physics and Astronomy, Stony Brook University,
             Stony Brook, NY 11794-3800}
\affiliation{Center for Computational Astrophysics, Flatiron Institute, New York, NY 10010}

\author[0000-0003-2300-5165]{D.\ E.\ Willcox}
\affiliation{Lawrence Berkeley National Laboratory, Berkeley, CA}

\author[0000-0002-6447-3603]{Y.\ Cavecchi}
\affiliation{Mathematical Sciences and STAG Research Centre, University of Southampton, SO17 1BJ}

\author[0000-0003-0439-4556]{M.\ P.\ Katz}
\affiliation{NVIDIA Corp}


\correspondingauthor{A.\ Harpole}
\email{alice.harpole@stonybrook.edu}


%======================================================================
% Abstract and Keywords
%======================================================================
\begin{abstract}
We continue to investigate laterally propagating flames in XRBs using
fully compressible hydrodynamics simulations.  In the current study we
relax the approximations used previously and explore the effect of
rotation rate and the thermal structure of the neutron star on the
flame.  We find that higher temperature atmospheres lead to faster
flames and observe acceleration of the flame front as it propagates
through the atmosphere.  All of the software used for these
simulations is freely available.


\end{abstract}

\keywords{X-ray bursts (1814), Nucleosynthesis (1131), Hydrodynamical simulations (767), Hydrodynamics (1963), Neutron stars (1108), Open source software (1866), Computational methods (1965)}

%======================================================================
% Introduction
%======================================================================
\section{Introduction}\label{Sec:Introduction}

Considerable evidence suggests that ignition in an X-ray burst starts
in a localized region and then spreads across the surface of the
neutron star~\citep{bhattacharyya:2007,chakraborty:2014}.  We continue
our study of flame spreading through fully compressible hydrodynamics
simulations of the flame.  Building on our previous
paper~\citep{flame_wave1}, we relax the approximations we made
previously (artificially boosting the speed of the flame in order to
reduce the computational cost) and explore how the flame properties
depend on rotation rate and the thermal structure of the neutron
star. This new set of realistic simulations is possible because of the
work done to offload our simulation code, \castro~\citep{castro_joss}, to GPUs, where it
runs significantly faster.


We investigate the effect of rotation rate on the flame. With the
exception of \citet{altamirano2010discovery}, most observations of
XRBs come from sources with rotation rates of 200-600Hz
\citep{bilous:2019,galloway:2020} \MarginPar{check whether this is
true for XRBs or just burst oscillations}. There are a number of
factors that could explain this lack of observations below 200 Hz. It
could be that there is some physical process which inhibits the flame
ignition and/or spread at lower rotation rates. It could be that burst
oscillations at lower rotation rates are smaller in amplitude and
therefore more difficult to detect. It could be that it does not have
anything to do with the flame at all, but that neutron stars in
accreting low mass X-ray binaries rarely have rotation rates below 200
Hz \MarginPar{look into this?}.
 
Previous studies have found that rotation can have a significant
effect on the flame's propagation. As the rotation rate increases, the
Coriolis force whips the spreading flame up into a hurricane-like
structure \citep{spitkovsky2002,cavecchi:2013}. The stronger Coriolis
force leads to greater confinement of the hot accreted matter, leading
to easier ignition of the flame \citep{art-2015-cavecchi-etal}.

The temperature structure of the accreted fuel layer can also affect
the flame propagation.  \citet{Timmes00} showed that laminar helium flames 
have higher speeds when moving into hotter upstream fuel.
It is known that crustal heating is stronger at lower
accretion rates and weaker at higher accretion rates, due to the
effect of neutrino losses \citep{Cumming2006,johnston:2019}. 
A shallow heating
mechanism of as yet unknown origin has been found necessary to
reproduce observed properties of XRBs in 1D simulations
\citep{Deibel2015,Turlione2015,Keek2017}.
In our
models, we keep the crust at a constant temperature, so by increasing
this temperature we can effectively increase the crustal heating, shallow heating
and/or
mimic the effects of lower accretion rates. 

In the following sections, we conduct a series of simulations at various rotation 
rates and temperatures to investigate their effects on the flame. We find that 
at lower rotation rates, the flame itself becomes harder to ignite. At higher 
rotation rates, nuclear burning is enhanced and the flame propagates more quickly. 
At higher crustal temperatures, burning is greatly enhanced and the flame accelerates as 
it propagates. We discuss the implications that this may have for burst physics 
and observations. 


%======================================================================
% Numerical Approach
%======================================================================
\section{Numerical Approach}\label{Sec:numerics}

We use the Castro hydrodynamics code \citep{castro,castro_joss} and
the simulation framework introduced in \citet{flame_wave1}.  The
current simulations are all performed in a two-dimensional
axisymmetric geometry.  For these axisymmetric simulations, we add an
additional geometric source term from \citet{bernard-champmartin} that
captures the effects of the divergence of the flux operating on the
azimuthal unit vector.  This term is a small correction, but was
missing from our previous simulations.  The simulation framework
initializes a fuel layer in hydrostatic equilibrium, laterally
blending a hot model on the left side of the domain (the coordinate
origin) and a cool model on the right.  The initial temperature
gradient between the hot and cool models drives a laterally
propagating flame through the cool fuel.  In our original set of
calculations, we artificially boosted the flame speed by adjusting the
conductivity and reaction rate to produce a flame moving 5--10$\times$
faster than the nominal laminar flame speed.  We also used high
rotation rates to reduce the lateral lengthscale at which the Coriolis
force balances the lateral flame spreading in order to reduce the size 
of the simulation domain.  For the current
simulations, we no longer boost the flame speed---the true
conductivities (taken from \citealt{Timmes00}) and reaction rates are
used.  We continue to use a 13-isotope $\alpha$-chain to describe the
helium burning.  The port of \castro\ to GPUs~\citep{sc20_gpu} 
significantly improved its overall performance, enabling
us to run these simulations without the previous approximations while
continuing to resolve the burning front.

The initial model is set up in the same fashion as described in
\citet{flame_wave1}.  In particular, we create a ``hot'' and ``cool''
hydrostatic model representing the ash and fuel states and blend the
two models laterally to create a hot region near the origin of
coordinates and a smooth transition to the cooler region at larger
radii.  The cool initial model is characterized by 3 temperatures:
$T_\mathrm{star}$ is the isothermal temperature of the underlying
neutron star, $T_\mathrm{hi}$ is the temperature at the base of the
fuel layer, and $T_\mathrm{lo}$ is the minimum temperature of the
atmosphere. The atmosphere structure is isentropic as it falls from
$T_\mathrm{hi}$ down to $T_\mathrm{lo}$.  For the hot model, we
replace $T_\mathrm{hi}$ with $T_\mathrm{hi} + \delta T$.  In the
calculations presented here, we explore the structure of the initial
models by varying these parameters.  All models have the same
peak temperature in the hot model, $T_\mathrm{star} + \delta T$.

\begin{figure}[t]
\centering
\plotone{hot}
\caption{\label{fig:hot_profiles} Initial temperature and density
  structure vs.\ height in the ``hot'' state.}
\end{figure}

\begin{figure}[t]
\centering
\plotone{cold}
\caption{\label{fig:cool_profiles} Initial temperature and density
  structure vs.\ height in the ``cool'' state.}
\end{figure}

For the current simulations, we explore a variety of initial rotation
rate and temperature conditions for the flame. The main parameters
describing the models are provided in Table~\ref{table:runs}. Figure~\ref{fig:hot_profiles} shows the
temperature and density structure for our hot models and
figure~\ref{fig:cool_profiles} shows the temperature and density
structure for the cool models.


\begin{deluxetable}{lrrrrr}
	\tablecaption{\label{table:runs} Rotation rate and temperature properties of the simulations. In the left-hand column we list the names we shall use to refer to each simulation throughout this paper.}
	\tablehead{\colhead{run} & \colhead{Rotation Rate (Hz)} & \colhead{$\delta T$ (K)} & \colhead{$T_\mathrm{hi}$ (K)} & \colhead{$T_\mathrm{star}$ (K)} & \colhead{$T_\mathrm{lo}$ (K)}} 
	\startdata
	{\tt F1000}     & $1000$ & $1.2\times 10^9$ & $2\times 10^8$ & $10^8$ & $8\times 10^6$ \\
	{\tt F500}      & $500$ & $1.2\times 10^9$ & $2\times 10^8$ & $10^8$ & $8\times 10^6$ \\
	{\tt F500\_2E8} & $500$ & $1.2\times 10^9$ & $2\times 10^8$ & $2\times 10^8$ & $8\times 10^6$ \\
	{\tt F500\_3E8} & $500$ & $1.1\times 10^9$ & $3\times 10^8$ & $3\times 10^8$ & $8\times 10^6$ \\
	{\tt F500\_4E8} & $500$ & $10^9$ & $4\times 10^8$ & $4\times 10^8$ & $8\times 10^6$ \\
	{\tt F250}      & $250$ & $1.2\times 10^9$ & $2\times 10^8$ & $10^8$ & $8\times 10^6$ \\
	\enddata
\end{deluxetable}



%======================================================================
% Results
%======================================================================
\section{Simulations and Results}\label{Sec:results}


We present six simulations in total, summarized in
Table~\ref{table:runs}.  These simulations encompass three different
rotation rates: 250~Hz, 500~Hz, and 1000~Hz, and for the 500~Hz run, four
different temperature profiles.  In the following subsections, we look
at how the flame properties depend on the model parameters.  All
simulations are run in a domain of $1.8432\times 10^5~\mathrm{cm} \times
3.072\times 10^4~\mathrm{cm}$ with a coarse grid of $1152 \times 192$
zones and two levels of refinement (the first level refining the resolution 
by factor of four, and the second by a factor of two again).  This gives a
fine-grid resolution of $20~\mathrm{cm}$.  In these simulations, refinement 
is carried out in all zones within the atmosphere
with $\rho > 2.5\times 10^4~\gcc$.

\subsection{Effect of Rotation Rate on Flame Structure}\label{ssec:rot_structure}

We run three models ({\tt F250}, {\tt F500}, and {\tt F1000}) with the same initial model in terms 
of temperature but differing rotation rates. {\color{blue}We saw in \citet{flame_wave1} that 
increasing the 
neutron star rotation rate reduces the horizontal lengthscale of the flame. An estimate of 
this lengthscale is given by the Rossby radius of deformation, $L_R$. The Rossby radius may be 
thought of as the scale over which the balance between the Coriolis force and horizontal pressure 
gradient becomes important, and is approximately:
\begin{equation}
	\label{eqn:rossby}
	L_R \approx \frac{\sqrt{g H_0}}{\Omega},
\end{equation}
where $g$ is gravitational acceleration, $H_0$ is the atmospheric scale height, and $\Omega$ is the 
neutron star rotation rate. In Figure \ref{fig:time_series_enuc_500} and Figure 
\ref{fig:time_series_enuc_1000}, we use the energy generation rate $\dot{e}_\mathrm{nuc}$ at $50~\mathrm{ms}$ and $100~\mathrm{ms}$ for {\tt F500} and {\tt F1000} to discern the horizontal extent of the flame at different rotation rates. Taking the rightmost edge of the bright teal/green region (where significant energy generation is occuring) as the leading edge of the flame in each plot, we see that the horizontal extent of the $\tt{F1000}$ flame appears to be 
reduced compared to the lower rotation $\tt{F500}$ run. From Equation \ref{eqn:rossby}, increasing 
the rotation rate from 500 Hz to 1000 Hz should decrease $L_R$ by a factor of two, and the greater 
confinement from the Coriolis force should reduce the horizontal extent of the flame by a similar 
factor. The Rossby radius is not a perfect measure of this horizontal lengthscale, however, and 
here the difference is not a factor of two. The simulations seem to follow the theoretical
scaling described in Equation \ref{eqn:rossby} more closely at higher rotation rates (1000 Hz and 
up), based on the results of \citet{flame_wave1}.}

\begin{figure}[t]
\centering
\plotone{time_series_enuc_500}
\caption{\label{fig:time_series_enuc_500} Time series of the 500~Hz run {\tt F500} showing the nuclear energy generation rate, $\dot{e}_\mathrm{nuc}$.}
\end{figure}


\begin{figure}[t]
\centering
\plotone{time_series_enuc_1000}
\caption{\label{fig:time_series_enuc_1000} Time series of the 1000~Hz run {\tt F1000} showing the nuclear energy generation rate, $\dot{e}_\mathrm{nuc}$.}
\end{figure}

\begin{figure}[t]
\centering
\plotone{time_series_500}
\caption{\label{fig:time_series_500} Time series of the 500~Hz run {\tt F500} showing the mean molecular weight, $\bar{A}$.}
\end{figure}


\begin{figure}[t]
\centering
\plotone{time_series_1000}
\caption{\label{fig:time_series_1000} Time series of the 1000~Hz run {\tt F1000} showing the mean molecular weight, $\bar{A}$.}
\end{figure}

The {\tt F500} and {\tt F1000} runs both qualitatively resemble the flame structure in 
\citet{flame_wave1} --- a laterally propagating flame that is lifted off of the bottom of the fuel 
layer --- but they differ in their burning structures. Figures~\ref{fig:time_series_500} and 
\ref{fig:time_series_1000} show time series of the mean molecular weight, $\bar{A}$, for the {\tt 
F500} and {\tt F1000} runs.  Compared to those in \cite{flame_wave1}, ashes behind the flame do not 
reach as high atomic weights. This is not surprising, since those previous runs artificially 
boosted the reaction rates.  Comparing these two new runs, the burning is much more evolved for the 
higher rotation rate, and the ash is actually able to move ahead of the flame front (visible in the 
Figure  \ref{fig:time_series_1000} 100 ms snapshot). We believe that this is because the increased 
rotation better confines the initial perturbation and subsequent expansion from the burning, 
increasing the temperature and density in the flame front such that the reaction rate increases, 
which allows the reactions to progress further. {\color{blue} The $\dot{e}_\mathrm{nuc}$ plots in Figure \ref{fig:time_series_enuc_1000} also support this interpretation, with the interior of the {\tt F1000} flame reaching higher $\dot{e}_\mathrm{nuc}$ values than for the {\tt F500} run in Figure \ref{fig:time_series_enuc_500}.} In contrast to {\tt F500} and {\tt F1000}, the 
lowest rotation run --- {\tt F250} --- failed to ignite. The lack of ignition for {\tt F250} also 
aligns with the reasoning given above, with the lower rotation in this case potentially leading to 
insufficient confinement such that the temperature and density required for ignition is not 
achieved. In this scenario, another source of confinement (e.g.\ magnetic fields) would need to 
take over at lower rotation rates to allow a burst to occur, at least for the initial model used 
here. 

Burning in the {\tt F500} and {\tt F1000} runs is concentrated in a dense region with circular 
motion. Figure \ref{fig:urho} compares the horizontal $x$-velocity $u$, density $\rho$ and the 
nuclear energy generation rate $\dot{e}_{\rm nuc}$ for the {\tt F250}, {\tt F500}, and {\tt F1000} 
runs. Most of the burning for each of the simulations occurs in a high density region $\rho > 3 
\times 10^5 {\rm g\,cm^{-3}}$. The fluid in this dense, high energy generation region undergoes 
vortical motion, shown in the Figure \ref{fig:uv} phase plots comparing $u$, $v$ and $\dot{e}_{\rm 
nuc}$. From the previous plots (see: streamlines/baroclinicity?) \MarginPar{NF: need to add 
streamline plots?}, this most likely corresponds to the leading edge of the flame where fresh fuel 
is being entrained. This feature is not developed in the 250~Hz flame; it could potentially develop 
at later times (past the point at which we terminated our simulation), or the flame could just 
fizzle out and fail to ignite entirely. 

The mean molecular weight $\bar{A}$ within each of our simulations seems to grow along defined tracks confined to certain temperatures $T$, as shown in the Figure \ref{fig:abar} phase plots. We believe that the tracks in the plot correspond to reaction pathways in the reaction network. Comparing Figure \ref{fig:abar} to \citet{flame_wave1}, these tracks are much more neat and clearly defined. The ``messiness" of the tracks may be dependent on how mixed the flame interior is. Since these new simulations are un-boosted, they may be inherently less mixed than those in \citet{flame_wave1}. {\tt F1000} aligns with this interpretation: its $\bar{A}$ tracks are somewhat disrupted compared to the slower rotation runs, possibly due to the more vigorous mixing of the vortex at the flame front. Comparing the different runs, we also see that as the rotation rate increases, so does the peak temperature. This makes sense if higher rotation leads to a more concentrated, intense vortex near the flame front. It also agrees with our earlier interpretation of the enhanced burning seen in Figure \ref{fig:time_series_1000} for {\tt F1000}. 

\begin{figure}[t]
    \centering
    \plotone{urho_plot_0_1}{}
    \caption{\label{fig:urho}Phase plot of the horizontal $x$-velocity $u$, density $\rho$ and the nuclear energy generation rate $\dot{e}_{\rm nuc}$ for the 250, 500 and 1000~Hz runs ({\tt F250}, {\tt F500} and {\tt F1000}) at $t = 100$~ms. The slightly lower $\dot{e}_{\rm nuc}$ values along the $u = 0$ axis are most likely a numerical artifact related to the density gradient setup and finite resolution of these simulations.}
\end{figure}

\begin{figure}[t]
    \centering
    \plotone{uv_plot_0_1}
    \caption{\label{fig:uv}Phase plot of the horizontal $x$-velocity $u$, vertical $y$-velocity $v$ and the nuclear energy generation rate $\dot{e}_{\rm nuc}$ for the 250, 500 and 1000~Hz runs ({\tt F250}, {\tt F500} and {\tt F1000}) at $t = 100$~ms.}
\end{figure}

\begin{figure}[t]
    \centering
    \plotone{abar_plot_0_1}
    \caption{\label{fig:abar}Phase plot of the mean molecular weight $\bar{A}$, temperature $T$ and the nuclear energy generation rate $\dot{e}_{\rm nuc}$ for the 250, 500 and 1000~Hz runs ({\tt F250}, {\tt F500} and {\tt F1000}) at $t = 100$~ms.}
\end{figure}

\MarginPar{Mike wants to check vertical profiles of atmosphere behind flame}

\subsection{Effect of Rotation Rate on Flame Propagation}\label{ssec:rot_propagation}

For the purpose of measuring the flame propagation speed and acceleration, we track the position of 
each of our flames as a function of time. We define the position in terms of a specific value of 
the energy generation rate, $\enucdot$, as we did in \citet{flame_wave1}. To recapitulate: we first 
reduce the 2D $\enucdot$ data for each simulation run to a set of 1D radial profiles by averaging
over the vertical coordinate. After averaging, we take our reference $\enucdot$ value to be some 
fraction of the global $\enucdot$ maximum across all of these profiles. Since the flames in our 
simulations propagate in the positive radial direction, we then search the region of each profile 
at greater radius than the local $\enucdot$ maximum for the point where the $\enucdot$ first drops 
below this reference value. This point gives us the location of our flame front.

In \citet{flame_wave1}, we used $0.1\%$ of the global $\enucdot$ maximum for our reference value. 
For the high temperature unboosted flames, however, we found that the $\enucdot$ profiles failed to 
reach that small a value across the domain at most times, which prevented us from getting reliable 
position measurements. We therefore use $1\%$ of the global $\enucdot$ maximum in this paper
rather than $0.1\%$. This is still sufficiently small that our measurements are not overly 
sensitive to turbulence and other local fluid motions (the issue with simply tracking the local 
maximum), but allows us to avoid the pitfall encountered by the $0.1\%$ metric.

\begin{deluxetable}{lrrr}
	\tablecaption{\label{table:flame_speeds} Flame speed and acceleration values measured for each 
	simulation. {\color{blue}$v_0$} is the flame velocity {\color{blue}in the $+r$ direction} at $t 
	= 0$. The initial flame 
	velocities and accelerations are derived from a quadratic least-squares fit to each of the 
	datasets for times $t \gtrsim 20~\mathrm{ms}$. {\color{blue} Using these fit parameters we calculate the velocity at $t = 50~\mathrm{ms}$ ($v_{50}$), at which point the flames have reached steady propagation.}}
	\tablehead{\colhead{run} & \colhead{$v_0$ (cm s$^{-1}$)} & \colhead{$a$ (cm s$^{-2}$)} & \colhead{$v_{50}$ (cm s$^{-1}$)}} 
	\startdata
  {\tt F1000}     & $(3.414 \pm 0.008) \times 10^5$ & $(-1.03 \pm 0.13) \times 10^5$ & $(3.362 \pm 0.010) \times 10^5$\\
  {\tt F500}      & $(3.077 \pm 0.013) \times 10^5$ & $(5.41 \pm 0.19) \times 10^5$ & $(3.347 \pm 0.016) \times 10^5$\\
  {\tt F500\_2E8} & $(3.760 \pm 0.015) \times 10^5$ & $(1.74 \pm 0.25) \times 10^5$ & $(3.847 \pm 0.020) \times 10^5$\\
  {\tt F500\_3E8} & $(-5.293 \pm 0.100) \times 10^5$ & $(3.57 \pm 0.02) \times 10^7$ & $(1.258 \pm 0.015) \times 10^6$\\
	\enddata
	%
\end{deluxetable}

\begin{figure}[t]
	\centering
	\plotone{flame_speeds_rotation.pdf}
	\caption{\label{fig:flame_speeds_1} Flame front position vs.\ time for the standard ($10^8~\mathrm{K}$) 500 Hz and 1000 Hz runs ({\tt F500} and {\tt F1000}). The dashed lines
		show quadratic least-squares fits to the data for $t \gtrsim 20~\mathrm{ms}$.}
\end{figure}

Figure~\ref{fig:flame_speeds_1} gives position vs.\ time for the $\tt{F500}$ and $\tt{F1000}$ runs 
(blue and orange, respectively) to show the dependence on rotation rate. {\color{blue}In 
\citet{flame_wave1}, we 
used linear least-squares fits to our position vs.\ time data to estimate the propagation velocity. 
For this study we instead fit a quadratic curve to each set of data, as some of the hotter 500~Hz 
runs exhibit significant acceleration. We discard the data points with $t \lesssim 20~\mathrm{ms}$, 
since these correspond to the initial transient period before the flame has begun to propagate 
steadily.} The fit parameters yield estimates of the flame acceleration and velocity at $t = 0$, 
which are provided in Table~\ref{table:flame_speeds}. {\color{blue}Note that the listed velocity is 
only 
a parameter that may be used to calculate the flame speed at an arbitrary time. It is not an 
estimate of the true initial velocity of the flame, since the flame has not achieved ignition yet 
at $t = 0$. We use the fit parameters to calculate the flame speeds at $t = 50~\mathrm{ms}$ (when the flame has reached steady propagation), given in the fourth column of Table \ref{table:flame_speeds}.} \MarginPar{KE: Outside of updating the data in the table, is there anything else we 
want to add to this para?}

We found in \citet{flame_wave1} that the flame speed, $s$, obeys the following relation:
\begin{equation}
	\label{eqn:flame_speed_scaling}
	s \appropto L_R (\enucdot)^{\half} \propto \frac{(\enucdot)^{\half}}{\Omega},
\end{equation}
where $\enucdot$ is the specific nuclear energy generation rate and $L_R$ is the Rossby length 
described by Equation \ref{eqn:rossby}. This finding was consistent with the results of 
\citet{cavecchi:2013}, who noted that the flame speeds in their simulations scaled with the Rossby 
length. As seen in Figure~\ref{fig:flame_speeds_1}, there is no obvious inverse scaling of the 
flame speed with rotation rate in this set of runs. We observed earlier, however, that nuclear 
reactions progress more quickly at higher rotation rate. This results in a higher $\enucdot$ --- up 
to four to six times higher near the burning front after the flame ignites --- which may counteract 
the reduction in flame speed from the increased Coriolis confinement. We also observe that 
$\tt{F500}$ accelerates faster than $\tt{F1000}$, which appears to experience a small deceleration 
at early times. This disparity may be a direct result of the difference in Coriolis force.

% It might be worth plotting peak enucdot as a function of time for each run

\subsection{Effect of Temperature on Flame Structure}\label{ssec:temp_structure}

 \MarginPar{NF: I think we should plot peak enucdot as a function of time for each run, it would be useful for me to reference it in one of the later sections}

To explore the effect of different initial temperature configurations, we run four simulations fixed at a rotation rate of 500 Hz with temperatures as shown in Table~\ref{table:runs}. For all the 500 Hz simulations (with the exception of the coolest run, $\tt{F500}$), we set $T_\mathrm{star}$ = $T_\mathrm{hi}$, scaling $\delta T$ accordingly to maintain a consistent value of $T_\mathrm{hi} + \delta T$. If we let $T_\mathrm{star} < T_\mathrm{hi}$, the cooler neutron star surface might act as a heat sink and siphon away energy that would otherwise go into heating the burning layer. By choosing $T_\mathrm{star}$ = $T_\mathrm{hi}$, we can effectively explore simulations with greater surface heating. There are several physically distinct mechanisms that could produce an increased temperature at the crust: crustal heating, some other form of shallow heating or a higher accretion rate. In these simulations, we do not model the mechanism producing the heating effect, just the effect itself, so we do not distinguish between which of these mechanisms cause the heating.

\begin{figure}[t]
\centering
\plotone{compare_500Hz_abar}
\caption{\label{fig:compare_500Hz_abar} Comparison of $\bar{A}$ for 3 different 500~Hz models with different neutron star temperatures, $T_\mathrm{star}$, and resulting envelope structures.  Each flame is shown at 70~ms.} 
\end{figure}

Figure~\ref{fig:compare_500Hz_abar} shows $\bar{A}$ for three 500 Hz simulations with different initial temperature structures at $t = 70~\mathrm{ms}$. The hottest of the three models, $\tt{F500\_3E8}$, has a faster propagating flame (this will be discussed further in Section \ref{ssec:temp_prop}).  

Similar to the $\bar{A}$ trends found for different rotation rates in Figure \ref{fig:abar}, the $\bar{A}$-$T$ phase plot in Figure~\ref{fig:abar_hot} suggests that the actual reactions happening in the hottest run may be different than for the cooler runs. Not only is the $\tt{F500\_3E8}$ flame itself significantly hotter overall, but there is a large causally connected region across a wide range of $\bar{A}$. This indicates that the hotter temperature has facilitated significant burning in reaction pathways that were not favored at the lower temperatures. The reaction pathways are also very disrupted for $\tt{F500\_3E8}$ compared to the cooler runs, suggesting that the hotter temperature leads to more turbulent mixing. Compared to the highest rotation rate run, $\tt{F1000}$, $\tt{F500\_3E8}$ reaches a peak temperature roughly $1 \times 10^9~\mathrm{K}$ higher ($\sim 2.7 \times 10^9~\mathrm{K}$ versus $\sim 1.7 \times 10^9~\mathrm{K}$). Though increasing the rotation rate does seem to concentrate the burning and increase the peak temperature, it does not reach the same level as directly increasing the temperature at the crust.

In contrast to the models with $T_{\mathrm{star}} \leq 3 \times 10^8~\mathrm{K}$, $\tt{F500\_4E8}$ is so hot that the organized flame structure is lost. The fuel layer undergoes stable burning and essentially simmers in place, rather than forming a flame. This process has been observed in neutron stars with accretion rates that exceed the Eddington limit \citep{fujimoto1981,bildsten1998thermonuclear,keek2009effect}; in this model with have essentially recreated the conditions found in such stars with high accretion rates. This model burns so strongly that it is only run out to $40$ ms. After an initial period of burning moves across the domain, residual burning continues and eventually ignites the entire fuel layer at late times, as shown in Figures \ref{fig:4e8_stacked_enuc} and \ref{fig:4e8_stacked_abar} for three snapshots taken in the last ten seconds of the simulation. In Figure~\ref{fig:4e8_stacked_enuc}, $\enucdot$ reaches values of $10^{18} - 10^{20}~\mathrm{erg}\,\mathrm{g}^{-1} \mathrm{s}^{-1}$ across the domain and at heights up to $\sim 0.5\times 10^4$ cm. There is still significant burning occurring even higher, with $\enucdot$ reaching $\sim 10^{15}~\mathrm{erg}\,\mathrm{g}^{-1} \mathrm{s}^{-1}$ at heights up to $\sim 0.8\times 10^4$ cm. For comparison, the next hottest run ($\tt{F500\_3E8}$) only reaches maximum $\enucdot$ values of $10^{18}~\mathrm{erg}\,\mathrm{g}^{-1} \mathrm{s}^{-1}$ and a maximum height of $\sim 0.5\times 10^4$ cm, even at the latest timesteps ($\sim 70~\mathrm{ms}$, when the flame is most developed). Significant $\enucdot$ values for $\tt{F500\_3E8}$ are confined to the flame front, rather than spanning the entire domain.

\MarginPar{NF: should I include a plot for this or is it ok to just say it?}
Figure~\ref{fig:4e8_stacked_abar} shows $\bar{A}$ for $\tt{F500\_4E8}$. Again, burning extends across the domain and high into the atmosphere and fuel layer, lacking the characteristic flame structure shown in $\bar{A}$ plots for the lower temperature $500$ Hz runs (see Figure~\ref{fig:compare_500Hz_abar}). A distinct mass of material appears to have broken out of the atmosphere near the axis of symmetry. The atmosphere edge is located at $\sim 1.2\times 10^4~\mathrm{cm}$. {\color{blue}A similar effect is also visible (to a lesser extent) in the $\tt{F500\_3E8}$ plot in Figure \ref{fig:compare_500Hz_abar}, with a faint haze of material rising above the flame. This effect, however, may be numerical rather than physical, resulting from the boundary conditions imposed at the axis of symmetry for these simulations.} $\tt{F500\_4E8}$ clearly reaches much higher $\bar{A}$ values across the domain (especially at the latest snapshot, $t = 40~\mathrm{ms}$) compared to all the other runs described in this paper (see Figures \ref{fig:compare_500Hz_abar}, \ref{fig:time_series_500}, and \ref{fig:time_series_1000}). This is not surprising, since as we have shown in our simulations with differing initial temperature structures, the burning is very temperature-sensitive. 


\begin{figure}[t]
	\centering
	\plotone{compare_500Hz_enuc_4e8.pdf}
	\caption{\label{fig:4e8_stacked_enuc} Time series of the 500~Hz run $\tt{F500\_4E8}$, with $T_{\mathrm{star}} = 4 \times 10^8~\mathrm{K}$, showing $\dot{e}_{nuc}$. This model burns so strongly that it is only run out to 40 ms; the snapshots shown here are at T = 30 ms, 35 ms, and 40 ms.}
\end{figure}

\begin{figure}[t]
	\centering
	\plotone{compare_500Hz_abar_4e8.pdf}
	\caption{\label{fig:4e8_stacked_abar} Time series of the 500~Hz run $\tt{F500\_4E8}$, with $T_{\mathrm{star}} = 4 \times 10^8~\mathrm{K}$, showing $\bar{A}$.}
\end{figure}


\subsection{Effect of Temperature on Flame Propagation}\label{ssec:temp_prop}

The method for measuring flame propagation described in Section \ref{ssec:rot_propagation} is applied in Figure \ref{fig:flame_speeds_2} to the three 500 Hz runs with $T_{\mathrm{star}} \leq 3 \times 10^8~\mathrm{K}$. Due to the lack of a clear burning front in $\tt{F500\_4E8}$, we do not analyze its propagation speed and acceleration. As the initial $T_{\mathrm{star}}$ is increased, the flame becomes greatly accelerated. The initial flame velocities and accelerations derived from a quadratic least-squares fit to each of the datasets, {\color{blue} as well as the flame velocities at $t = 50~\mathrm{ms}$ calculated using the fit parameters,} are provided in Table~\ref{table:flame_speeds}. {\color{blue} Comparing the flame propagation at different initial temperatures, the most robust feature is the acceleration of the $\tt{F500\_3E8}$ run at $t \sim 40~\mathrm{ms}$.} The reason for the acceleration of the flame is not entirely clear. It could be that energy released from burning begins to dominate the flame's propagation as it evolves, increasing the flame speed over time. Another possibility is that the increased temperatures lead to enhanced turbulent mixing effects that pull in more fresh fuel for the flame to burn. Yet another possibility is that the higher initial $T_{\mathrm{star}}$ leads to a greater average temperature in the fuel layer over time, making it easier for the flame to burn the fuel and propagate.


\begin{figure}[t]
	\centering
	\plotone{500Hz_stacked_enuc1_2order.pdf}
	\caption{\label{fig:flame_speeds_2} Flame front position vs.\ time for the three 500 Hz runs with
		$T_{\mathrm{star}} \leq 3 \times 10^8~\mathrm{K}$. The dashed lines show quadratic least-squares fits to the data
		for $t \gtrsim 20~\mathrm{ms}$. Note that due to its rapid acceleration, $\tt{F500\_3E8}$ 
		(green) is only run out to $\sim 70~\mathrm{ms}$ to avoid surpassing the domain boundary.}
\end{figure}

The $u$-$\rho$ phase plot in Figure~\ref{fig:urho_hot} shows that the horizontal velocity for the $T_{\rm star} = 3\times10^8$~K is sufficiently  \MarginPar{should this say "significantly"?} greater in magnitude than for the two cooler runs. The $u$-$v$ plot in Figure~\ref{fig:uv_hot} further demonstrates this, as again we see that the magnitude of the velocity in the vertical direction of the flame for the hottest run is also significantly larger. This motion therefore most likely corresponds to the vortical motion rather than to the propagation of the flame itself, something which is further supported by the fact that the velocity magnitude is 2-3 orders of magnitude greater than the flame speeds listed in Table~\ref{table:flame_speeds}. Also of interest in these plots is that the coolest run does not seem to have developed the characteristic vortex structure at the burning front (i.e. where $\dot{e}_{\rm nuc}$ is greatest) that can be clearly seen for the hotter runs at this time (40~ms). As can be seen in Figure~\ref{fig:uv} (which was plotted at 100~ms), this does develop more at later times. Similar to what saw when comparing the runs with different rotation rates, it would therefore appear that it takes longer for the flame to develop when $T_{\rm star}$ is cooler. 

\begin{figure}[t]
    \centering
    \plotone{urho_plot_hotter}
    \caption{\label{fig:urho_hot}Phase plot of the horizontal $x$-velocity $u$, density $\rho$ and the nuclear energy generation rate $\dot{e}_{\rm nuc}$ for the 500~Hz runs with $T_{\rm star}$ of $10^8$~K, $2\times10^8$~K and $3\times10^8$~K at $t = 40$~ms.}
\end{figure}

\begin{figure}[t]
    \centering
    \plotone{uv_plot_hotter}
    \caption{\label{fig:uv_hot}Phase plot of the horizontal $x$-velocity $u$, vertical $y$-velocity $v$ and the nuclear energy generation rate $\dot{e}_{\rm nuc}$ for the 500~Hz runs with $T_{\rm star}$ of $10^8$~K, $2\times10^8$~K and $3\times10^8$~K at $t = 40$~ms.}
\end{figure}

\begin{figure}[t]
    \centering
    \plotone{abar_plot_hotter}
    \caption{\label{fig:abar_hot}Phase plot of the mean molecular weight $\bar{A}$, temperature $T$ and the nuclear energy generation rate $\dot{e}_{\rm nuc}$ for the 500~Hz runs with $T_{\rm star}$ of $10^8$~K, $2\times10^8$~K and $3\times10^8$~K at $t = 40$~ms.}
\end{figure}

%======================================================================
% Conclusions
%======================================================================
\section{Discussion and Conclusions}\label{Sec:conclusions}

We ran a number of simulations of laterally propagating flames in XRBs in order to explore the effects and rotation and thermal structure. We found that increasing the rotation rate increased the energy generation rate within the flame, enhancing its propagating \MarginPar{I thought the increased rotation rate enhanced nuclear burning, but suppressed flame propagation?}. This is likely due to greater confinement of the flame from the increased Coriolis force. We explored several models with different crustal temperatures, to determine what effects mechanisms such as higher accretion rates, crustal heating and shallow heating may have on flame propagation. We found that increasing the temperature of the crust significantly enhanced the flame propagation. This we believe to be because a cooler crust allows heat to more efficiently be transferred from the flame itself, therefore reducing the flame's temperature, slowing burning and consequently the reducing its propagation speed. We saw that if the crust temperature was too high, then instead of a flame, the atmosphere would burn stably. This is reminiscent of what is seen for neutron stars with accretion rates exceeding the Eddington luminosity.

In future simulations, we would like to improve and expand our
simulations in order to better understand the processes at play and to
include more physics. This includes adding tracer particles to the
simulations so we can monitor the fluid motion and perform more
detailed nucleosynthesis; extending our simulations to 3D, which would
hopefully alleviate some of the boundary effects we have observed in
these simulations but will require significant computational resource;
and exploring the resolution of our simulations more so that we can
ensure that we have resolved all the necessary physical processes. We
would also like to model H/He flames, as these are the sites of
rp-process nucleosynthesis~\citep{rpprocess}.  Initially we will use the same reaction
sequence explored in our previous convection
calculations~\citep{xrb2}.   Finally, we
recently added an MHD solver to \castro~\citep{sazo2020thesis}; this
will allow us to explore the effects of magnetic fields on flame
propagation in XRBs.

% tl;dr: faster rotation and/or hotter surface temperature = more burning.

% Next steps: tracer particles, 3-d, explore resolution more, H/He flames, MHD flames


\acknowledgments \castro\ is open-source and freely available at
\url{http://github.com/AMReX-Astro/Castro}.  The problem setup used
here is available in the git repo as {\tt
  flame\_wave}.  The work at Stony Brook was supported by DOE/Office
of Nuclear Physics grant DE-FG02-87ER40317.  This material is based upon work supported by the
U.S. Department of Energy, Office of Science, Office of Advanced
Scientific Computing Research and Office of Nuclear Physics, Scientific
Discovery through Advanced Computing (SciDAC) program under Award
Number DE-SC0017955.  This material is also based upon work supported by the U.S. Department
of Energy, Office of Science, Office of Advanced Scientific Computing Research, Department of
Energy Computational Science Graduate Fellowship under Award Number DE-SC0021110. This work was supported in part by the U.S. Department of Energy, Office of Science, Office of Workforce Development for Teachers and Scientists (WDTS) under the Science Undergraduate Laboratory Internship (SULI) program. MZ acknowledges support from the Simons Foundation. 
This research used resources of the National Energy
Research Scientific Computing Center, a DOE Office of Science User
Facility supported by the Office of Science of the U.~S.\ Department
of Energy under Contract No.\ DE-AC02-05CH11231.  This research used
resources of the Oak Ridge Leadership Computing Facility at the Oak
Ridge National Laboratory, which is supported by the Office of Science
of the U.S. Department of Energy under Contract No. DE-AC05-00OR22725,
awarded through the DOE INCITE program.  We thank NVIDIA Corporation
for the donation of a Titan X and Titan V GPU through their academic
grant program.  This research has made use of NASA's Astrophysics Data
System Bibliographic Services.

\facilities{NERSC, OLCF}

\software{AMReX \citep{amrex_joss},
          Castro \citep{castro,castro_joss},
          GCC (\url{https://gcc.gnu.org/}),
          Jupyter \citep{Kluyver2016},
          linux (\url{https://www.kernel.org/}),
          matplotlib (\citealt{Hunter:2007}, \url{http://matplotlib.org/}),
          NumPy \citep{numpy,numpy2},
          python (\url{https://www.python.org/}),
          valgrind \citep{valgrind},
          VODE \citep{vode},
          yt \citep{yt}}



%======================================================================
% References
%======================================================================

\bibliographystyle{aasjournal}
\bibliography{ws}


\end{document}
