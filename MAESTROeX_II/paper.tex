%% Beginning of file 'sample63.tex'
%%
%% Modified 2019 June
%%
%% This is a sample manuscript marked up using the
%% AASTeX v6.3 LaTeX 2e macros.
%%
%% AASTeX is now based on Alexey Vikhlinin's emulateapj.cls 
%% (Copyright 2000-2015).  See the classfile for details.

%% AASTeX requires revtex4-1.cls (http://publish.aps.org/revtex4/) and
%% other external packages (latexsym, graphicx, amssymb, longtable, and epsf).
%% All of these external packages should already be present in the modern TeX 
%% distributions.  If not they can also be obtained at www.ctan.org.

%% The first piece of markup in an AASTeX v6.x document is the \documentclass
%% command. LaTeX will ignore any data that comes before this command. The 
%% documentclass can take an optional argument to modify the output style.
%% The command below calls the preprint style which will produce a tightly 
%% typeset, one-column, single-spaced document.  It is the default and thus
%% does not need to be explicitly stated.
%%
%%
%% using aastex version 6.3
%\documentclass[trackchanges]{aastex63}
\documentclass{aastex63}

%% The default is a single spaced, 10 point font, single spaced article.
%% There are 5 other style options available via an optional argument. They
%% can be invoked like this:
%%
%% \documentclass[arguments]{aastex63}
%% 
%% where the layout options are:
%%
%%  twocolumn   : two text columns, 10 point font, single spaced article.
%%                This is the most compact and represent the final published
%%                derived PDF copy of the accepted manuscript from the publisher
%%  manuscript  : one text column, 12 point font, double spaced article.
%%  preprint    : one text column, 12 point font, single spaced article.  
%%  preprint2   : two text columns, 12 point font, single spaced article.
%%  modern      : a stylish, single text column, 12 point font, article with
%% 		  wider left and right margins. This uses the Daniel
%% 		  Foreman-Mackey and David Hogg design.
%%  RNAAS       : Preferred style for Research Notes which are by design 
%%                lacking an abstract and brief. DO NOT use \begin{abstract}
%%                and \end{abstract} with this style.
%%
%% Note that you can submit to the AAS Journals in any of these 6 styles.
%%
%% There are other optional arguments one can invoke to allow other stylistic
%% actions. The available options are:
%%
%%   astrosymb    : Loads Astrosymb font and define \astrocommands. 
%%   tighten      : Makes baselineskip slightly smaller, only works with 
%%                  the twocolumn substyle.
%%   times        : uses times font instead of the default
%%   linenumbers  : turn on lineno package.
%%   trackchanges : required to see the revision mark up and print its output
%%   longauthor   : Do not use the more compressed footnote style (default) for 
%%                  the author/collaboration/affiliations. Instead print all
%%                  affiliation information after each name. Creates a much 
%%                  longer author list but may be desirable for short 
%%                  author papers.
%% twocolappendix : make 2 column appendix.
%%   anonymous    : Do not show the authors, affiliations and acknowledgments 
%%                  for dual anonymous review.
%%
%% these can be used in any combination, e.g.
%%
%% \documentclass[twocolumn,linenumbers,trackchanges]{aastex63}
%%
%% AASTeX v6.* now includes \hyperref support. While we have built in specific
%% defaults into the classfile you can manually override them with the
%% \hypersetup command. For example,
%%
%% \hypersetup{linkcolor=red,citecolor=green,filecolor=cyan,urlcolor=magenta}
%%
%% will change the color of the internal links to red, the links to the
%% bibliography to green, the file links to cyan, and the external links to
%% magenta. Additional information on \hyperref options can be found here:
%% https://www.tug.org/applications/hyperref/manual.html#x1-40003
%%
%% Note that in v6.3 "bookmarks" has been changed to "true" in hyperref
%% to improve the accessibility of the compiled pdf file.
%%
%% If you want to create your own macros, you can do so
%% using \newcommand. Your macros should appear before
%% the \begin{document} command.
%%

\newcommand{\vdag}{(v)^\dagger}
\newcommand\aastex{AAS\TeX}
\newcommand\latex{La\TeX}

% for non-stacked fractions
\newcommand{\sfrac}[2]{\mathchoice
  {\kern0em\raise.5ex\hbox{\the\scriptfont0 #1}\kern-.15em/
   \kern-.15em\lower.25ex\hbox{\the\scriptfont0 #2}}
  {\kern0em\raise.5ex\hbox{\the\scriptfont0 #1}\kern-.15em/
   \kern-.15em\lower.25ex\hbox{\the\scriptfont0 #2}}
  {\kern0em\raise.5ex\hbox{\the\scriptscriptfont0 #1}\kern-.2em/
   \kern-.15em\lower.25ex\hbox{\the\scriptscriptfont0 #2}}
  {#1\!/#2}}

\newcommand{\myhalf}{\sfrac{1}{2}}
\newcommand{\thalf}{\sfrac{3}{2}}

\newcommand{\eb}{{\bf{e}}}
\newcommand{\Ub}{{\bf{U}}}
\newcommand{\Ubt}{\widetilde{\Ub}}
\newcommand{\Vb}{{\bf{V}}}
\newcommand{\xb}{{\bf{x}}}

\newcommand{\dr}{\Delta r}
\newcommand{\dt}{\Delta t}

\newcommand{\etarho}{\eta_\rho}
\newcommand{\gammaonebar}{\overline{\Gamma}_1}
\newcommand{\Hnuc}{H_{\rm nuc}}
\newcommand{\omegadot}{\dot\omega}
\newcommand{\pred}{{\rm pred}}
\newcommand{\Sbar}{\overline{S}}

\newcommand{\inp}{\mathrm{in}}
\newcommand{\outp}{\mathrm{out}}
\newcommand{\nph}{{n+\myhalf}}
\newcommand{\nmh}{{n-\myhalf}}
\newcommand{\ow}{\overline{w_0}}
\newcommand{\dw}{\delta w_0}
\newcommand{\uadvone}{\Ub^{\mathrm{ADV},\star}}
\newcommand{\uadvonedag}{\Ub^{\mathrm{ADV},\dagger,\star}}
\newcommand{\uadvtwo}{\Ub^{\mathrm{ADV}}}
\newcommand{\uadvtwodag}{\Ub^{\mathrm{ADV},\dagger}}
\newcommand{\gcc}{\mathrm{g~cm^{-3} }}
\newcommand{\kth}{{k_{\mathrm{th}}}}

% for the red MarginPars
\usepackage{color}
% make the MarginPars look pretty
\setlength{\marginparwidth}{0.5in}
\newcommand{\MarginPar}[1]{\marginpar{\vskip-\baselineskip\raggedright\tiny\sffamily
\hrule\smallskip{\color{red}#1}\par\smallskip\hrule}}

\usepackage{url}

\usepackage{mathtools}

\usepackage{multirow}

%% Reintroduced the \received and \accepted commands from AASTeX v5.2
\received{XXX X, XXXX}
\revised{XXX X, XXXX}
\accepted{XXX X, XXXX}
%% Command to document which AAS Journal the manuscript was submitted to.
%% Adds "Submitted to " the argument.
\submitjournal{ApJ}

%% For manuscript that include authors in collaborations, AASTeX v6.3
%% builds on the \collaboration command to allow greater freedom to 
%% keep the traditional author+affiliation information but only show
%% subsets. The \collaboration command now must appear AFTER the group
%% of authors in the collaboration and it takes TWO arguments. The last
%% is still the collaboration identifier. The text given in this
%% argument is what will be shown in the manuscript. The first argument
%% is the number of author above the \collaboration command to show with
%% the collaboration text. If there are authors that are not part of any
%% collaboration the \nocollaboration command is used. This command takes
%% one argument which is also the number of authors above to show. A
%% dashed line is shown to indicate no collaboration. This example manuscript
%% shows how these commands work to display specific set of authors 
%% on the front page.
%%
%% For manuscript without any need to use \collaboration the 
%% \AuthorCollaborationLimit command from v6.2 can still be used to 
%% show a subset of authors.
%
%\AuthorCollaborationLimit=2
%
%% will only show Schwarz & Muench on the front page of the manuscript
%% (assuming the \collaboration and \nocollaboration commands are
%% commented out).
%%
%% Note that all of the author will be shown in the published article.
%% This feature is meant to be used prior to acceptance to make the
%% front end of a long author article more manageable. Please do not use
%% this functionality for manuscripts with less than 20 authors. Conversely,
%% please do use this when the number of authors exceeds 40.
%%
%% Use \allauthors at the manuscript end to show the full author list.
%% This command should only be used with \AuthorCollaborationLimit is used.

%% The following command can be used to set the latex table counters.  It
%% is needed in this document because it uses a mix of latex tabular and
%% AASTeX deluxetables.  In general it should not be needed.
%\setcounter{table}{1}

%%%%%%%%%%%%%%%%%%%%%%%%%%%%%%%%%%%%%%%%%%%%%%%%%%%%%%%%%%%%%%%%%%%%%%%%%%%%%%%%
%%
%% The following section outlines numerous optional output that
%% can be displayed in the front matter or as running meta-data.
%%
%% If you wish, you may supply running head information, although
%% this information may be modified by the editorial offices.
\shorttitle{SDC Scheme for Low Mach Number Flows}
\shortauthors{Fan et al.}
%%
%% You can add a light gray and diagonal water-mark to the first page 
%% with this command:
%% \watermark{text}
%% where "text", e.g. DRAFT, is the text to appear.  If the text is 
%% long you can control the water-mark size with:
%% \setwatermarkfontsize{dimension}
%% where dimension is any recognized LaTeX dimension, e.g. pt, in, etc.
%%
%%%%%%%%%%%%%%%%%%%%%%%%%%%%%%%%%%%%%%%%%%%%%%%%%%%%%%%%%%%%%%%%%%%%%%%%%%%%%%%%

%% This is the end of the preamble.  Indicate the beginning of the
%% manuscript itself with \begin{document}.

\begin{document}

\title{Spectral Deferred Corrections Scheme for Low Mach Number Flows}

%% LaTeX will automatically break titles if they run longer than
%% one line. However, you may use \\ to force a line break if
%% you desire. In v6.3 you can include a footnote in the title.

%% A significant change from earlier AASTEX versions is in the structure for 
%% calling author and affiliations. The change was necessary to implement 
%% auto-indexing of affiliations which prior was a manual process that could 
%% easily be tedious in large author manuscripts.
%%
%% The \author command is the same as before except it now takes an optional
%% argument which is the 16 digit ORCID. The syntax is:
%% \author[xxxx-xxxx-xxxx-xxxx]{Author Name}
%%
%% This will hyperlink the author name to the author's ORCID page. Note that
%% during compilation, LaTeX will do some limited checking of the format of
%% the ID to make sure it is valid. If the "orcid-ID.png" image file is 
%% present or in the LaTeX pathway, the OrcID icon will appear next to
%% the authors name.
%%
%% Use \affiliation for affiliation information. The old \affil is now aliased
%% to \affiliation. AASTeX v6.3 will automatically index these in the header.
%% When a duplicate is found its index will be the same as its previous entry.
%%
%% Note that \altaffilmark and \altaffiltext have been removed and thus 
%% can not be used to document secondary affiliations. If they are used latex
%% will issue a specific error message and quit. Please use multiple 
%% \affiliation calls for to document more than one affiliation.
%%
%% The new \altaffiliation can be used to indicate some secondary information
%% such as fellowships. This command produces a non-numeric footnote that is
%% set away from the numeric \affiliation footnotes.  NOTE that if an
%% \altaffiliation command is used it must come BEFORE the \affiliation call,
%% right after the \author command, in order to place the footnotes in
%% the proper location.
%%
%% Use \email to set provide email addresses. Each \email will appear on its
%% own line so you can put multiple email address in one \email call. A new
%% \correspondingauthor command is available in V6.3 to identify the
%% corresponding author of the manuscript. It is the author's responsibility
%% to make sure this name is also in the author list.
%%
%% While authors can be grouped inside the same \author and \affiliation
%% commands it is better to have a single author for each. This allows for
%% one to exploit all the new benefits and should make book-keeping easier.
%%
%% If done correctly the peer review system will be able to
%% automatically put the author and affiliation information from the manuscript
%% and save the corresponding author the trouble of entering it by hand.

\correspondingauthor{Duoming Fan}
\email{DFan@lbl.gov}

\author[0000-0002-3246-4315]{Duoming Fan}
\affiliation{Lawrence Berkeley National Laboratory \\
Center for Computational Sciences and Engineering \\
One Cyclotron Road, MS 50A-3111 \\
Berkeley, CA 94720, USA}

\author[0000-0003-1791-0265]{Andrew Nonaka}
\affiliation{Lawrence Berkeley National Laboratory \\
Center for Computational Sciences and Engineering \\
One Cyclotron Road, MS 50A-3111 \\
Berkeley, CA 94720, USA}

\author[0000-0003-2103-312X]{Ann S. Almgren}
\affiliation{Lawrence Berkeley National Laboratory \\
Center for Computational Sciences and Engineering \\
One Cyclotron Road, MS 50A-3111 \\
Berkeley, CA 94720, USA}

\author[0000-0002-1530-781X]{Alice Harpole}
\affiliation{Stony Brook University \\
Department of Physics and Astronomy \\
Stony Brook, NY 11794-3800, USA}

\author[0000-0001-8401-030X]{Michael Zingale}
\affiliation{Stony Brook University \\
Department of Physics and Astronomy \\
Stony Brook, NY 11794-3800, USA}

%% Note that the \and command from previous versions of AASTeX is now
%% depreciated in this version as it is no longer necessary. AASTeX 
%% automatically takes care of all commas and "and"s between authors names.

%% AASTeX 6.3 has the new \collaboration and \nocollaboration commands to
%% provide the collaboration status of a group of authors. These commands 
%% can be used either before or after the list of corresponding authors. The
%% argument for \collaboration is the collaboration identifier. Authors are
%% encouraged to surround collaboration identifiers with ()s. The 
%% \nocollaboration command takes no argument and exists to indicate that
%% the nearby authors are not part of surrounding collaborations.

%% Mark off the abstract in the ``abstract'' environment. 

\begin{abstract}
  We describe a spectral deferred corrections scheme for MAESTROeX, an adaptive
  low Mach number hydrodynamics code that allows for efficient, long-time integration. 
\end{abstract}

%% Keywords should appear after the \end{abstract} command. 
%% See the online documentation for the full list of available subject
%% keywords and the rules for their use.
\keywords{Stellar convective zones, Hydrodynamics, Computational methods, 
Nuclear astrophysics, Nucleosynthesis, Nuclear abundances, Type Ia supernovae}

%% From the front matter, we move on to the body of the paper.
%% Sections are demarcated by \section and \subsection, respectively.
%% Observe the use of the LaTeX \label
%% command after the \subsection to give a symbolic KEY to the
%% subsection for cross-referencing in a \ref command.
%% You can use LaTeX's \ref and \label commands to keep track of
%% cross-references to sections, equations, tables, and figures.
%% That way, if you change the order of any elements, LaTeX will
%% automatically renumber them.
%%
%% We recommend that authors also use the natbib \citep
%% and \citet commands to identify citations.  The citations are
%% tied to the reference list via symbolic KEYs. The KEY corresponds
%% to the KEY in the \bibitem in the reference list below. 

\section{Introduction} \label{sec:intro}


\section{Numerical Considerations and Algorithm} \label{sec:equations}

MAESTROeX integrates the following system of equations, 
%%
\begin{eqnarray}
\frac{\partial\Ub}{\partial t} &=& -\Ub\cdot\nabla\Ub  - \frac{\beta_0}{\rho}\nabla\left(\frac{\pi}{\beta_0}\right) - \frac{\rho-\rho_0}{\rho} g\eb_r,\label{eq:momentum}\\
\frac{\partial(\rho X_k)}{\partial t} &=& -\nabla\cdot(\rho X_k\Ub) + \rho\omegadot_k,\label{eq:species}\\
\frac{\partial(\rho h)}{\partial t} &=& -\nabla\cdot(\rho h\Ub) + \frac{Dp_0}{Dt} + \rho\Hnuc + \nabla\cdot\kth\nabla T,\label{eq:enthalpy}
\end{eqnarray}
%%
along with a velocity constraint equation,
%
\begin{equation}
\nabla\cdot(\beta_0\Ub) = \beta_0\left(S - \frac{1}{\gammaonebar p_0}\frac{\partial p_0}{\partial t}\right), \label{eq:U divergence}
\end{equation}
%
as described in previous papers \citep{MAESTRO_V,MAESTRO_VI}. Note that we have included thermal diffusion
in the governing equations with the term 
\begin{equation}
\nabla\cdot\kth\nabla T = \nabla\cdot\frac{\kth}{c_p}\nabla h - \sum_k\nabla\cdot\frac{\xi_k k_{\rm th}}{c_p}\nabla X_k - \nabla\cdot\frac{h_p k_{\rm th}}{c_p}\nabla p_0 .
\end{equation}
The species are also constrained such that $\sum_k X_k = 1$ giving $\rho = \sum_k (\rho X_k)$ and
%
\begin{equation}
\frac{\partial\rho}{\partial t} = -\nabla\cdot(\rho\Ub).
\end{equation}
%

In past papers, MAESTROeX advance the thermodynamic variables by coupling advection, diffusion,
and reactions using Strang splitting.  Specifically, we integrate the reaction terms 
over half a time step ignoring contributions from advection and diffusion,
then from this intermediate state we integrate the advection and diffusion terms over
a full time step (while ignoring reactions), and finally integrate the reactions over
a half time step (ignoring advection and diffusion). For problems where
the reactions and/or diffusion greatly alter the energy balance or composition 
as compared to advection, this operator splitting approach can lead to large 
splitting errors and highly inaccurate solutions.
This issue is can be particularly exasperating for low Mach number methods that can
theoretically take large advection-based time steps.

An alternate approach to advancing the thermodynamic variables is to use
spectral deferred corrections (SDC). SDC is an iterative scheme used to couple the various processes
together, with each process seeing an  approximation of the other processes as a source term.
The SDC algorithm converges to an integral representation of the solution in
time that couples all of the processes together in a self-consistent fashion.
It has been shown to be more accurate and efficient than Strang splitting in a
terrestrial, non-stratified, low Mach number reacting flow solver \cite{nonaka2012sdc},
so we believe an SDC version of MAESTROeX would greatly improve the efficiency of the code.
\\

\emph{As a first attempt, we will work on coupling advection and reactions only
via SDC, with a base state that is fixed in time, but not space.}
\\

The time-advancement is divided into three major steps.  The first step is the predictor, where we integrate the thermodynamic variables, $(\rho,\rho X_k,\rho h)$, over the full time step.  The second step is corrector, where we use the results from the predictor to perform a more accurate temporal integration of the thermodynamic variables.  The third step is the velocity and dynamic pressure update.\\

{\bf Step 1:} ({\it Compute advection velocities})\\ \\
Use $\Ub^n$ and a second-order Godunov method to compute time-centered edge velocities, $\uadvonedag$, with time-lagged dynamic pressure and explicit buoyancy as forcing terms.  The $\dagger$ superscript indicates that this field does not satisfy the divergence constraint.  Compute $S^{n+\myhalf,\star}$ by extrapolating in time,
\begin{equation}
S^{n+\myhalf,\star} = S^n + \frac{\Delta t^n}{2}\frac{S^n - S^{n-1}}{\Delta t^{n-1}},
\end{equation}
and project $\uadvonedag$ to obtain $\uadvone$, which satisfies
\begin{equation}
\nabla\cdot\left(\beta_0^n\uadvone\right) = S^{n+\myhalf,\star}.
\end{equation}

{\bf Step 2:} ({\it Predictor})\\ \\
 In this step, we integrate $(\rho, \rho X_k, \rho h)$ over the full time step.  The quantities $(S, \beta_0, k_{\rm th}, c_p, \xi_k, h_p)^n$ are computed from the the thermodynamic variables at $t^n$.  This step is divided into several sub-steps:\\

{\bf Step 2A:} ({\it Compute advective flux divergences})\\ \\
Use $\uadvone$ and a second-order Godunov integrator to compute time-centered edge states, $(\rho X_k, \rho h)^{n+\myhalf,(0)}$, with time-lagged reactions ($I^{\rm lagged} = I^{(j_{\rm max})}$ from the previous time step), explicit diffusion, and time-centered thermodynamic pressure as source terms.  Define the advective flux divergences as
\begin{eqnarray}
A_{\rho X_k}^{(0)} &=& -\nabla\cdot\left[\left(\rho X_k\right)^{n+\myhalf,{(0)}}\uadvone\right],\\
A_{\rho h}^{(0)} &=& -\nabla\cdot\left[\left(\rho h\right)^{n+\myhalf,(0)}\uadvone\right] + \uadvone\cdot\nabla p_0.
\end{eqnarray}
Next, use these fluxes to compute the time-advanced density,
\begin{equation}
\frac{\rho^{n+1} - \rho^n}{\Delta t} = \sum_k A_{\rho X_k}^{(0)}.
\end{equation}
Then, compute preliminary, time-advanced species using
\begin{equation}
\frac{\rho^{n+1}\widehat{X}_k^{n+1,(0)} - (\rho X_k)^n}{\Delta t} = A_{\rho X_k}^{(0)} + I_{\rho X_k}^{\rm lagged}.\label{eq:sdc species 2}
\end{equation}

{\bf Step 2B:} ({\it Compute diffusive flux divergence})\\ \\
Solve a Crank-Nicolson-type diffusion equation for $\widehat{h}^{n+1,(0)}$, using transport coefficients evaluated at $t^n$ everywhere,
\begin{eqnarray}
\frac{\rho^{n+1}\widehat{h}^{n+1,(0)} - (\rho h)^n}{\Delta t} &=& A_{\rho h}^{(0)} + I_{\rho h}^{\rm lagged}\nonumber\\
&& + \frac{1}{2}\left(\nabla\cdot\frac{\kth^n}{c_p^n}\nabla h^n + \nabla\cdot\frac{\kth^n}{c_p^n}\nabla \widehat{h}^{n+1,(0)}\right)\nonumber\\
&& - \frac{1}{2}\left(\sum_k\nabla\cdot\frac{\xi_k^n k_{\rm th}^n}{c_p^n}\nabla X_k^n + \sum_k\nabla\cdot\frac{\xi_k^n k_{\rm th}^n}{c_p^n}\nabla\widehat{X}_k^{n+1,(0)}\right)\nonumber\\
&& - \frac{1}{2}\left(\nabla\cdot\frac{h_p^n k_{\rm th}^n}{c_p^n}\nabla p_0 + \nabla\cdot\frac{h_p^n k_{\rm th}^n}{c_p^n}\nabla p_0\right),\label{eq:sdc enthalpy 2}
\end{eqnarray}

{\bf Step 2C:} ({\it Advance thermodynamic variables})\\ \\
Define $Q_{\rho X_k}^{(0)}$ as the right hand side of (\ref{eq:sdc species 2}) without the $I_{\rho X_k}^{\rm lagged}$ term, and define $Q_{\rho h}^{(0)}$ as the right hand side of (\ref{eq:sdc enthalpy 2}) without the $I_{\rho h}^{\rm lagged}$ term.  Use {\tt VODE} to integrate (\ref{eq:species}) and (\ref{eq:enthalpy}) over $\Delta t$ to advance $(\rho X_k, \rho h)^n$ to $(\rho X_k, \rho h)^{n+1,(0)}$ using the piecewise-constant advection and diffusion source terms:
\begin{eqnarray}
\frac{\partial(\rho X_k)}{\partial t} &=& Q_{\rho X_k}^{(0)} + \rho\dot\omega_k\\
\frac{\partial(\rho h)}{\partial t} &=& Q_{\rho h}^{(0)} + \rho\Hnuc.
\end{eqnarray}
At this point we can define $I_{\rho X_k}^{(0)}$ and $I_{\rho h}^{(0)}$, or whatever term we need depending on our species and enthalpy edge state prediction types, for use in the corrector step.  In our first implementation, we are predicting $\rho X_k$ and $\rho h$, in which case we define:
\begin{eqnarray}
I_{\rho X_k}^{(0)} &=& \frac{(\rho X_k)^{n+1,(0)} - (\rho X_k)^n}{\Delta t} - Q_{\rho X_k}^{(0)}\\
I_{\rho h}^{(0)} &=& \frac{(\rho h)^{n+1,(0)} - (\rho h)^n}{\Delta t} - Q_{\rho h}^{(0)}.
\end{eqnarray}

{\bf Step 3:} ({\it Update advection velocities})\\ \\
First, compute $S^{n+\myhalf}$ and $\beta_0^{n+\myhalf}$ using
\begin{equation}
S^{n+\myhalf} = \frac{S^n + S^{n+1,(0)}}{2}, \qquad \beta_0^{n+\myhalf} = \frac{\beta_0^n + \beta_0^{n+1,(0)}}{2}.
\end{equation}
Then, project $\uadvtwodag$ to obtain $\uadvtwo$, which satisfies
\begin{equation}
\nabla\cdot\left(\beta_0^{n+\myhalf}\uadvtwo\right) = S^{n+\myhalf}.
\end{equation}

{\bf Step 4:} ({\it Corrector Loop})\\ \\
We loop over this step from $j=1,j_{\rm max}$ times.  In the corrector, we use the time-advanced state from the predictor to perform a more accurate integration of the thermodynamic variables.  The quantities $(S, \beta_0, k_{\rm th}, c_p, \xi_k, h_p)^{n+1,(j-1)}$ are computed from $(\rho,\rho X_k,\rho h)^{n+1,(j-1)}$.  This step is divided into several sub-steps:\\

{\bf Step 4A:} ({\it Compute advective flux divergences})\\ \\
Use $\uadvtwo$ and a second-order Godunov integrator to compute time-centered edge states, $(\rho X_k, \rho h)^{n+\myhalf}$, with iteratively-lagged reactions ($I^{(j-1)}$), explicit diffusion, and time-centered thermodynamic pressure as source terms.  Define the advective flux divergences as
\begin{eqnarray}
A_{\rho X_k}^{(j)} &=& -\nabla\cdot\left[\left(\rho X_k\right)^{n+\myhalf,(j)}\uadvtwo\right],\\
A_{\rho h}^{(j)} &=& -\nabla\cdot\left[\left(\rho h\right)^{n+\myhalf,(j)}\uadvtwo\right] + \uadvtwo\cdot\nabla p_0.
\end{eqnarray}
Then, compute preliminary, time-advanced species using
\begin{equation}
\frac{\rho^{n+1}\widehat{X}_k^{n+1,(j)} - (\rho X_k)^n}{\Delta t} = A_{\rho X_k}^{(j)} + I_{\rho X_k}^{(j-1)}.\label{eq:sdc species 3}
\end{equation}

{\bf Step 4B:} ({\it Compute diffusive flux divergence})\\ \\
Solve a backward-Euler-type correction equation for $\widehat{h}^{n+1,(j)}$,
\begin{eqnarray}
\frac{\rho^{n+1}\widehat{h}^{n+1,(j)} - (\rho h)^n}{\Delta t} &=& A_{\rho h}^{(j)} + I_{\rho h}^{(j-1)}\nonumber\\
&& + \nabla\cdot\frac{\kth^{n+1,(j-1)}}{c_p^{n+1,(j-1)}}\nabla\widehat{h}^{n+1,(j)} + \frac{1}{2}\left(\nabla\cdot\frac{\kth^n}{c_p^n}\nabla h^n - \nabla\cdot\frac{\kth^{n+1,(j-1)}}{c_p^{n+1,(j-1)}}\nabla h^{n+1,(j-1)}\right)\nonumber\\
&& - \frac{1}{2}\left(\sum_k\nabla\cdot\frac{\xi_k^n k_{\rm th}^n}{c_p^n}\nabla X_k^n + \sum_k\nabla\cdot\frac{\xi_k^{n+1,(j-1)} k_{\rm th}^{n+1,(j-1)}}{c_p^{n+1,(j-1)}}\nabla\widehat{X}_k^{n+1,(j)}\right)\nonumber\\
&& - \frac{1}{2}\left(\nabla\cdot\frac{h_p^n k_{\rm th}^n}{c_p^n}\nabla p_0 + \nabla\cdot\frac{h_p^{n+1,(j-1)}k_{\rm th}^{n+1,(j-1)}}{c_p^{n+1,(j-1)}}\nabla p_0\right),\label{eq:sdc enthalpy 3}
\end{eqnarray}

{\bf Step 4C:} ({\it Advance thermodynamic variables})\\ \\
Define $Q_{\rho X_k}^{(j)}$ as the right hand side of (\ref{eq:sdc species 3}) without the $I_{\rho X_k}^{(j-1)}$ term, and define $Q_{\rho h}^{(j)}$ as the right hand side of (\ref{eq:sdc enthalpy 3}) without the $I_{\rho h}^{(j-1)}$ term.  Use {\tt VODE} to integrate (\ref{eq:species}) and (\ref{eq:enthalpy}) over $\Delta t$ to advance $(\rho X_k, \rho h)^n$ to $(\rho X_k, \rho h)^{n+1,(j)}$ using the piecewise-constant advection and diffusion source terms:
\begin{eqnarray}
\frac{\partial(\rho X_k)}{\partial t} &=& Q_{\rho X_k}^{(j)} + \rho\dot\omega_k\\
\frac{\partial(\rho h)}{\partial t} &=& Q_{\rho h}^{(j)} + \rho\Hnuc.
\end{eqnarray}
At this point we can define $I_{\rho X_k}^{(j)}$, $I_{\rho h}^{(j)}$, and any other $I$ terms we need depending on 
our species and enthalpy edge state prediction types, for use in the predictor in the next time step.  In our first implementation, we are predicting $\rho X_k$ and $\rho h$, in which case we define:
\begin{eqnarray}
I_{\rho X_k}^{(j)} &=& \frac{(\rho X_k)^{n+1,(j)} - (\rho X_k)^n}{\Delta t} - Q_{\rho X_k}^{(j)}\\
I_{\rho h}^{(j)} &=& \frac{(\rho h)^{n+1,(j)} - (\rho h)^n}{\Delta t} - Q_{\rho h}^{(j)}.
\end{eqnarray}

{\bf Step 5:} ({\it Advance velocity and dynamic pressure})\\ \\
Similar to the original MAESTROeX algorithm, we first compute the face-centered, time-centered velocities, $\Ub^{\nph,\pred}$
using a Godunov approach \citep{XRB_III}. Then, we update
the velocity field $\Ub^n$ to $\Ub^{n+1,\dagger}$ by discretizing
equation (\ref{eq:momentum}) as
\begin{equation}
\Ub^{n+1,\dagger}
= \Ub^n - \dt \left[\uadvtwo \cdot \nabla \Ub^{\nph,\pred} \right]
 - \dt \left[ \frac{\beta_0^\nph}{\rho^\nph} \nabla \left( \frac{\pi^\nmh}{\beta_0^\nmh} \right) + \frac{\left(\rho^\nph-\rho_0^\nph\right)}{\rho^\nph} g^{\nph} \eb_r \right],
\end{equation}
where
\begin{equation}
\rho^\nph = \frac{\rho^n + \rho^{n+1}}{2}, \qquad \rho_0^\nph = \frac{\rho_0^n + \rho_0^{n+1}}{2}.
\end{equation}
Again, the $\dagger$ superscript refers
to the fact that the updated velocity does not satisfy the divergence constraint,
\begin{equation}
\nabla \cdot \left(\beta_0^{n+1} \Ub^{n+1} \right) = \beta_0^{n+1} \left[ S^{n+1} - \frac{1}{\gammaonebar^{n+1}p_0^{n+1}}\left(\frac{\partial p_0}{\partial t}\right)^{\nph}\right].\label{eq:div3}
\end{equation}
We use an approximate projection to project $\Ub^{n+1,\dagger}$ onto the space of velocities that satisfy the constraint to obtain $\Ub^{n+1}$ using a ``nodal'' projection.
This projection necessarily differs from the MAC projection used in
{\bf Step 1} and {\bf Step 3} because the velocities in those steps are defined
on faces and $\Ub^{n+1}$ is defined at cell centers, requiring different divergence
and gradient operators..


\section{Numerical Algorithm} \label{sec:algorithm}


\section{Results} \label{sec:results}


%% For this sample we use BibTeX plus aasjournals.bst to generate the
%% the bibliography. The sample63.bib file was populated from ADS. To
%% get the citations to show in the compiled file do the following:
%%
%% pdflatex sample63.tex
%% bibtext sample63
%% pdflatex sample63.tex
%% pdflatex sample63.tex

\bibliography{references}{}
\bibliographystyle{aasjournal}

%% This command is needed to show the entire author+affiliation list when
%% the collaboration and author truncation commands are used.  It has to
%% go at the end of the manuscript.
%\allauthors

%% Include this line if you are using the \added, \replaced, \deleted
%% commands to see a summary list of all changes at the end of the article.
%\listofchanges
\listofchanges

\end{document}

% End of file `sample63.tex'.
