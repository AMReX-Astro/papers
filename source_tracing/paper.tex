\documentclass[times,modern]{aastex63}

% these lines seem necessary for pdflatex to get the paper size right
\pdfpagewidth 8.5in
\pdfpageheight 11.0in

\usepackage[T1]{fontenc}
\usepackage{epsf,color,amsmath}

\usepackage{cancel}

\input mysymbols

\newcommand{\Sb}{{\bf S}}

\newcommand{\sfrac}[2]{\mathchoice%
  {\kern0em\raise.5ex\hbox{\the\scriptfont0 #1}\kern-.15em/
    \kern-.15em\lower.25ex\hbox{\the\scriptfont0 #2}}
  {\kern0em\raise.5ex\hbox{\the\scriptfont0 #1}\kern-.15em/
    \kern-.15em\lower.25ex\hbox{\the\scriptfont0 #2}}
  {\kern0em\raise.5ex\hbox{\the\scriptscriptfont0 #1}\kern-.2em/
    \kern-.15em\lower.25ex\hbox{\the\scriptscriptfont0 #2}} {#1\!/#2}}

\newcommand{\myhalf}{{\sfrac{1}{2}}}


\newcommand{\castro}{{\sf Castro}}
\newcommand{\maestro}{{\sf Maestro}}
\newcommand{\flash}{{\sf Flash}}
\newcommand{\amrex}{{\sf AMReX}}

\newcommand{\isot}[2]{$^{#2}\mathrm{#1}$}
\newcommand{\isotm}[2]{{}^{#2}\mathrm{#1}}

\newcommand{\gcc}{\mathrm{g~cm^{-3} }}
\newcommand{\cms}{\mathrm{cm~s^{-1} }}


\usepackage{bm}


\setlength{\marginparwidth}{0.75in}
\newcommand{\MarginPar}[1]{\marginpar{\vskip-\baselineskip\raggedright\tiny\sffamily\hrule\smallskip{\color{red}#1}\par\smallskip\hrule}}

\begin{document}
%======================================================================
% Title
%======================================================================
%\title{A Simplified Spectral Deferred Correction Method for Coupling Hydrodynamics with Reaction Networks and Nuclear Statistical Equilibrium}
\title{Source Tracing in PPM}

%\shorttitle{A Simplified SDC Method}
\shorttitle{Source Tracing in PPM}


\shortauthors{}

%% \author[0000-0001-8401-030X]{M.~Zingale}
%% \affiliation{Dept.\ of Physics and Astronomy, Stony Brook University,
%%        Stony Brook, NY 11794-3800}

%% \author[0000-0003-0439-4556]{M.~P.~Katz}
%% \affiliation{Nvidia Corp}

%% \author[0000-0002-5749-334X]{J.~B.~Bell}
%% \affiliation{Center for Computational Sciences and Engineering, Lawrence Berkeley National Laboratory, Berkeley, CA  94720}

%% \author[0000-0003-1791-0265]{A.~J.~Nonaka}
%% \affiliation{Center for Computational Sciences and Engineering, Lawrence Berkeley National Laboratory, Berkeley, CA  94720}

%% \author[0000-0001-8092-1974]{W.~Zhang}
%% \affiliation{Center for Computational Sciences and Engineering, Lawrence Berkeley National Laboratory, Berkeley, CA  94720}

%% \correspondingauthor{Michael Zingale}
%% \email{michael.zingale@stonybrook.edu}


%======================================================================
% Abstract and Keywords
%======================================================================
\begin{abstract}
\end{abstract}

\keywords{hydrodynamics---methods: numerical}

%======================================================================
% Introduction
%======================================================================
\section{Introduction}\label{Sec:Introduction}

Consider the primitive variable system
\begin{equation}
\qb_t + \Ab(\qb) = \Sb(\qb)
\end{equation}
where $\Sb$ are the source terms to the primitive variables.

Recall that for the piecewise linear method, we predict states to the interface, $i+\myhalf$,
and the midpoint in time $n+\myhalf$.
\begin{align}
\qb_{i+\myhalf,L}^{n+\myhalf} &= \qb_i^n +
    \left . \frac{\Delta x}{2} \frac{\partial \qb}{\partial x} \right |_i +
    \frac{\Delta t}{2} \underbrace{\left .\frac{\partial \qb}{\partial t} \right |_i}_{= -\Ab \partial \qb / \partial x + \Sb} + \ldots \\
&= \qb_i^n + \frac{\Delta x}{2} \left . \frac{\partial \qb}{\partial x} \right |_i
          + \frac{\Delta t}{2} \left ( -\Ab \frac{\partial \qb}{\partial x} + \Sb \right )_i\\
&= \qb_i^n + \frac{1}{2} \left [ 1 - \frac{\Delta t}{\Delta x} \Ab_i \right ] \Delta \qb_i + \frac{\Delta t}{2} \Sb_i \label{eq:taylorstate}
\end{align}
Using a characteristic projection, we recognize:
\begin{equation}
\Ab \Delta \qb = \sum_\nu (\lb^\enu \cdot \Delta \qb) \lambda^\enu \rb^\enu
\end{equation}
We can modify the sum to only include waves moving toward the interface, then
\begin{equation}
\qb_{i+\myhalf,L}^{n+\myhalf} = \qb_i^n + \frac{1}{2} \sum_{\nu,\lambda^\enu > 0}
   \left (1 - \frac{\Delta t}{\Delta x} \lambda_i^\enu \right )
   (\lb_i^\enu \cdot \Delta \qb_i) \rb_i^\enu + \frac{\Delta t}{2} \Sb_i
\end{equation}

What about including the source in the characteristic projection?  We can write
\begin{equation}
\Sb = \sum_\nu (\lb^\enu \cdot \Sb) \rb^\enu
\end{equation}
then we have:
\begin{equation}
\qb_{i+\myhalf,L}^{n+\myhalf} = \qb_i^n + \frac{1}{2} \sum_{\nu,\lambda^\enu > 0}
   \lb^\enu \cdot \left [ \left (1 - \frac{\Delta t}{\Delta x} \lambda_i^\enu \right )
   \Delta \qb_i + \Delta t \Sb_i \right ] \rb_i^\enu
\end{equation}

Instead of this, we could introduce a reference state before the projection, which has the
effect of minimizing the jumps seen by the linearization in the projection operation.
If we consider the difference between $\qb_{i+\myhalf,L}^{n+\myhalf}$ and a reference state, $\qb_\refs$ as:
\begin{equation}
\qb_{i+\myhalf,L}^{n+\myhalf}  - \qb_\refs = 
  \qb_i^n - \qb_\refs + \frac{1}{2} \left [ \left (1 - \frac{\Delta t}{\Delta x} \Ab_i \right ) \Delta \qb_i + \Delta t \Sb_i \right ]
\end{equation}
and then project the RHS, we have
\begin{equation}
\qb_{i+\myhalf,L}^{n+\myhalf}  = \qb_\refs - \sum_{\nu,\lambda^\enu > 0} \lb_i^\enu \cdot \left \{
  \qb_\refs - \qb_i^n - \frac{1}{2} \left [ \left (1 - \frac{\Delta t}{\Delta x} \lambda_i^\enu \right ) \Delta \qb_i - \Delta t \Sb_i \right ] \right \} \rb^\enu \label{eq:plm_project}
\end{equation}

Now PPM recognizes that the this construction looks like the integral under $\qb(x)$ that
can reach the interface in the timestep \citep{millercolella:2002},
\begin{equation}
\qb_i + \frac{1}{2} \left (1 - \frac{\Delta x}{\Delta t} \lambda^\enu \right ) \Delta \qb \approx
  \frac{1}{\lambda^\enu \Delta t} \int_{x_{i+\myhalf} - \lambda^\enu \Delta t}^{x_{i+\myhalf}} \qb(x) dx \equiv \Ic_+^\enu(\qb)
\end{equation}
(This is exact for piecewise linear reconstruction).  For PPM, the interface state is then
written as:
\begin{equation}
\qb_{i+\myhalf,L}^{n+\myhalf}  = \qb_\refs - \sum_{\nu,\lambda^\enu > 0} \lb_i^\enu \cdot \left  [
  \qb_\refs - \Ic_+^\enu(\qb) - \Delta t \Ic_+^\enu(\Sb) \right ] \rb^\enu
\end{equation}
Note, this is the form that \citet{ppm} suggest, we'll revisit the
$\Delta t$ factor in front of the source term shortly.

Let's see how this PPM form reduces when we assume a piecewise linear profile for $\qb$ and $\Sb$:
\begin{align}
\qb(x) &= \qb_i + \frac{\Delta \qb_i}{\Delta x} (x - x_i) \\
\Sb(x) &= \Sb_i + \frac{\Delta \Sb_i}{\Delta x} (x - x_i) 
\end{align}
Putting these forms in the integral, we get:
\begin{equation}
\qb_{i+\myhalf,L}^{n+\myhalf}  = \qb_\refs - \sum_{\nu,\lambda^\enu > 0} \lb_i^\enu \cdot \left  [
  \qb_\refs - \qb_i - \frac{1}{2} \left ( 1 - \lambda_i^\enu \frac{\Delta t}{\Delta x} \right ) \Delta \qb_i
           - \Delta t \Sb_i - \frac{\Delta t}{2} \left ( 1 - \lambda_i^\enu \frac{\Delta t}{\Delta x} \right ) \Delta \Sb_i \right ] \rb_i^\enu
\end{equation}
This almost looks like Eq.~\ref{eq:plm_project}, but there are two
differences: (1) the PPM form has a term proportional to $\Delta
\Sb_i$ that captures the variation of the source term over the cell
and (2) the PPM form has a full $\Delta t$ in front of the source term
instead of $\Delta t/2$.  We believe that it should be $\Delta t/2$ (as also discussed in \citealt{wdmergerI}),
which would make the correct PPM interface state with source
projection:
\begin{equation}
\qb_{i+\myhalf,L}^{n+\myhalf}  = \qb_\refs - \sum_{\nu,\lambda^\enu > 0} \lb_i^\enu \cdot \left  [
  \qb_\refs - \Ic_+^\enu(\qb) - \frac{\Delta t}{2} \Ic_+^\enu(\Sb) \right ] \rb^\enu
\end{equation}
Now there is not a mathematical reason to include the $\Delta \Sb$
variation, but physically it makes sense to consider only the
acceleration that a quantity will see over a timestep as it approaches
the interface.

Now we can choose the reference state.  In the PPM paper, the choice was the fastest wave
moving toward the interface,
\begin{equation}
\qb_\refs = \Ic_+^\evp(\qb)
\end{equation}
with the justification that the choice of reference state should
reduce the size of the jump that the linearized projection works on.
But if the source terms are large, a better choice is:
\begin{equation}
\qb_\refs = \Ic_+^\evp(\qb) + \frac{\Delta t}{2} \Ic_+^\evp(\Sb)
\end{equation}

%======================================================================
% References
%======================================================================

\bibliographystyle{aasjournal}
\bibliography{ws}

\end{document}
