\documentclass[times,preprint]{aastex63}

% these lines seem necessary for pdflatex to get the paper size right
\pdfpagewidth 8.5in
\pdfpageheight 11.0in

\usepackage[T1]{fontenc}
\usepackage{epsf,color,amsmath}

\usepackage{cancel}

\newcommand{\sfrac}[2]{\mathchoice%
  {\kern0em\raise.5ex\hbox{\the\scriptfont0 #1}\kern-.15em/
    \kern-.15em\lower.25ex\hbox{\the\scriptfont0 #2}}
  {\kern0em\raise.5ex\hbox{\the\scriptfont0 #1}\kern-.15em/
    \kern-.15em\lower.25ex\hbox{\the\scriptfont0 #2}}
  {\kern0em\raise.5ex\hbox{\the\scriptscriptfont0 #1}\kern-.2em/
    \kern-.15em\lower.25ex\hbox{\the\scriptscriptfont0 #2}} {#1\!/#2}}


\newcommand{\castro}{{\sf Castro}}
\newcommand{\maestro}{{\sf Maestro}}
\newcommand{\flash}{{\sf Flash}}
\newcommand{\amrex}{{\sf AMReX}}

\newcommand{\isot}[2]{$^{#2}\mathrm{#1}$}
\newcommand{\isotm}[2]{{}^{#2}\mathrm{#1}}

\newcommand{\gcc}{\mathrm{g~cm^{-3} }}
\newcommand{\cms}{\mathrm{cm~s^{-1} }}

\newcommand{\nablab}{{\mathbf{\nabla}}}
\newcommand{\Ub}{\mathbf{U}}
\newcommand{\gb}{\mathbf{g}}
\newcommand{\omegadot}{\dot{\omega}}
\newcommand{\Sdot}{\dot{S}}
\newcommand{\ddx}[1]{{\frac{{\partial#1}}{\partial x}}}
\newcommand{\ddt}[1]{{\frac{{\partial#1}}{\partial t}}}
\newcommand{\odt}[1]{{\frac{{d#1}}{dt}}}
\newcommand{\divg}[1]{{\nablab \cdot \left (#1\right)}}
\newcommand{\dedr}{\left . {\partial{}e}/{\partial\rho}\right |_{T, X_k}}
\newcommand{\dedrd}{\left . \frac{\partial{}e}{\partial\rho}\right |_{T, X_k}}
\newcommand{\dedX}{\left . {\partial{}e}/{\partial{}X_k} \right |_{\rho, T}}
\newcommand{\dedXd}{\left . \frac{\partial{}e}{\partial{}X_k} \right |_{\rho, T, X_{j,j\ne k}}}
\newcommand{\dedT}{\left . {\partial{}e}/{\partial{}T} \right |_{\rho,X_k}}
\newcommand{\dedTd}{\left . \frac{\partial{}e}{\partial{}T} \right |_{\rho,X_k}}

\newcommand{\Ic}{{\boldsymbol{\mathcal{I}}}}
\newcommand{\Ics}{{\mathcal{I}}}
\newcommand{\smax}{{s_\mathrm{max}}}
\newcommand{\kth}{k_\mathrm{th}}
\usepackage{bm}

\newcommand{\Uc}{{\,\bm{\mathcal{U}}}}
\newcommand{\Fb}{\mathbf{F}}
\newcommand{\Sc}{\mathbf{S}}

\newcommand{\xv}{{(x)}}
\newcommand{\yv}{{(y)}}
\newcommand{\zv}{{(z)}}

\newcommand{\ex}{{\bf e}_x}
\newcommand{\ey}{{\bf e}_y}
\newcommand{\ez}{{\bf e}_z}

\newcommand{\Ab}{{\bf A}}
\newcommand{\Sq}{{\bf S}_\qb}
\newcommand{\Sqhydro}{{\Sq^{\mathrm{hydro}}}}
\newcommand{\qb}{{\bf q}}

\newcommand{\Shydro}{{{\bf H}}}
\newcommand{\Hb}{{\bf H}}
\newcommand{\Rb}{{\bf R}}
\newcommand{\Rq}{{\bf R}}
\newcommand{\Adv}[1]{{\left [\boldsymbol{\mathcal{A}} \left(#1\right)\right]}}
\newcommand{\Advt}[1]{{\left [\boldsymbol{\mathcal{\tilde{A}}} \left(#1\right)\right]}}
\newcommand{\Advss}[1]{{\left [{\mathcal{{A}}} \left(#1\right)\right]}}
\newcommand{\Advs}[1]{\boldsymbol{\mathcal{A}} \left(#1\right)}

\newcommand{\avg}[1]{{\left \langle #1 \right \rangle}}

\newcommand{\nse}[1]{{\mathrm{NSE}( #1 )}}

\newcommand{\rhonse}{{\rho_\mathrm{nse}}}
\newcommand{\tnse}{{T_\mathrm{nse}}}
\newcommand{\Anse}{{A_\mathrm{nse}}}
\newcommand{\Bnse}{{B_\mathrm{nse}}}

\newcommand{\out}{{\rm out}}
\newcommand{\inp}{{\rm in}}

\setlength{\marginparwidth}{0.75in}
\newcommand{\MarginPar}[1]{\marginpar{\vskip-\baselineskip\raggedright\tiny\sffamily\hrule\smallskip{\color{red}#1}\par\smallskip\hrule}}

\begin{document}
%======================================================================
% Title
%======================================================================
%\title{A Simplified Spectral Deferred Correction Method for Coupling Hydrodynamics with Reaction Networks and Nuclear Statistical Equilibrium}
\title{Practical Effects of Integrating Temperature with Strang Split Reactions}

%\shorttitle{A Simplified SDC Method}
\shorttitle{Temperature Integration with Strang}


\shortauthors{}

%% \author[0000-0001-8401-030X]{M.~Zingale}
%% \affiliation{Dept.\ of Physics and Astronomy, Stony Brook University,
%%        Stony Brook, NY 11794-3800}

%% \author[0000-0003-0439-4556]{M.~P.~Katz}
%% \affiliation{Nvidia Corp}

%% \author[0000-0002-5749-334X]{J.~B.~Bell}
%% \affiliation{Center for Computational Sciences and Engineering, Lawrence Berkeley National Laboratory, Berkeley, CA  94720}

%% \author[0000-0003-1791-0265]{A.~J.~Nonaka}
%% \affiliation{Center for Computational Sciences and Engineering, Lawrence Berkeley National Laboratory, Berkeley, CA  94720}

%% \author[0000-0001-8092-1974]{W.~Zhang}
%% \affiliation{Center for Computational Sciences and Engineering, Lawrence Berkeley National Laboratory, Berkeley, CA  94720}

%% \correspondingauthor{Michael Zingale}
%% \email{michael.zingale@stonybrook.edu}


%======================================================================
% Abstract and Keywords
%======================================================================
\begin{abstract}
For astrophysical reacting flows, operator splitting is commonly used
to couple hydrodynamics and reactions.  Each process operates
independent of one another, but by staggering the updates in a
symmetric fashion (via Strang splitting) second order accuracy in time
can be achieved.  However, approximations are often made to the
reacting system, including the choice of whether or not to integrate
temperature with the species.  Here we demonstrate through a simple
convergence test that integrating temperature together with reactions
achieves the best convergence when modeling reactive flows with Strang
splitting.
\end{abstract}

\keywords{hydrodynamics---methods: numerical}

%======================================================================
% Introduction
%======================================================================
\section{Introduction}\label{Sec:Introduction}

Simulations of stellar flows require solving the equations of
hydrodynamics coupled with a nuclear reaction network.  The equations
of hydrodynamics with reacting sources are:
\begin{align}
\frac{\partial \rho}{\partial t} + \nabla \cdot (\rho \Ub) &= 0 \\
\frac{\partial (\rho X_k)}{\partial t} + \nabla \cdot (\rho \Ub X_k) &= \rho \omegadot_k \\
\frac{\partial (\rho \Ub)}{\partial t} + \nabla \cdot (\rho \Ub \Ub) + \nabla p &= 0 \\
\frac{\partial (\rho E)}{\partial t} + \nabla \cdot \left [ (\rho E + p) \Ub \right ] &= \rho \Sdot
\end{align}
where $\rho$ is the density, $\Ub$ is the velocity vector, $X_k$ are the species mass fractions,
$\omegadot_k$ are the species creation rates (from reactions), $p$ is the pressure, $E$ is the
total energy / unit mass, and $\Sdot$ is the nuclear energy generation rate.

When we are reacting, we need to evolve the temperature, which we can write as:
\begin{equation}
\rho c_v \frac{DT}{Dt} = \rho \left (\frac{p}{\rho^2} - e_\rho \right ) \frac{D\rho}{Dt} + \rho \Sdot
\end{equation}
(this form neglects the change in temperature due to composition
changes from reactions, see \citealt{ABNZ:III}, which assumes chemical
equilibrium). 


\section{Numerical Methodology}

We are using the freely-available \castro\ simulation
code~\citep{castro,castro_joss}.  \castro\ solves the equations of
hydrodynamics using an unsplit piecewise parabolic method coupled with
gravity and reactions.  For reactive flow, either Strang splitting or
spectral deferred corrections are used to couple the hydrodynamics and
reactions~\citep{castro_sdc}.  Here we focus on the Strang splitting.

In a Strang split evolution~\cite{strang:1968}, we update the full
hydrodynamics state, $\Uc$ as:
\begin{equation}
  \Uc^{n+1} = {\bf R}_{\Delta t/2} {\bf A}_{\Delta t} {\bf R}_{\Delta t/2} \Uc^n
\end{equation}
where ${\bf R}_{\Delta t/2}$ is the reaction update through a timestep
$\Delta t/2$ and ${\bf A}_{\Delta t}$ is the advective update through
$\Delta t$.  We see with this splitting, each process operates on the
state left behind by the previous operation, and the staggering of the
physics is done to give second order accuracy in time.

During reactions, we neglect the hydrodynamics terms,
so the reactive system updates according to:
\begin{align}
\frac{D\rho}{Dt} &= 0 \\
\frac{DX_k}{Dt} &= \omegadot_k \label{eq:species} \\
\frac{DT}{Dt} &= \frac{\Sdot}{c_v}
\end{align}


There are several different approximations that people make in the
literature to this operator-split reacting system, in order to make
integrating the reaction system computationally less expensive:
\begin{itemize}
\item Neglect temerature evolution completely in the burn, only
  evolving Eq.~\ref{eq:species}.  This is the method used in
  \citet{flash}. \MarginPar{other codes?}
\item Evolve the temperature equation, but ``freeze'' the specific
  heat, $c_v$, at the start of integration to avoid potentially
  expensive equation of state calls.  This was discussed in
  \cite{Bell:2004} and is the default method in \castro.
\item Evolve the full system, calling the equation of state each time
  we evaluate the righthand side of the system to get an updated
  $c_v$.  This is an option in \castro.
\end{itemize}
Depending on how vigorous the burning is and how much the temperature
changes during a hydrodynamic timestep, one or more of these
approximations may be reasonable.  For explosive reactions, we expect
that evolving the full system will be required.  The goal of this
note is to try to quantify the convergence of a reacting hydrodynamics
code with these different approximations.

In \citet{castro_sdc}, we introduced a test problem where we could
measure the convergence of a reacting hydrodynamics problem via
Richardson extrapolation\MarginPar{ref?}---this was an acoustic pulse with helium
burning via $3$-$\alpha$ and
$\isotm{C}{12}(\alpha,\gamma)\isotm{O}{16}$.  Those tests focused on
SDC coupling and showed that we can get overall 4th order in space and
time convergence.  Here we run the same test with Strang coupling,
looking at the various approaches at incorporating a temperature
equation in the reactive portion of the update.

For each method, we run the \castro\ {\tt reacting\_convergence} test
problem at 5 resolutions: $64^2$, $128^2$, $256^2$, $512^2$, and
$1024^2$, with the timestep kept fixed in proportion to the grid resolution.
We then compute four errors between adjacent resolutions by
coarsening the finer resolution run, and computing the $L_1$ norm over
all zones.  Figure~\ref{thefigure} shows the norm of the error
vs.\ the coarse run resolution.  Guidelines showing ideal first and
second order convergence are shown.

\begin{figure}[t]
\centering
\plotone{test}
\caption{\label{thefigure} our results}
\end{figure}



\section{Summary}

Looking at global convergence of a reacting hydrodynamics problem we
see that second order convergence is only realized when temperature is
evolved alongside the composition when using an operator-split
approach to reactions.  This is just a single, rather simple problem,
but this suggests that reactive hydrodynamics simulations should
switch to integrating temperature or another energy equation together
with reactions to yield better overall convergence and accuracy.  We
expect that for problems with vigorous burning, such as detonation,s
this will be especially important.

\acknowledgements \castro\ is freely available at
\url{http://github.com/AMReX-Astro/Castro}.  All of the code and
problem setups used here are available in the git repo.  The work at
Stony Brook was supported by DOE/Office of Nuclear Physics grant
DE-FG02-87ER40317.  This material is based upon work supported by the
U.S. Department of Energy, Office of Science, Office of Advanced
Scientific Computing Research and Office of Nuclear Physics,
Scientific Discovery through Advanced Computing (SciDAC) program under
Award Number DE-SC0017955.  This research was supported by the
Exascale Computing Project (17-SC-20-SC), a collaborative effort of
the U.S. Department of Energy Office of Science and the National
Nuclear Security Administration.

\software{\amrex\ \citep{amrex_joss},
          \castro\ \citep{castro},
          GNU Compiler Collection (\url{https://gcc.gnu.org/}),
          Linux (\url{https://www.kernel.org}),
          matplotlib (\citealt{Hunter:2007},  \url{http://matplotlib.org/})
          NumPy \citep{numpy,numpy2},
          python (\url{https://www.python.org/}),
          VODE \cite{vode}
         }




%======================================================================
% References
%======================================================================

\bibliographystyle{aasjournal}
\bibliography{ws}

\end{document}
