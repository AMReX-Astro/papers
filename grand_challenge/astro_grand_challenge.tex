\documentclass[11pt]{article}

\usepackage[margin=1in]{geometry}

\usepackage{hyperref}
\usepackage{times}

\usepackage[para]{footmisc}

\usepackage{fancyhdr}
\pagestyle{fancy}
\renewcommand{\headrulewidth}{0pt}

\rfoot{\footnotesize \em Future Directions in Extreme Scale\\ Computing for Scientific Grand Challenges}

\begin{document}

\begin{center}
{\large Challenges in Modeling Astrophysical Thermonuclear Explosions} \\[0.25em]
Michael Zingale (Stony Brook)
\end{center}

The end states of stellar evolution result in exotic compact objects:
white dwarfs, neutron stars, and black holes.  When these remnants are
in binary systems, interactions with the companion star can lead to a
variety of different thermonuclear explosions: classical novae, X-ray
bursts (XRBs), and thermonuclear (Type Ia) supernovae (SN Ia).  SN Ia
and novae are important sites of nucleosynthesis in the galaxy; XRBs
allow us to probe the physics of dense nuclear matter; and SN Ia are
important distance indicators for measuring the expansion history of
the Universe.  Simulations on these events have been ongoing for
decades and allow us to understand the physics of the explosions,
interpret the observations, and test different progenitor models for
viability.  Next generation supercomputers will allow us to push these
simulations to unprecedented realism.

Modeling these events requires multiscale and multiphysics simulations.
Spatial scales range from the size of the compact object
down to the burning or dissipation scale.  Temporal scales range from
the millions of years (SN Ia) to hours (XRBs) of evolution that
brought the star to the brink of explosion to the seconds-long
explosion.  Models must incorporate hydrodynamics, nuclear reactions,
gravity, thermal conduction, magnetic fields, and radiation.

While a variety of approximations are used in astrophysics to model
hydrodynamic flows, for stellar explosions, conservative finite-volume
methods are the most popular.  Adaptive mesh refinement (AMR) allows
us to increase the range of length scales captured.  Our group develops
the AMReX Astrophysics Suite of open-source AMR
multiphysics hydrodynamics
codes\footnote{\url{https://github.com/AMReX-Astro/}}.  Two codes in particular address stellar explosions: MAESTROeX
is a low Mach number hydrodynamics code for modeling the highly
subsonic stellar convection leading up to explosion and Castro is a
fully compressible radiation hydrodynamics code that efficiently
models the explosive phase.  These codes share a common framework
(AMR library, microphysics routines), are written in C++ and Fortran,
and use MPI+OpenMP or MPI+CUDA.

Advances in our understanding of these events have been made both
through the development of new algorithms (e.g.\ low-speed solvers) as
well as new computer architectures.  Today's supercomputers are
dominated by GPU accelerators, as will the next generation of
DOE supercomputers.  Over the past year,
we've ported the AMReX Astrophysics Suite to GPUs, enabling a large
performance gain and opening new areas for scientific exploration.  An
important part of that process is performance portability.  Our path
allows us to use the exact same compute kernels on CPUs and GPUs,
eliminating the burden of maintaining redundant code paths for
different architectures.

To highlight the way that new computer architectures enable new
science, we focus on our current work on XRBs.  Our interest is in
modeling the nucleosynthesis as a burning front propagates laterally
through the accreted fuel layer on a neutron star.  Due to the range
of length and time scales, various approximations have been used in the
literature.  Our group started performing 2D
hydrodynamical simulations of the flame propagating using the NERSC
Cori and OLCF Titan (on CPUs)  machines over a year ago.  These calculations were
the first to perform full hydrodynamics in both the lateral and
vertical directions, allowing us to understand the complex dynamics of
the flame front with realistic microphysics.  The simulations were
challenging, and we needed to make approximations to both the length
scales (using artificially high neutron star rotation rates to compress
the lateral scales) and time scales (boosting the flame speed).
  Taking advantage of GPUs,
we are now able to eliminate these approximations in our current runs on the
OLCF Summit machine.  The same code is used,
but running entirely on the GPUs, it is 5--10$\times$ faster than
CPUs.  We are on target to perform the first three-dimensional
simulations this year as well as to explore the more interesting case
of mixed H/He bursts, which involve more complicated nuclear
reactions.

The present generation of supercomputers allow us to do simulations
that were simply not possible with the previous generation.  Future
computers will give the same boost.  With the next generation
machines, we will be able to routinely do 3D simulations, cover a
larger portion of the neutron star surface, increase the size of our
reaction networks, and couple radiation.These will be the first global
simulations of burning fronts on neutron stars that capture the
realistic physics and allow us to connect directly to observations.

New architectures bring new programming challenges, and to date we've
managed to keep pace with these developments.  A key point is that
funding needs to be provided to aid in preparing codes, not just for
the science, especially for novel architectures.  NSF should recognize
this need.  The software stack on the next generation machine is as
important as the hardware.  It should support the popular
community-supported programming models today and have wide support for
C/C++ and Fortran.  To ensure the largest benefit to the community,
application codes for these machines should be Open Source to the
fullest extend allowed.


\end{document}
