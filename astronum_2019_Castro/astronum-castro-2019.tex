\documentclass[a4paper]{jpconf}

\usepackage{graphicx}

\usepackage{hyperref}

\usepackage[sort&compress,numbers]{natbib}

%\usepackage{cite}

\usepackage{amsmath}
\usepackage{amssymb}
\usepackage{bm}

\newcommand{\maestro}{{\sffamily Maestro}}
\newcommand{\castro}{{\sffamily Castro}}
\newcommand{\starkiller}{{\sffamily StarKiller}}
\newcommand{\starlord}{{\sffamily StarLord}}
\newcommand{\nyx}{{\sffamily Nyx}}
\newcommand{\amrex}{{\sffamily AMReX}}
\newcommand{\vode}{{\sffamily VODE}}

\newcommand{\Uc}{{\,\bm{\mathcal{U}}}}
\newcommand{\Adv}[1]{{\left [\boldsymbol{\mathcal{A}} \left(#1\right)\right]}}
\newcommand{\Advt}[1]{{\left [\mathcal{\tilde{A}} \left(#1\right)\right]}}
\newcommand{\Advs}[1]{\boldsymbol{\mathcal{A}} \left(#1\right)}

\newcommand{\isot}[2]{$^{#2}\mathrm{#1}$}

\newcommand{\gcc}{\mathrm{g~cm^{-3} }}
\newcommand{\cms}{\mathrm{cm~s^{-1} }}



\newcommand{\cpp}{C\nolinebreak\hspace{-.05em}\raisebox{.4ex}{\tiny\bf +}\nolinebreak\hspace{-.10em}\raisebox{.4ex}{\tiny\bf +}}

\usepackage{color}
\setlength{\marginparwidth}{0.75in}
\newcommand{\MarginPar}[1]{\marginpar{\vskip-\baselineskip\raggedright\tiny\sffamily\hrule\smallskip{\color{red}#1}\par\smallskip\hrule}}

\newcommand{\apj}{Astrophysical Journal}
\newcommand{\aap}{Astronomy and Astrophysics}
\newcommand{\mnras}{Monthly Notices of the Royal Astronomical Society}
\newcommand{\prd}{Physical Review D}
\newcommand{\apjs}{Astrophysical Journal Supplement}

\begin{document}

\title{The Castro AMR Simulation Code: Current and Future Developments}

\author{M. Zingale$^1$,
        A.~S.~Almgren$^2$,
        M.~Barrios Sazo$^1$,
        J.~B. Bell$^4$,
        K.~Eiden$^1$,
        A.~Harpole$^1$,
        M.~P. Katz$^3$,
        A.~J. Nonaka$^2$,
        D.~E. Willcox$^2$, and
        W. Zhang$^2$}

\address{$^1$Department of Physics and Astronomy, Stony Brook
  University, Stony Brook, NY 11794-3800 USA}

\address{$^2$Center for Computational Sciences and Engineering,
  Lawrence Berkeley National Lab, Berkeley, CA 94720 USA}

\address{$^3$NVIDIA Corporation, 2788 San Tomas Expressway,
  Santa Clara, CA, 95050 USA}

\ead{michael.zingale@stonybrook.edu}

\begin{abstract}
We discuss 
\end{abstract}



\section{Introduction}

The \castro\ astrophysical simulation code~\cite{castro} is designed
for modeling problems in nuclear astrophysics, with the ability to
accurately capture the interplay between hydrodynamics, reactions,
gravity, and radiation in stars with complex equations of state.
Since \castro\ was first development, there have been a number of
enhancements to the code base, expanding its applicability to a new
range of scientific problems.

\castro\ has been applied to models of Type Ia supernovae,
core-collapse supernovae, pair-instability supernova, and most
recently X-ray bursts.  A common challenge in modeling these events is
the range of length and timescales involved.  

To capture length scales, \castro\ uses adaptive mesh refinement,
through the \amrex\ library~\cite{amrex_joss}.

Approximations are usually made for multi-dimensional simulations.

\section{Modeling Reactive Flow}

The recent focus in \castro\ has been on high-fidelity simulations of
reactive flow.  In \cite{castro:sdc} we introduced a new time
integration strategy, spectral deferred corrections (SDC), that
eliminates the coupling error introduced by commonly used operator
splitting techniques (see the discussion in \cite{astronum:2018} for a
graphical illustration of splitting error).  The SDC algorithm used in
\castro\ follows the ideas of \cite{dutt:2000,minion:2003}, and uses
low order explicit advection and implicit reaction updates in a
correction equation that when applied iteratively achieves high-order
time-accuracy.  \castro\ implements both a second-order and
fourth-order method (in space and time) coupling reactions and
hydrodynamics.  At the moment, the method is limited to single levels.


\section{Performance Portability}

The original approach to parallelism in \castro\ was MPI + OpenMP,
with scaling on manycore architectures achieve using a tiling approach
to OpenMP~\cite{tiling}.  

\section{Example Application}

The SDC approach was demonstrated in \cite{castro:sdc} through a large
suite of convergence tests and a thermonuclear flame problem.  

\section{Future Developments}

MHD

Other SDC quadrature schemes / detonations

SDC multilevel + XRBs


\ack The work at Stony Brook was supported by DOE/Office of Nuclear
Physics grant DE-FG02-87ER40317 and contract 7418390 with Lawrence
Berkeley National Laboratory as part of the Exascale Compute Project
ExaStar collaboration.  This research was supported by the Exascale
Computing Project (17-SC-20-SC), a collaborative effort of the
U.S. Department of Energy Office of Science and the National Nuclear
Security Administration.  The work at LBNL was supported by the DOE
Office of Advanced Scientific Computing Research under Contract No,
DE-AC02-05CH11231.  An award of computer time was provided by the
Innovative and Novel Computational Impact on Theory and Experiment
(INCITE) program. This research used resources of the Oak Ridge
Leadership Computing Facility at the Oak Ridge National Laboratory,
which is supported by the Office of Science of the U.S. Department of
Energy under Contract No.\ DE-AC05-00OR22725.  This research used
resources of the National Energy Research Scientific Computing Center,
which is supported by the Office of Science of the U.S. Department of
Energy under Contract No.\ DE-AC02-05CH11231.  Visualizations were
done using yt~\cite{yt}.  This research has made use of NASA's
Astrophysics Data System Bibliographic Services.


%\section*{References}

\bibliographystyle{iopart-num}
\bibliography{ws}


\end{document}
